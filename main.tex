\documentclass[11pt]{article}
\usepackage[numbib]{tocbibind}
\usepackage[utf8]{inputenc}
\usepackage{lmodern}
\usepackage[T1]{fontenc}
\usepackage{natbib}
\usepackage{amsmath}
\usepackage{caption}
\usepackage{graphicx,subcaption}
\usepackage[a4paper, total={6in, 8in}]{geometry}
\setcitestyle{authoryear,open={(},close={)}} %Citation-related commands
\usepackage[labelfont=bf]{caption}
\usepackage[flushleft]{threeparttable}
%\usepackage{titlesec}
%\titleformat{\section}{\normalfont\fontsize{19pt}{21pt}\bfseries}{\thesection}{1em}{}
%\titleformat{\subsection}{\normalfont\fontsize{16pt}{18pt}\bfseries}{\thesubsection}{1em}{}

\makeatletter
\renewcommand{\maketitle}{\bgroup\setlength{\parindent}{0pt}
\begin{flushright}
  \textbf{\@title}\\
  \vspace{5mm}
  \@author\\
  \vspace{5mm}
  \@date
\end{flushright}\egroup
}
\makeatother

\renewenvironment{abstract}
 {\small
  \begin{flushleft}
  \bfseries \abstractname\vspace{-.5em}\vspace{0pt}
  \end{flushleft}
  \list{}{%
    \setlength{\leftmargin}{0mm}% <---------- CHANGE HERE
    \setlength{\rightmargin}{\leftmargin}%
  }%
  \item\relax}
 {\endlist}

\title{\large Regional Convergence in the European Union - Factors of Growth Between the Great Recession and the COVID Crisis}
\author{
        \begin{large}Jan Pintera\end{large} \\\vspace{5mm} \begin{small} Institute of Economic Studies, Faculty of Social Sciences, Charles University,\\ Prague, Czech Republic\\
        Email (corresponding author): jan.pintera@fsv.cuni.cz 
        \end{small}
}
\date{November 2023}
\def \Keywords {Club Convergence, European Regions, log $t$ test, Logistic regression}



\begin{document}

\maketitle


\thispagestyle{empty}
\begin{abstract}
In this paper, we provide a new look at convergence in the EU while focusing on development at the regional level between the Great Recession and the recent COVID crisis. We use the log $t$ convergence test by \citet{phillips2007transition} to analyze convergence in the level of income among the European regions. We identified five convergence clubs rather than supporting the overall convergence hypothesis. Furthermore, we investigated the determinants of convergence club membership using logistic regression. Our results confirmed high inequality within the member states and a shifting geographic pattern of the top performing regions, with the increasing prominence of the manufacturing core in southern Germany and the surrounding areas. We found a positive association between membership in higher clubs, research and patent activities, and specialization in manufacturing. We also confirmed the good economic performance of capital cities and the main metropolitan areas.

\bigskip


\textbf{JEL Classification:} C23, C40, R11 \\
\textbf{Keywords:}  \Keywords \\

\bigskip

\end{abstract}
\clearpage
\setcounter{page}{1}

\section{Introduction}
The widening gap between prosperous and divergent regions has attracted attention from policymakers and economists, potentially shaping a new economic paradigm that steers away from globalization, emphasizing more equitable distribution of resources and opportunities across regions \citep{rodrik_2022}. This transformative shift can be seen as a recognition of the deep socioeconomic impact of regional differences, with far-reaching consequences for political consensus. The European Union's cohesion policy strives for the convergence of its regions, yet despite substantial transfers and a prolonged history of such policies, significant income disparities persist, and the dynamics of convergence are evolving \citep{eckey2007convergence,zarotiadis2013european, iammarino2019regional}.

This article contributes by providing evidence of income convergence across the regions of the European Union between the Great Recession and the Covid-related crisis.

Previous empirical research has suggested several distinguishing patterns of convergence in the EU.  Notably, economic development in large urban areas, particularly capital cities, exhibits higher dynamism compared to other regions, fostering economic inequality within EU member states \citep{geppert2008regional}. This trend is particularly evident in the new member states, where capital cities and major urban regions rank among the EU's wealthiest by GDP per capita. However, a significant number of other regions in these states lag behind, contributing to a slow reduction in the overall disparity between the old and the new member states. This discrepancy underscores the persistence of inequality within the Central and Eastern European (CEE) countries \citep{geppert2008regional,sme2012regional, smketkowski2013regional}.

Our investigation focuses on the dynamics of regional income convergence, aiming to determine whether there is uniform convergence across all EU countries or evidence of convergence clubs—specific regions converging with each other without an overarching trend toward convergence. To explore this, we use a test by \citet{phillips2007transition}, which has been used to analyze convergence at various levels and, in the case of the EU regions, usually led to the identification of several convergence clubs \citep{bartkowska2012regional, borsi2015evolution, von2017regional}. Its design is especially suited for investigating the club convergence that can be expected among EU regions because it allows for an endogenous spectrum of transitional behavior among the regions. Furthermore, we employ logistic regression on the resulting convergence clubs, examining the contribution of a wide range of variables suggested in the economic growth literature to the current state of convergence in the EU.

This study makes two primary contributions to existing literature. First, it examines the economic convergence in the EU between two major crises (2008 - 2019) and discusses the changes in convergence dynamics leading up to the COVID-related crisis. This period was previously not discussed in its entirety. Our approach allows us to compare two important states of the European economy, 2015 as the year by which most European countries achieved full recovery after the Great Recession in terms of output per capita, and 2019 as representing a point after a sustained period of economic growth and just before the COVID outbreak. As its second contribution, this work attempts to draw a connection between various socioeconomic characteristics of EU regions and their relative economic performance.

Our results support the club convergence hypothesis over general convergence, consistent with earlier findings that aspirations for overall cohesion among EU regions remain unfulfilled. This underscores the challenge for policymakers in balancing economic cohesion and sustaining growth drivers within top-performing regions \citep{iammarino2017regional}.

 Additionally, we show that the convergence patterns evolved during the post-crisis recovery and the late 2010s expansion. Despite the continuing dominance of large agglomerations, there is a shift from a large continuous area of top-performing regions in the urbanized part of western Europe (the "Blue Banana") to a central European Manufacturing Core encompassing regions of both old and new EU member states and centered around southern Germany. We also found the worst-performing regions concentrated on both the eastern and southern peripheries.

The paper is structured as follows. Section 2 provides an overview of the methods used for convergence analysis. Section 3 briefly describes the \citeauthor{phillips2007transition}'s test. It also discusses the test results and the composition of the convergence clubs. Section 4 analyzes the factors determining the membership of convergence clubs, and finally, Section 5 provides conclusions and policy implications.  


\section{Literature Review}\label{Methodology}
\subsection{The Convergence Hypothesis and its Empirical Measurement}

\citet{eckey2007convergence} provide a comprehensive survey of methods for convergence analysis and their empirical results. Probably the most common form of the convergence test is the $\beta$-convergence, derived from the transitional income dynamics equation \eqref{eq3}. In its basic version, the $\beta$-convergence has the following form:

\begin{equation}
\label{eq1}\log y_{it} = a_{0} + (1 - a_{1})\log y_{it-1} + u_{it}
\end{equation}

Here, we assume \(0 < a_{1} < 1\). In this setting, \(a_{1} > 0 \) implies convergence, as the growth rate \(\log y_{it} - \log y_{it-1}\) is inversely related to income in period $t$. The error term $u_{it}$ in the equation captures all sorts of temporary changes in the parameters of the production function.

The regression above is often augmented by control variables, creating the conditional convergence model \citep{sala1996regional}. It is worth noting that the $\beta$-convergence is designed to test the sign of the convergence parameter $\beta$ in Equation \ref{eq3} while we assume the homogeneity of the parameter across time and cross sections. This is a key distinction 
between the classical  $\beta$-convergence and the log $t$ test of \citet{phillips2007transition} used in this work, which relaxes this assumption.

While the $\beta$-convergence remains widely used, various other methods have emerged to identify convergence at both state and regional levels. The following part provides a brief description of some of these methods and their results.
  
\citet{eckey2007convergence} conclude in their meta-analysis that empirical convergence literature generally finds a significant, albeit rather small, convergence rate among European regions from the 1980s to the early 2000s. However, this finding is not universal. Some papers report a high rate of convergence, while others find no convergence at all. Such disparities result from methodological differences and variations in the number of regions included \citep{eckey2007convergence}. Additionally, it is suggested that the convergence pattern among European regions was not stable in the post-war period. Some studies indicate a diminishing convergence rate among EU regions in the second half of the twentieth century, while others propose a U-shaped pattern with a trough around the early 1980s, followed by acceleration \citep{basile2001regional, geppert2008regional}. \citet{geppert2008regional} also conclude that convergence is primarily occurring at the national level, while regionally, there is an evident strengthening of metropolitan areas. For example, \citet*{sme2012regional} identify, using the $\beta$-convergence, only a weak tendency for regional convergence in the CEE countries. Notably, in the case of certain countries, such as Poland, absolute divergence persists even when excluding the capital region.
 
\subsection{Club Convergence} 
 Another concept of convergence that has gained prominence in recent years is club convergence. \citet{iammarino2017regional} notes a reversal in inequality trend after the 1990s, with medium-sized manufacturing cities stagnating while large metropolitan areas become more dynamic. She also identifies four convergence clubs in the EU based on their level of GDP per capita. A standard model for testing this type of convergence is the so-called LISA (Local Indicators of Spatial Association). It involves identifying clusters of regions with a variable measuring spatial association consistently above average for an extended period. The statistic defined by \citet{getis1992analysis} is then used as this variable most often. One of the distinguishing features of the LISA methodology is its focus on finding clusters of neighboring regions, often resulting in large clusters along the south-north axis (as found in \citet{baumont2003spatial}) or between new and old member states \citep{eckey2007convergence}. The evidence for regional clustering, however, goes beyond methods like LISA. \citet{cartone2021does} utilize quantile regression within the traditional beta-convergence framework, confirming convergence among European regions but with a faster rate at lower quantiles, supporting the evidence of clustering. These results are supported by \citet{panzera2022impact}, who find that for less developed regions, the convergence process occurs at a faster rate than for more developed ones while investigating the impact of inequality on economic growth.

 
Other approaches used for convergence clubs detection are kernel density estimation, Markov chains \citep{eckey2007convergence}, Bayesian \citep{fischer2015bayesian} and clustering methods \citep{maasoumi2008economic}. Among the methods used for convergence analysis is also the log $t$ test of \citet{phillips2007transition} used in this work. This framework can distinguish between absolute convergence and transitory divergence and also detect convergence in situations where traditional convergence tests fail due to its treatment of convergence as an asymptotic property \citep{bartkowska2012regional, borsi2015evolution}.

Moreover, as suggested by the results of the cited papers below, the log $t$ test is capable of producing convergence clubs, members of which are not necessarily geographic neighbors.This is in contrast with the LISA analysis described previously, which almost necessarily leads to large continuous blocks of regions.

The test proposed by \citet{phillips2007transition} has been relatively widely used at the national level, as shown by \citet{borsi2015evolution, fritsche2011analysing, monfort2013real,apergis2010old} for Europe and by \citet{rodriguez2014there} for Latin America. The test is also used at the regional level where both \citet{bartkowska2012regional} and \citet{pinho2010regional} work with Western European regions. In the first case between the years 1990 and 2005, and in the second case between 1980 and 2007, both using NUTS 2 units. \citet{ghosh2013regional} employs the same method for states in India. \citet{bartkowska2012regional} found, using the log $t$ test, five separate clubs in Western Europe. Analyzing the spatial distribution of these clubs, they observed an agglomeration effect among Western European regions, where regions with capital cities tended to belong to higher convergence clubs than their neighboring regions. Furthermore, there was also a tendency for regions within one country and regions belonging to the same club to cluster together \citep{bartkowska2012regional}. The newest contributions for Western Europe include \citet{von2017regional}, finding four convergence Clubs in the 1980–2011 regional dataset and \citet{cutrini2019economic} finding five clubs among the EU regions, working with data ending in 2015. 

Other empirical works confirm an inclination of large urban areas and, above all, capital cities to grow faster than other regions. This results in increasing regional income inequality, a phenomenon particularly pronounced among the Central and Eastern European countries (\citet{cuaresma2014determinants}; \citet{sme2012regional}; \citet{szendi2013convergence}; \citet{chapman2012income}; \citet{monastiriotis2011regional}. Specifically, \citet{sme2012regional} conclude, using the LISA analysis, that in the CEE countries, regions with large metropolitan areas and some nearly stagnating agricultural areas form different convergence clubs. Nevertheless, there is also relatively high income mobility found in the case of the remaining regions. Interestingly, the authors find that slow-growing regions tend to be located at the eastern border of the EU or in geographically disadvantaged locations. \citet{monastiriotis2011regional}, among others, mentions a strong tendency for income inequality to grow in the CEEC, visible already shortly after the fall of the Iron Curtain. According to \citeauthor{monastiriotis2011regional}, this increase is significantly higher than in the case of the old EU members. Persistent regional disparities in CEE countries, even in composite measures of well-being such as the Human Development Index, are also confirmed by \citet{benedek2015paths}. They found no signs of $\sigma$-convergence between 1995 and 2000.  

In further research, \citet{ehrlich2020place} provide an extensive discussion of EU cohesion policy in light of metro areas' performance, concluding that these place-based policies have a positive impact, especially for areas with higher education levels. \citet{rodriguez2020institutional} focus on investigating the impact of government quality on the economic performance of European regions in a dynamic panel regression framework between 2009 and 2013. They find that government quality is a key determinant of economic growth in the southern periphery and less important in the lagging regions of Eastern Europe. These findings help us to justify the choice of the explanatory variable in the regression below.

Our review suggests that most of the convergence literature works with samples ending in the 2000s and 2010s. As documented above, the convergence pattern changes over time. Therefore, there is room to reassess convergence and its development in light of key economic events. This work will pay attention primarily to the aftermath of the Great Recession and the development preceding the COVID crisis.

\section{Analysis of Convergence}
\subsection{Methodology}

This work employs the log $t$ test by \citet{phillips2007transition} to identify the convergence pattern within the EU and assess the convergence in log GDP per capita across its regions. \citet{phillips2007transition} show that the assumption of a homogenous technological process leads to inconsistency of classical $\beta$-convergence due to omitted variable bias and endogeneity. Consequently, they proposed an alternative testing procedure capable of capturing multiple types of convergence observed in reality, known as the log $t$ test. This convergence test is a variation on the neoclassical growth model that allows for transitional cross-sectional divergence because the model parameters are allowed to vary across cross-sections. Crucially, the model tests the overall relative convergence among the cross-sections, with the alternative allowing both overall divergence and club convergence within regional subgroups. 

\citeauthor{phillips2009economic}'s framework also enables tracing the trajectory of each cross-section $i$ relative to the club's average over time, called by \citet{phillips2009economic} the relative transition path (denoted as $h_{it}$ ). The transition path reflects any divergence of the individual unit $i$ from the common trend, allowing monitoring of the relative economic path of each unit, including transitional or permanent divergence.

To identify the convergence clubs, we used the clustering mechanism proposed by \citet{phillips2009economic} and \citet{bartkowska2012regional}. This procedure allows testing for convergence within subgroups formed through a multistep process. Importantly, it utilizes the log $t$ test's statistic that is being compared against a selected criterion level to determine if certain regions form a convergence club. Details of the test and the clustering procedure are available in the Appendix.

As frequently noted (\citet{dall2008regional}, \citet{magrini2004regional}, \citet{anselin1991properties}, \citet{anselin2001spatial}), spatial autocorrelation poses a common problem in the analysis of regional units, leading to biased $t$-tests and measures of fit. We confirm the existence of autocorrelation among EU regions in log GDP per capita in our data. This could impact the results of \citeauthor{phillips2007transition}'s log $t$ test, as the key parameter $b$ is based on an OLS regression of this variable. Therefore, we use a filtering approach by \cite{getis2002comparative} based on the number of geographic connections of each region within a certain distance to remove the spatial dependence from an autocorrelated variable and produce a new, spatially independent variable. We apply filtering by \cite{getis2002comparative} to log GDP per capita and use the resulting variable in the log $t$ test below.

\subsection{Results: Convergence versus Convergence Clubs}
In this study, we analyzed 275 European NUTS 2 regions in the new and old member states of the EU. The log $t$ test and the associated clustering procedure were applied to the log GDP per capita (in PPS) between the years 2003 and 2019. To assess changes in the convergence patterns, we also estimate the convergence clubs in the sample until 2015. The primary data source was
 the "Regional statistics by NUTS classification" database on the Eurostat website.

Applying the log $t$ test (Equation \ref{eq10} in the Appendix), the convergence among all regions of the EU is rejected with the log $t$ test statistic ($t_b$) equal to -20.110. Similar results are also obtained for the sample ending in 2015, with the test statistics equal to -20.576. Therefore, overall convergence in the EU is not observed in the measured periods.

Subsequently, we tested for the club convergence. Following the clustering procedure proposed by \citeauthor{phillips2007transition}, we initially identified seven convergence clubs and one diverging region, Inner London – West. As suggested by \citet{bartkowska2012regional}, we then attempted to merge adjoining groups and tested whether they converge. This process resulted in five convergence clubs with the third and fourth clubs merging into a new club. The same procedure was applied to the 2015 sample, also leading to the same number of merged clubs and two diverging regions, Luxembourg and Inner London – West. Table \ref{Table_clubs1} shows the results of the procedure after merging for the year 2019, as well as the results of the \citeauthor{phillips2007transition} procedure until 2015.\footnote{Tables \ref{Table_clubs2} and \ref{Table_clubs2_2015} in the Appendix show the clubs before merging for the 2019 and 2015 sample, respectively.}  We can observe, above all, that the higher clubs have more members than in 2019; top economic performance in Europe seems to be becoming more exclusive. This shift is accompanied by a significant increase in the number of regions in the poorest Club 5. On the other hand, there is also an expansion of the intermediate Club 3.

\begin{table}[!htbp] \centering 
 \caption{\textbf{Convergence club classification after merging}} 
  \label{Table_clubs1}
\begin{center}
 \scalebox{0.85}{
\begin{tabular}{@{\extracolsep{5pt}}lclllll} 
\\[-1.8ex]\hline 
\hline \\[-1.8ex] 
Club & \multicolumn{1}{c}{N} & \multicolumn{1}{c}{log($t$)} & \multicolumn{1}{c}{$t$-stat.} & \multicolumn{1}{c}{Y}
& \multicolumn{1}{c}{$\Delta$Y} & \multicolumn{1}{c}{$\sigma$Y} \\ 
\hline \\[-1.8ex]
\multicolumn{7}{c}{2019 Convergence Clubs}\\
Club 1 & 10 &  0.019  & 0.113 &  11.000 & 0.584 & 0.156  \\ 
Club 2 & 24 & 0.108 & 0.710 & 10.700 & 0.461 & 0.133   \\ 
Club 3 & 111 & -0.123   & -1.400 & 10.400 &  0.418 & 0.181 \\ 
Club 4 & 52 & 0.163  & 1.530 & 10.100 & 0.407 & 0.148 \\ 
Club 5 & 79 & -0.096 & -0.968 & 9.880 & 0.340 & 0.229 \\
\hline
\multicolumn{7}{c}{2015 Convergence Clubs}\\
Club 1 & 17 &  0.266  & 2.693 & 10.900 & 0.444  & 0.190 \\ 
Club 2 & 36 & -0.015 & -0.088 & 10.600 & 0.350 & 0.169 \\ 
Club 3 & 93 &  -0.111  & -1.032 & 10.200 & 0.307 & 0.224\\ 
Club 4 & 98 & -0.063  & -0.593 & 10.000 & 0.289 & 0.197\\ 
Club 5 & 29 & -0.099 & -0.638 & 9.770 & 0.236 & 0.263\\ 
\hline
\end{tabular}
}
\caption*{\scriptsize Note: the table contains from left to right: the number of regions included in each Club, coefficient $b$ of the convergence test (eq. \ref{eq10}) and its $t$-statistics, and statistics relating to the log GDP per capita - its value in  2019 or 2015, change from 2008 and standard deviation in 2015/19.}
\end{center}
\end{table}


The value of coefficient $b$ from equation \ref{eq10} plays a crucial role in the log $t$ test. It is interpreted as the speed of convergence and also shows the sign and magnitude of the $t$-statistic. By analyzing the value of $b$ for both versions of the test for 2019 (five and seven clubs), we found that the convergence clubs are rather weak. We can see mostly negative, although insignificant, values of the $b$ coefficients suggesting relative convergence at a very slow rate, as the estimate of the parameter $\alpha$ is not significantly different from 0 \citep{phillips2007transition}. The clubs before merging show a higher value of $t$-statistics with all $b$ coefficients positive. However, the quantitative conclusions are the same for both versions.

We also plotted the convergence behavior using the relative transition coefficient $h_{it}$ from Equation \ref{eq8}, which traces the individual transition path of each cross-sectional unit with respect to a group average for the graphical investigation of relative convergence over time. Figures \ref{paths1} and \ref{paths5} show the relative transition paths for the first and the last convergence clubs in 2019.\footnote{Analogical Figures for the rest of the Clubs can be found in Figures \ref{paths2} - \ref{paths4} in the Appendix.} These pictures confirm that the convergence paths can be various, despite the ultimate convergence, with some regions rising from very low relative levels, others descending from relatively higher levels of GDP per capita to the group average. We can also note alternating states of convergence and divergence for some regions. Figure \ref{paths_overall} then documents the overall non-convergence in the EU as a whole using average transition paths for all five clubs.

\begin{figure}[!htbp]%
    \centering
\begin{subfigure}{0.45\textwidth}
    \centering
    \includegraphics[width=0.95\linewidth]{RTC_mer_club1.png}
    \caption{First Club}
    \label{paths1}   
\end{subfigure}\hfill
\begin{subfigure}{0.45\textwidth}
    \centering
    \includegraphics[width=0.95\linewidth]{RTC_mer_club5.png}
    \caption{Fifth Club}
    \label{paths5}
\end{subfigure}
\caption{\textbf{Transition paths for members of the First and Last Convergence Clubs (2019)}}
\label{convergence_paths}

\end{figure}


Figure \ref{clubs_graphic} illustrates the geographical distribution of the convergence clubs in Europe.Focusing first on results up to 2019 (Figure \ref{clubs_graphic}\subref{clubs_graphic_2019}), we observe a tendency for capital regions to be associated with the highest clubs. This trend extends to the eastern periphery, with the notable exception of the south, where major cities belong to Club 3.\footnote{Complete list of the convergence clubs is available in the Appendix.} The robust economic performance of capital regions is further underscored by their often immediate proximity to areas belonging to considerably lower clubs. The Paris region serves as a prominent example of this pattern. Importantly, this observed trend holds true for both old and new member states Additionally, there is a clear spatial clustering tendency for regions within the same club, evident in the continuous bright-colored areas versus darker blue areas in Figure \ref{clubs_graphic}..\footnote{Note that there is also an outlying region (Inner London - West) included, forming the "Club 0".} These clusters do not adhere to national borders but rather exhibit an interstate nature.For instance, there is a cluster of Club 3 regions spanning from the Czech Republic to southeastern France, as well as a group of low-performing areas along the eastern border.

The evident prominence of capital cities in our findings aligns with earlier observations on convergence clubs by \citet{sme2012regional} and \citet{bartkowska2012regional}. Notably, our results diverge from supporting a country-centric effect, where regions from the same country tend to belong to the same club, as identified by \citet{bartkowska2012regional} for old EU members in 2005. Instead, we consistently observe a diverse distribution of convergence clubs within individual countries, challenging the notion of a uniform country effect. This phenomenon is notable in Eastern European countries like Poland and Romania. Strikingly, Western European countries, including France and the UK, also exhibit this trend, dispelling the assumption that this clustering effect is exclusive to specific regions or country groupings.

\begin{figure}[!htbp]%
\centering
\begin{subfigure}[c]{0.77\linewidth}
  {
  \includegraphics[width=\linewidth]{convergence_clubs2019.png}
    \caption{2019}
    \label{clubs_graphic_2019}
  }
\end{subfigure}
\qquad
\begin{subfigure}[c]{0.77\linewidth}
  \includegraphics[width=\linewidth]{map_4062.jpeg}
  \caption{2015}
  \label{clubs_graphic_2015}
\end{subfigure}
  \caption{\textbf{Comparison of convergence clubs and diverging regions in 2015 and 2019}}
\label{clubs_graphic}
\end{figure} 

When comparing the outcomes for 2015 and 2019, a notable shift is evident with an increased concentration of high-convergence clubs around southern Germany. Conversely, there is a noticeable decline in relative economic dynamism in Northern Italy and, to some extent, in the Benelux region. This trend aligns with the "Manufacturing core" hypothesis \citep{cutrini2019economic, stollinger2016structural}, emphasizing the significance of manufacturing-oriented areas in Central Europe. On the other hand, the so-called "Blue banana," a large urbanization corridor in Western Europe \citep{hospers2002beyond}, visible until 2015, disappears as the timespan extends to 2019. The top club membership can thus be characterized by southern Germany and capitals/major urban areas.

It is noteworthy that the distinction in club membership between older and newer member states appears rather flexible, as substantial portions of rural France and Poland share membership in the same convergence club. Moreover, the top convergence club is becoming more exclusive, with the number of regions decreasing from 15 to 10,  indicating uneven economic development in Europe prior to the COVID-19 pandemic.

Simultaneously, numerous capitals from Central and Eastern European (CEE) countries occupy the top two clubs, challenging the notion of a clear-cut new-old division proposed in prior research \citep{eckey2007convergence}

Conversely, there is a discernible concentration of regions in Clubs 4 and 5 along the eastern border of the EU. In contrast to the 2015 results, where the highest concentration of the lowest clubs was in the southern part of the EU (Figure \ref{clubs_graphic}\subref{clubs_graphic_2015}), their relative performance seems to be deteriorating. Consequently, we cannot affirm any east-west (old-new) general convergence and instead incline toward a broadly defined center-periphery division. This division is marked by financial and administrative urban centers on one hand and the EU's manufacturing core, concentrated around Southern Germany, on the other.

\section{Determinants of Club Membership}
\label{Determinants of club membership}

After identifying the convergence clubs, we analyzed the factors that influence the economic performance of European regions using the ordered logistic regression framework.

\subsection{Explanatory Variables}


\begin{table}[!htbp] \centering 
\resizebox{0.8\textwidth}{!}{\begin{minipage}{\textwidth}
  \caption{\textbf{Explanatory variables - description.}} 
  \label{var_legend}
  \renewcommand{\arraystretch}{1.5}
\begin{tabular}{@{\extracolsep{5pt}} p{4cm}p{9.5cm}} 
\\[-1.8ex]\hline 
\hline \\[-1.8ex] 
  Regression variable  & Description \\ 
\hline \\[-1.8ex] 
Initial GDP level &  Level of GDP per capita (PPS) in 2008 (log) \\
Investment & Gross fixed capital formation in millions of Euros, across all NACE activities (log) \\
Tertiary edu. share & Average percentage of population with tertiary education. \\ 
Share of scientists  & Share of scientists in active population  \\ 
Spec. agriculture & Share of population from 15 to 64 years employed in agriculture (NACE Rev.2 category A) in all employed (in 2008).  \\ 
Spec. Industry & Share of population from 15 to 64 years employed in Industry (NACE Rev.2 categories B-E) in all employed (in 2008)  \\
Spec. Financial Services & Share of population from 15 to 64 years employed in Financial and insurance activities (NACE Rev.2 category K) in all employed (in 2008)\\
Spec. Trade, transport, accommodation \& food services & Share of population from 15 to 64 years employed in NACE Rev.2 categories G-I (2008). \\
Spec. Other Services & Share of population from 15 to 64 years employed in Arts, entertainment and recreation (NACE Rev.2 categories R-U, 2008). \\
Spec. Business Services & Specialization in Business Services (NACE Rev.2 subcategories J58, J62, J63, M69, M70, M71, M73, N78) in 2008.  \\ 
Metropolitan area & Dummy indicating presence of a large urban area.  \\
Capital & Dummy indicating a capital city.  \\
Patent activity & Average number of patents per million inhabitants \\
Regional Innovation Index & European and Regional Innovation Scoreboards an extension of the European innovation scoreboard (EIS) published by the European Commission between years 2014 and 2019.
\\
Governance Quality & Measured by European Quality of Government Index (EQI) in 2010  \citep{charron2014regional}.\\
Highest neighbor Income & Log GDP per capita of the richest neighboring region.  \\
Border Region &  Dummy indicating region lying on EU's internal or external border.\\
\hline 
\hline \\[-1.8ex]
\textit{Note:} & Unless stated otherwise, the variables are used as initial conditions in 2008.   \\
\end{tabular}
\end{minipage}}
\end{table} 

We employed various explanatory variables in our analysis, utilizing Eurostat data ("Regional statistics by NUTS classification"). Given our focus on post-crisis development and a specific interest in investigating the role of sectoral composition, we employed data from 2008 for the logistic regression. This choice deviates from the 2003 - 2019 period used in the convergence analysis due to the limited availability of older data for new member states and changes in the Eurostat classification of economic activity affecting sectoral variables. It also allows us to concentrate on the EU's development between two distinct crises—the Great Recession of 2008 and 2009 and the COVID crisis. The selection of these variables aligns with New Economic Geography (NEG) and New Growth Theory (NGT), as well as previous research papers such as \citet{mora2008factors} or \citet{bartkowska2012regional}. Table \ref{var_legend} presents an overview of the variables with their names as they appear in the regression results below.


Consistent with the literature, we incorporated variables measuring the level of initial economic specialization in most NACE Rev.2 classification sectors.\footnote{Certain sectors, such as Real Estate activities, were omitted due to a large number of missing observations.} Specialization is defined as the share of the population employed in a given sector over overall employment in the year 2008. We considered six sectors spanning from Industry to Financial Services, as defined by the NACE classification of economic activity, and Business Services. Business services, a dynamically developing subsector, have been identified in previous research as potentially highly innovative and productivity-promoting \citep{corrocher2014kibs}. Specialization in Business Services is defined as the share of workers employed in Business services in the total employed population, following \citet{guastella2015knowledge}.This study adopts the definition of business services outlined in Regulation of the European Parliament No. 298/2008. According to this regulation, business services comprise five divisions and three groups in the NACE Rev. 2 classification, specifically divisions 62, 69, 71, 73, and 78, and groups 58.2, 63.1, and 70.2. However, the Eurostat database does not provide data at this detailed level for business services. Consequently, we use whole divisions (58, 63, and 70) as reasonable approximations for the respective groups.

Additionally, we included the log of initial (as of 2008) GDP per capita and investment as measures of initial economic conditions, in line with neoclassical growth theory \cite{iammarino2017regional}.

In addition to the variable measuring sectoral specialization mentioned earlier, we incorporated variables assessing human capital developments in the regions. We also experimented with several variables measuring regional innovation levels, including the percentage of the population with tertiary education, the share of scientists in the active population, and the average number of patent applications to the European Patent Organization per million inhabitants.

Furthermore, we included the Regional Innovation Index issued by the European Commission and the European Quality of Government Index (EQI), assuming that better-governed regions are more attractive for both business and workers, and thus, are likely to perform better in all aspects of economic life.

Inspired by the emphasis on agglomeration economies in New Economic Geography as key drivers of growth and territorial divergence \citep{iammarino2017regional}, we included dummies measuring the level of regional urbanization, indicators for capital cities, and a dummy variable for border regions.
 %As an alternative, we utilize Eustostat's distinction between urban, rural and intermediate areas.
Lastly, we considered the income level of the richest neighbor of a given region to assess the impact of a region's surroundings.

The next section shows the impact of the explanatory variables discussed above on the probability of membership in a given convergence club, tested through logistic regression. pecifically, we were interested in the impact of the initial conditions on future economic performance. Therefore, the 2008 values of the listed variables were used.\footnote{Models presented below use Factorial Analysis for Mixed Data (FAMD) as an imputation method to deal with the missing observations.}

\subsection{Regression Results}

The outcomes of the ordered logistic regression are displayed in the tables below. We focus on marginal effects that are calculated at the means of the independent variables and show the impact of a unit change in an explanatory variable on the probability that a certain region will enter a particular club \citep{carrolloglmx}.\footnote{Regression coefficients are reported in the Appendix.}


\begin{table}[!htbp] \centering 
\resizebox{0.8\textwidth}{!}{\begin{minipage}{\textwidth}
  \caption{\textbf{Marginal effects of Logistic regression with Sectoral and Human capital variables (2008 levels). Logistic regression with merged Clubs, FAMD imputation and robust standard errors. Nine regions with no neighboring regions removed.}}
  \label{famd_filtered_marginal_spec1}
\begin{tabular}{@{\extracolsep{5pt}} lccccc} 
\\[-1.8ex]\hline 
\hline \\[-1.8ex] 
 & Club 1 & Club 2 & Club 3 & Club 4 & Club 5 \\ 
\hline \\[-1.8ex] 
Initial GDP level & $0.011$$^{*}$ & $0.108$$^{***}$ & $1.712$$^{***}$ & $ $-$0.980$$^{***}$ & $ $-$0.851$$^{***}$ \\ 
& (0.006) &  (0.032) & (0.253) & (0.183) & (0.145)\\
&\\
Investment  & $0.000$ & $0.000$ & $ $-$0.001$ & $0.000$ & $0.000$ \\
& (0.000) &  (0.003) & (0.052) & (0.030) & (0.026)\\
&\\
Spec. Agriculture & $0.021$ & $0.202$$^{**}$ & $3.211$$^{***}$ & $ $-$1.838$$^{***}$ & $ $-$1.596$$^{***}$ \\ 
& (0.013) &  (0.090) & (1.061) & (0.681) & (0.508)\\
&\\
Spec. Finances & $0.053$ & $0.519$$^{**}$ & $8.229$$^{**}$ & $ $-$4.712$$^{***}$ & $ $-$4.090$$^{***}$ \\
& (0.035) &  (0.241) & (2.746) & (1.636) & (1.462)\\
&\\
Spec. Industry & $0.023$$^{*}$ & $0.230$$^{***}$ & $3.648$$^{***}$ & $ $-$2.089$$^{***}$ & $ $-$1.813$$^{***}$ \\
& (0.014) &  (0.078) & (0.637) & (0.454) & (0.341)\\
&\\
Spec. Trade, Accommodation & $0.005$ & $0.053$ & $0.836$ & $ $-$0.479$ & $ $-$0.416$ \\ 
& (0.008) &  (0.080) & (1.193) & (0.692) & (0.590)\\
&\\
Spec. Other Services & $ $-$0.025$ & $ $-$0.248$ & $ $-$3.939$ & $2.255$ & $1.958$ \\ 
& (0.027) &  (0.216) & (3.119) & (1.804) & (1.572)\\
&\\
Business Services & $0.010$ & $0.095$$^{*}$  & $1.509$$^{*}$  & $ $-$0.864$$^{*}$  & $ $-$0.750$$^{*}$  \\
& (0.006) &  (0.057) & (0.810) & (0.479) & (0.401)\\
&\\
Share of Scientists & $0.225$$^{*}$  & $2.203$$^{***}$  & $34.931$$^{***}$  & $ $-$19.999$$^{***}$  & $ $-$17.359$$^{***}$  \\ 
& (0.132) &  (0.841) & (8.495) & (5.594) & (4.306)\\
&\\
Tertiary Edu. Share & $ $-$0.005$ & $ $-$0.051$$^{*}$  & $ $-$0.801$$^{**}$  & $0.458$$^{**}$  & $0.398$$^{**}$  \\
& (0.004) &  (0.026) & (0.390) & (0.228) & (0.196)\\
\hline \hline \\[-1.8ex]
McFadden's $R^{2}$: 0.366 \\
Observations: 267\\
No. of Parameters: 10\\
\hline
\end{tabular}
\begin{tablenotes}
\small 
\item Note: $^{*}$p$<$0.1; $^{**}$p$<$0.05; $^{***}$p$<$0.01
\item Club 1 represents the regions with the highest income, Club 5 with the lowest.
\item As in the remaining tables, the standard errors above are heteroskedasticity-consistent.
\end{tablenotes}
\end{minipage}}
\end{table} 

We consider three alternative model specifications. Initially, we examined only sectoral variables along with those related to human capital, as presented in Table \ref{famd_filtered_marginal_spec1}. In the second specification, outlined in Table \ref{famd_filtered_marginal_spec2}, we investigate geographic factors associated with economic performance, such as dummies for capital cities and the influence of neighboring regions.. The results in Table \ref{famd_selected_marginal_neigh_robust_filtered} then bring all the variables together. 

Across these specifications, our findings suggest that the magnitude of the marginal effects peaks around Club 3 and weakens for the very top Club. There is a minimal impact on the top Club membership for any of our variables, both in terms of significance and magnitude. 
Analyzing the influence of individual variables, it is evident that the initial income per capita significantly affects club membership. A higher initial GDP per capita in 2008 increases the probability of being in the highest convergence clubs in 2019, suggesting a certain level of inertia in economic development.

 
Table \ref{famd_filtered_marginal_spec1} documents the effect of structural specialization variables,  as suggested by \citet{cutrini2019economic}.The impact of the industrial sector\footnote{Primarily composed of Manufacturing but also encompassing NACE Sections B, E, F (Mining, Electricity, gas, steam and air conditioning supply, and Water supply) besides Manufacturing (NACE Section C) itself, which plays a dominant role in the category.} and Finance and Insurance Services is particularly pronounced.  The two variables turn out to be the most important predictors of Club 1 membership in Table \ref{famd_filtered_marginal_spec1}. This observation corresponds to two noticeable trends in contemporary Europe - the first being the significance and emergence of the 'Manufacturing Core' in its central part, and the second highlighting the pivotal role of urban financial centers 

The relevance of EU's manufacturing confirms the analysis in the preceding section, showing that manufacturing was a key driver of growth in the region pre-COVID. However, the sector's pivotal role raises a significant question about the future. The 2022 energy crisis caused a substantial upheaval in the EU industry, posing a threat of extensive relocation. Yet, the long-term macroeconomic impact of the crisis remains uncertain. Preliminary assessments of the crisis suggest that its repercussions may be confined to the relatively narrow, energy-intensive sector \citep{sgaravatti2023adjusting}. Consequently, policymakers' choices remain open and fall beyond the purview of this text.\footnote{Similar concerns for the EU industry may arise due to geopolitical competition, as evidenced by industrial policies like the Inflation Reduction Act (IRA), which could potentially lead to substantial industry reallocation from the EU.}

Examining the role of financial services, we can see that their concentration overlaps with the incidence of capital cities, raising questions about their significance and favoring the argument for agglomeration effects, as suggested by New Economic Geography.
We also observe a negative impact of the "Other Services" category, including entertainment and recreation, on the probability of  higher club membership. This may be attributed to the strong tourism dependence of the southern EU periphery. However, the marginal effects are never significant for this variable. Conversely, specialization in business services has a positive impact on the probability of entering the highest clubs, but the variable is again only marginally significant.


\begin{table}[!htbp] \centering 
\resizebox{0.8\textwidth}{!}{\begin{minipage}{\textwidth}
  \caption{\textbf{Marginal effects of Logistic regression with Geographic determinants (2008 levels). Logistic regression with merged Clubs, FAMD imputation and robust standard errors. Nine regions with no neighboring regions removed.}} 
  \label{famd_filtered_marginal_spec2} 
\begin{tabular}{@{\extracolsep{5pt}} lccccc} 
\\[-1.8ex]\hline 
\hline \\[-1.8ex] 
 & Club 1 & Club 2 & Club 3 & Club 4 & Club 5 \\ 
\hline \\[-1.8ex] 
Initial GDP level & $0.023$$^{**}$ & $0.173$$^{***}$ & $1.684$$^{***}$ & $ $-$0.870$$^{***}$ & $ $-$1.010$$^{***}$ \\ 
& (0.010) &  (0.045) & (0.260) & (0.173) & (0.166)\\
&\\
Investment & $ $-$0.001$ & $ $-$0.004$ & $ $-$0.040$ & $0.020$ & $0.024$ \\
& (0.001) &  (0.005) & (0.046) & (0.024) & (0.027)\\
&\\
Capital & $0.004$$^{*}$ & $0.031$$^{**}$ & $0.186$$^{***}$ & $ $-$0.119$$^{***}$ & $ $-$0.103$$^{***}$ \\
& (0.002) &  (0.013) & (0.054) & (0.040) & (0.029)\\
&\\
Metro area & $0.003$$^{*}$ & $0.019$$^{**}$ & $0.240$$^{***}$ & $ $-$0.089$$^{***}$ & $ $-$0.173$$^{***}$ \\
& (0.001) &  (0.008) & (0.076) & (0.027) & (0.064)\\
&\\
Governance Quality & $ $-$0.001$ & $ $-$0.006$ & $ $-$0.055$ & $0.028$ & $0.033$ \\ 
& (0.001) &  (0.005) & (0.039) & (0.021) & (0.023)\\
&\\
Highest neighbor Income & $ $-$0.004$$^{*}$ & $ $-$0.028$$^{**}$ & $ $-$0.273$$^{***}$ & $0.141$$^{***}$ & $0.164$$^{***}$ \\ 
& (0.002) &  (0.011) & (0.081) & (0.047) & (0.050)\\
&\\
Border Region & $0.004$$^{*}$ & $0.029$$^{***}$ & $0.269$$^{***}$ & $ $-$0.129$$^{***}$ & $ $-$0.173$$^{***}$ \\ 
& (0.002) &  (0.009) & (0.062) & (0.034) & (0.041)\\
\hline \hline \\[-1.8ex]
McFadden's $R^{2}$: 0.313 \\
Observations:  267\\
No. of Parameters:  7\\
\hline 
\end{tabular}
\begin{tablenotes}
\small 
\item Note: $^{*}$p$<$0.1; $^{**}$p$<$0.05; $^{***}$p$<$0.01
\end{tablenotes}
\end{minipage}}
\end{table} 

We can observe a strong effect of variables related to research activity in Table \ref{famd_filtered_marginal_spec1}, such as the share of scientists and patent activity.\footnote{Given the high correlation among research activity variables, including the Regional Innovation Index, are highly correlated, Table \ref{famd_filtered_marginal_spec1} shows only the share of the scientists, which proved to be the most impactful variable, other results are available upon request.}The findings suggest that the initial share of scientists significantly influences the probability of entering the First Club. Regarding the share of tertiary-educated individuals in the population, the results show that its initial levels are also significant. However, the marginal effects yield a somewhat counterintuitive result, indicating a positive association with the lower clubs. One possible interpretation of these results could be that increasing the formal level of education in a given region alone is insufficient for economic prosperity. It needs to be accompanied by functional innovation ecosystems capable of providing suitable employment opportunities for the growing number of tertiary education graduates.

In the second specification presented in Table \ref{famd_filtered_marginal_spec2}, we observe a positive impact of metropolitan areas, especially capital cities, which appear to predict high club membership. In fact, the highest club is primarily composed of capital cities, with six of the ten regions in Club 1 containing a capital city. The impact of the metropolitan areas dummy on club membership is very similar to what we observed for capital cities.

Table \ref{famd_filtered_marginal_spec2} further shows that the presence of a high-income neighbor decreases the economic prospects of its neighbors. These results suggest that high-income localities can exert a negative influence on their surroundings, most likely in the form of draining the human capital of less affluent neighbors. Additionally, the border regions dummy is significant, with a positive effect, which is not surprising, as many capital cities and prosperous regions are situated near the internal borders of the EU. While we also tested the impact of the Quality of Governance index, it turned out to be insignificant.


\begin{table}[!htbp] \centering
\resizebox{0.75\textwidth}{!}{\begin{minipage}{\textwidth}
  \caption{\textbf{Marginal effects: Logistic regression with merged Clubs, FAMD imputation and robust standard errors (variables in 2008 levels). Nine regions with no neighboring regions removed.}} 
  \label{famd_selected_marginal_neigh_robust_filtered} 
\begin{tabular}{@{\extracolsep{5pt}} lccccc} 
\\[-1.8ex]\hline 
\hline \\[-1.8ex] 
 & Club 1 & Club 2 & Club 3 & Club 4 & Club 5 \\ 
\hline \\[-1.8ex]
%log\_init\_gdp & $0.005$ & $0.076$ & $2.182$ & $$-$1.453$ & $$-$0.810$ \\
Initial GDP level & $0.005$ & $0.076$$^{***}$ & $2.182$$^{***}$ & $ $-$1.453$$^{***}$ & $ $-$0.810$$^{***}$ \\ 
%init\_gfcf & $0$ & $$-$0.001$ & $$-$0.022$ & $0.014$ & $0.008$ \\ 
& (0.004) &  (0.027) & (0.292) & (0.263) & (0.141)\\
&\\
Investment & $0.000$ & $ $-$0.001$ & $ $-$0.022$  & $0.014$ &  $0.008$ \\
& (0.000) &  (0.002) & (0.057) & (0.038) & (0.021)\\
&\\
\multicolumn{6}{c}{\textit{Sectoral Variables}}\\

%spec\_A & $0.010$ & $0.136$ & $3.921$ & $$-$2.611$ & $$-$1.456$ \\ 
Spec. Agriculture & $0.010$ & $0.136$$^{**}$ & $3.921$$^{***}$ & $ $-$2.611$$^{***}$ & $ $-$1.456$$^{***}$ \\ 
& (0.006) &  (0.058) & (1.110) & (0.806) & (0.439)\\
&\\
% spec\_K & $0.015$ & $0.213$ & $6.157$ & $$-$4.100$ & $$-$2.286$ \\  
Spec. Finances & $0.015$ & $0.213$ & $6.157$$^{**}$ & $ $-$4.100$$^{*}$ & $ $-$2.286$$^{*}$ \\ 
& (0.012) &  (0.132) & (3.105) & (2.079) & (1.230)\\
&\\
%spec\_BE & $0.011$ & $0.156$ & $4.512$ & $$-$3.004$ & $$-$1.675$ \\ 
Spec. Industry & $0.011$ & $0.156$$^{***}$ & $4.512$$^{***}$ & $ $-$3.004$$^{***}$ & $ $-$1.675$$^{***}$ \\ 
& (0.007) &  (0.059) & (0.799) & (0.642) & (0.355)\\
&\\
% spec\_GI & $0.002$ & $0.030$ & $0.865$ & $$-$0.576$ & $$-$0.321$ \\ 
Spec. Trade, Accommodation & $0.002$ & $0.005$ & $0.143$ & $ $-$0.099$ & $ $-$0.049$ \\
& (0.003) &  (0.041) & (1.127) & (0.753) & (0.421)\\
&\\
%spec\_RU & $$-$0.007$ & $$-$0.095$ & $$-$2.740$ & $1.825$ & $1.017$ \\ 
Spec. Other Services & $ $-$0.007$ & $ $-$0.095$ & $ $-$2.740$ & $1.825$ & $1.017$ \\ 
& (0.010) &  (0.117) & (3.083) & (2.066) & (1.156)\\
&\\
%BS & $0.005$ & $0.066$ & $1.913$ & $$-$1.274$ & $$-$0.710$ \\ 
Business Services & $0.005$ & $0.066$$^{*}$ & $1.913$$^{**}$ & $ $-$1.274$$^{**}$ & $ $-$0.710$$^{**}$ \\ 
& (0.003) &  (0.038) & (0.924) & (0.639) & (0.345)\\


\\
\multicolumn{6}{c}{\textit{Human Capital Variables}}\\
%init\_scientists\_share & $0.054$ & $0.745$ & $21.529$ & $$-$14.336$ & $$-$7.992$ \\ 
Share of Scientists & $0.054$ & $0.745$$^{*}$ & $21.529$$^{**}$ & $ $-$14.336$$^{**}$ & $ $-$7.992$$^{**}$ \\
& (0.039) &  (0.404) & (9.691) & (6.622) & (3.751)\\
&\\
%init\_ter\_edu & $$-$0.002$ & $$-$0.021$ & $$-$0.614$ & $0.409$ & $0.228$ \\ 
% init\_eqi & $0$ & $0.002$ & $0.046$ & $$-$0.031$ & $$-$0.017$ \\  
Governance Quality & $0.000$ & $0.002$ & $0.046$ & $ $-$0.031$ & $ $-$0.017$ \\
& (0.000) &  (0.002) & (0.063) & (0.042) & (0.024)\\
&\\
Tertiary Edu. Share  & $ $-$0.002$ & $ $-$0.021$ & $ $-$0.614$ & $0.409$ & $0.228$ \\
& (0.001) &  (0.015) & (0.397) & (0.267) & (0.151)\\
\\
\multicolumn{6}{c}{\textit{Geographic Determinants}}\\
%capital & $0.002$ & $0.027$ & $0.294$ & $$-$0.227$ & $$-$0.096$ \\ 
Capital & $0.002$ & $0.027$$^{**}$ & $0.294$$^{***}$ & $ $-$0.227$$^{***}$ & $ $-$0.096$$^{***}$ \\
& (0.001) &  (0.013) & (0.054) & (0.051) & (0.021)\\
&\\
% metro & $0$ & $0.004$ & $0.149$ & $$-$0.091$ & $$-$0.062$ \\ 
Metro area & $0.000$ & $0.004$ & $0.149$ & $ $-$0.091$ & $ $-$0.062$ \\ 
& (0.000) &  (0.003) & (0.106) & (0.061) & (0.050)\\
&\\
% neigh\_highest\_income & $$-$0.001$ & $$-$0.010$ & $$-$0.287$ & $0.191$ & $0.107$ \\ 
Highest neighbor Income & $ $-$0.001$ & $ $-$0.010$$^{**}$ & $ $-$0.287$$^{**}$ & $0.191$$^{**}$ & $0.107$$^{**}$ \\
& (0.001) &  (0.005) & (0.116) & (0.081) & (0.044)\\
&\\
% border\_region & $0.001$ & $0.009$ & $0.252$ & $$-$0.162$ & $$-$0.100$ \\
Border Region & $0.001$ & $0.009$$^{**}$ & $0.252$$^{***}$ & $ $-$0.162$$^{***}$ & $ $-$0.100$$^{***}$ \\ 
& (0.000) &  (0.004) & (0.074) & (0.051) & (0.031)\\

\hline \hline \\[-1.8ex]
McFadden's $R^{2}$: 0.42  \\
Observations: 267\\
No. of Parameters 17\\
\hline
\end{tabular}
\begin{tablenotes}
\small 
\item Note: $^{*}$p$<$0.1; $^{**}$p$<$0.05; $^{***}$p$<$0.01
\end{tablenotes}
\end{minipage}}
\end{table} 


When using all the explanatory variables together in a single regression, as demonstrated in Table \ref{famd_selected_marginal_neigh_robust_filtered}, any significant effect on the very first club is lost. However, the key conclusion from Tables \ref{famd_filtered_marginal_spec1} and \ref{famd_filtered_marginal_spec2} appears to remain unchanged - a strong positive impact of industry and human capital, represented by the share of scientists variable. Compared to the partial results presented in the tables above, financial services lose their impact on higher club membership when we add control for capital cities. As evident in Table \ref{famd_selected_marginal_neigh_robust_filtered}, the variable loses significance beyond Club 3. Additionally, the marginal effect of the dummy variable representing metropolitan cities becomes insignificant, despite the continuing impact of the capitals. This could suggest an increasing level of centralization of economic activity within national borders. The key factor for membership in the first two clubs thus becomes specialization in Manufacturing, which is positive and significant across the specifications.

Our work can be compared with previous studies, such as \citet{cutrini2019economic}, \citet{von2017regional}, and \citet{bartkowska2012regional}, contributing to the ongoing discussion between initial GDP levels and sectoral specialization as primary determinants of economic development. We confirm the importance of geographic factors, noted in \cite{von2017regional}, and the initial per capita GDP level, as stressed by \cite{bartkowska2012regional}. We also share the emphasis on manufacturing as an important determinant of economic growth with \citet{cutrini2019economic}, who concentrates on the initial specialization levels. However, we also find variables linked to the New Economic Geography, such as agglomeration effects or innovation activity, to play a role.

In summary, the results suggest that regional economic well-being is determined by a combination of initial sectoral specialization and its ability to create functional local innovation environments. Our findings highlight the importance of industrial concentration, which appears to be more significant than the initial level of economic development. Similarly, variables related to the development of the knowledge-based economy lead to higher club membership. A specific category that stands out is the large metropolitan areas, particularly the capitals, with their robust position in finance.

\subsection{Robustness Checks}
In addition to the aforementioned variables, we explore two alternative specifications as a robustness check. The tables above include a variable measuring the income of the richest neighbor, resulting in the exclusion of a small number of regions (9) without neighbors.\footnote{These regions are Cyprus, Madeira, the Balearic Islands, Ceuta and Melilla, Canary Islands, Guadeloupe, Martinique, and Malta.} Table \ref{famd_selected_marginal} in the Appendix presents the alternative version of the regression exercise with all regions but omitting the richest neighbor variable. Our results concerning sectoral specialization remain consistent, with specialization in manufacturing and human capital variables appearing closer to significantly impacting the top club.

As a second robustness check, we include regression using the 2003 data, as shown in Table \ref{famd_selected_marginal_ecosys_robust_filtered}.  We chose 2008 as the starting year of our analysis in the previous section due to the change in Eurostat's classification of economic activity from NACE Rev. 1.1 to NACE Rev. 2. Despite the imperfect correspondence in sectoral categories (e.g., the business services had to be dropped), the results in Table \ref{famd_selected_marginal_ecosys_robust_filtered} confirm the main conclusions reached thus far. The only significant difference seems to be the less significant impact of capital and metro dummies on the top clubs.

\section{Conclusion}

This study investigated regional convergence in the European Union between the Great Recession and the Covid-19 crisis of 2019, employing a convergence test developed by \citet{phillips2009economic}. This test allowed us to examine convergence among EU regions, distinguishing between absolute and club convergence.

The \citeauthor{phillips2009economic}'s test rejected the hypothesis of an overall convergence among the EU regions. Consequently, we explored potential convergence clubs, ultimately identifying five distinct clubs converging at a slow rate. Analyzing the composition of these clubs revealed significant disparities in club membership within many countries, indicating high within-country inequality. The highest club predominantly comprised capital cities from both old and new EU member countries, highlighting the capital vs. periphery contrast. Moreover, a notable concentration of regions belonging to the top two clubs was identified in southern Germany, indicating a shift from the 2015 scenario when top clubs were concentrated in a continuous area from northern Italy to the Benelux. This finding aligns more closely with the "Manufacturing Core" hypothesis, coupled with the sustained dominance of major urban areas. Significant concentrations of regions from the lowest clubs were observed in the southern periphery of the EU and its eastern border, affirming the prominence of the center-periphery division in the EU.

Subsequently, the results of this analysis served as the dependent variable in logistic regression, exploring key determinants of regional economic performance inspired by the New Economic Geography and New Growth Theory.

Interestingly, no substantial difference was found between new and old EU members in terms of their convergence pattern. The regression results highlighted that a combination of financial services, manufacturing, research, and innovation drives upward economic mobility among EU regions. However, none of these variables proved to be highly significant for the highest club, dominated by a group of capital cities and major agglomerations.
 
In summary, our findings provide further confirmation of the center-periphery division in the EU, particularly in its "capitals versus the rest" form. The top convergence clubs became more exclusive between 2015 and 2019, shrinking to an area around Southern Germany, supporting the validity of the Manufacturing Core hypothesis over the "Blue Banana" hypothesis. Despite changes in club composition, we consistently confirmed the importance of regional research as a success factor, along with the significance of sectoral specialization in economic success.

This work has important policy implications, confirming the absence of overall convergence among EU regions despite continuous efforts. The detailed comparison of club convergence patterns between 2015 and 2019 underscores the growing exclusivity of the top two convergence clubs and their significant geographic transformation from the "Blue Banana" to the "Manufacturing Core" within a short four-year period. Moreover, this transformation occurred during a time of relative economic expansion in Europe, emphasizing the unequal distribution of benefits across EU regions. Connected to this issue is the significance of manufacturing for pre-COVID economic growth in Europe, as suggested by our results. However, the sustainability of this manufacturing-based growth is now in question during the energy crisis, challenging German competitiveness and threatening widespread deindustrialization of the "Core." This underscores the need to explore a new growth model for Europe.






%\section{Acknowledgements} 
%This work was supported by the Grant Agency of the Czech Republic, Grant No. GACR no. 20-14990S. I would like to thank my supervisor, Jaromír Baxa, for his many insightful comments on this work.


\newpage
\bibliographystyle{apalike}
\bibliography{References}
\appendix
\section{Appendix}
\numberwithin{equation}{section}
\setcounter{table}{0}
\renewcommand{\thetable}{A\arabic{table}}
\renewcommand{\thefigure}{A\arabic{figure}}
\setcounter{figure}{0}


\subsection{Log $t$ test}

Log $t$ test departs from a variation of the neoclassical growth model where the transitional cross-sectional divergence is possible because the parameters $\beta_{it}$ and $x_{it}$ are allowed to vary across cross-sections:

\begin{equation} \label{eq3}
 y_{it} = y_i^* + a_{i0} + (y_{i0} - y_i^*)e^{-\beta_{it}t} + x_{it}t 
\end{equation}

Variables $y_{i0}$ and $y_{i}^{*}$ in \eqref{eq3} are initial and steady state levels of log per capita income, $x_{it}$ expresses technology accumulation over time and $a_{i0}$ captures initial technology accumulation. $\beta_{it}$ is a transition parameter and, together with the technology accumulation parameter $x_{it}$, it is assumed to be homogeneous between countries in the neoclassical theory.

\citeauthor{phillips2007transition} assume that $\beta_{it}$ is an increasing function of the technological progress parameter $x_{it}$ and therefore can explain divergence and income traps among countries (or regions).

If we assume $x_{it}t$ to contain both idiosyncratic and shared elements across economies and we express equation \eqref{eq3} as

\begin{equation} 
\label{eq4} y_{it} = \left(\frac{ y_i^* + a_{i0} + (y_{i0} - y_i^*)e^{-\beta_{it}t} + x_{it}t}{\mu_t}\right)\mu_t = \delta_{it}\mu_t
\end{equation}

Here, \(\mu_t\) represents a common growth component, incorporating both deterministic and stochastic components, arising from knowledge and technology sharing among countries. \(\mu_t\) also determines the common growth in the steady state. \(\delta_{it}\) then captures how close a particular economy is to steady-state growth represented by \(\mu_t\).

By defining \(a_{it} = y_i^* + a_{i0} + (y_{i0} - y_i^*)e^{-\beta_{it}t}\), we can rewrite the loadings as \(\delta_{it} = \frac{a_{it} + x_{it}t}{\mu_t}\). Furthermore, if we represent the common steady state growth element \(\mu_t\) by a simple linear deterministic trend \(\mu_t = t\), we see that

\begin{equation} 
\label{eq5}\delta_{it} = x_{it} + \frac{a_{it}}{t}
\end{equation}  

Thus, \(\delta_{it} \rightarrow x_i\) as $t$ comes to infinity, assuming that $x_{it}$ converges to $x_{i}$. $\delta_{it}$ therefore plays a key role of a transition parameter and is supposed to have the following structure: \(\delta_{it} = \delta_{i} + \sigma_{it}\xi_{it}\) where \( \sigma_{it} = \frac{\sigma_{i}}{log(t)t^\alpha}\). The parameter $\alpha$ then sets the rate at which \(\delta_{it} \rightarrow \delta_{i}\) with \(t \rightarrow \infty\) and can be interpreted as the speed of convergence. In particular, the convergence of $\delta_{it}$ to $\delta_{i}$ is guaranteed for all \(\alpha \geq 0\). This inequality, therefore, becomes subject of the null hypothesis of the test below together with the condition of shared value of $\delta_{i}$ across cross-sections: 

\begin{equation}
\label{eq6} H_0: \delta_{i} = \delta \quad \& \quad \alpha \geq 0
\end{equation} 

Thus, we test the overall relative convergence among the cross sections, with the alternative allowing both overall divergence and club convergence.
For the testing procedure as well as modelling of the transition parameter $\delta_{it}$ is then used following formula:

\begin{equation}\label{eq8}h_{it} = \frac{y_{it}}{N^{-1}\sum\limits_{i=1}^Ny_{it}} = \frac{\delta_{it}}{N^{-1}\sum\limits_{i=1}^N\delta_{it}}\end{equation}

This formula traces the trajectory of each cross-section $i$ relative to the club's average and is thus called by \citet{phillips2009economic} the relative transition path. It also reflects any divergence of the individual unit $i$ from the common trend $\mu_t$. While individual transition paths $h_{it}$ may be various, including transitional or permanent divergence, the ultimate growth convergence implies \(h_{it} \rightarrow 1\).

In the convergence test itself, the authors focus on cross-sectional convergence of individual $\delta_{it}$. Mainly due to the ease of calculation from the data, the authors concentrate on $h_{t}$, rather than $\delta_{it}$ itself, in the test derivation and they compute mean square cross-sectional "transition differential" of $h_{it}$:
\begin{equation}\label{eq9}H_t = N^{-1}\sum\limits_{i=1}^N(h_{it} - 1)^2 \end{equation}
As the ultimate growth convergence implies \(h_{it} \rightarrow 1\), the value $H_{t}$, which can also be interpreted as a quadratic distance of the club from the common limit, has to converge to zero with time going to infinity. If it remains positive, we conclude that convergence did not happen \citep{phillips2009economic}.

The test itself is motivated by the challenge of distinguishing whether \(H_t\) converges to zero or to a constant. \citet{phillips2007transition} developed a model based on the following OLS regression, along with a testing procedure introduced later. They first demonstrate that under the model specification shown above, the term \(H_{t}\) has the following limiting form: \(H_{t} = \frac{A}{\log (t)^2t^{2\alpha}}\) as \(t \rightarrow \infty\). This leads to the final formulation of the log \(t\) regression:

\begin{equation}\label{eq10}\log\left(\frac{H_1}{H_t}\right)-2\log(\log(t)) = a + b\log(t) + u_t\end{equation}

The test, referred to as the log \(t\) convergence test, is then a one-sided \(t\)-test of convergence against no or partial convergence. The coefficient \(b\) converges in probability to the speed of the convergence parameter \(2\alpha\), and the convergence hypothesis is tested using a one-sided \(t\)-test of the inequality \(\alpha \geq 0\), utilizing the estimated parameter \(b\) with the HAC standard errors.

\subsection{The log $t$ test - formation of convergence clubs }
If the hypothesis of overall convergence is rejected, we explore convergence within subgroups of the analyzed sample. The clustering procedure follows these steps:
\begin{itemize}
    \item Order the individuals by the amount of last period income (or other variable).
    \item  A core group of $k^{*}$ highest individuals is chosen by maximizing the log $t$ test's statistic $t_{k}$ over the various sizes of $k^{*}$:
    \(k^{*} = arg max_{k}\{t_{k}\}\) subject to min \( \{t_{k}\}> -1.65\). If \(t_{k} \leq -1.65 \) for \(k=2\), the highest individual is dropped and this step is repeated starting from the second highest observation. 
    \item One region at a time is added to the core group formed in the previous step, and the log $t$ test is run again. The respective  $t$-statistics is than compared to the criterion level $c^{*}$. In our case, we choose $c^{*}=0$. If the associated $t$-statistic is greater than  $c^{*}$, we add the individual to the club.
    \item We run the log $t$ test for all the remaining observations; if they fulfill \(t_b > -1.65\) we conclude that they form a second convergence club. If not, we repeat all the previous steps with the remaining observations to see whether we can find convergence clubs among these remaining individuals.
\end{itemize}    

 \(c^{*} = 0\), in the second step of the procedure, plays an important role in the final composition of the convergence clubs, with higher values of $c^*$ meaning a lower probability of including a wrong region in the club. \citet{phillips2009economic} note that $c^{*}$ can vary between 0 and -1.65 and $c^{*}=0$ is considered a very conservative choice, which tends to detect a larger number of clubs than it should. On the other hand, $c^{*}=-1.65$ was recommended when we have a relatively large dataset. As our data starts from various reasons only after 2003, we adopt $c^*= 0$ combined with a club merging procedure suggested by \citet{phillips2009economic} and further elaborated by \citet{bartkowska2012regional}. 

This procedure contains a step-by-step merging of several groups together and testing whether the log $t$ test statistics of this merged group is larger than -1.65. If it is larger, we conclude that these two clubs form a convergence club together. Concretely, we start by merging the first and second clubs together and proceed by adding the following clubs until the null hypothesis of the log $t$ test is rejected. We conclude that all clubs that passed the test form a single convergence club. Subsequently, we continue, starting again from the first club for which this merging hypothesis was rejected. We try to merge it with all remaining clubs, using the same procedure. If the null hypothesis is rejected for the first and second clubs, we leave the first club untouched and start by merging the second with the third club and continue in the same fashion as just described.


%\subsection{Club membership (2019)}
%\begin{itemize}
%\item Club 1:\\
%Région de Bruxelles-Capitale / Brussels Hoofdstedelijk Gewest, Praha, Hovedstaden, Oberbayern, Hamburg, Southern, Eastern and Midland, Île de France, Luxembourg, Bucuresti - Ilfov
%\item Club 2:\\
%Prov. Antwerpen, Prov. Brabant wallon, Stuttgart, Tübingen   , Oberpfalz, Mittelfranken, Bremen, Darmstadt, Braunschweig, Provincia Autonoma di Bolzano/Bozen, Kypros, Sostines regionas, Budapest, Utrecht, Noord-Holland, Wien, Salzburg, Tirol, Vorarlberg, Warszawski stoleczny, Bratislavský kraj, Helsinki-Uusimaa, Stockholm, Inner London - East
%\item Club 3:\\
%Prov. Limburg (BE), Prov. Oost-Vlaanderen, Prov. Vlaams-Brabant, Prov. West-Vlaanderen, Yugozapaden, Jihovýchod, Syddanmark, Midtjylland, Nordjylland, Karlsruhe, Freiburg, Niederbayern, Oberfranken, Unterfranken, Schwaben, Berlin, Brandenburg, Gießen, Kassel, Mecklenburg-Vorpommern, Hannover, Weser-Ems, Düsseldorf, Köln, Münster, Detmold, Arnsberg, Koblenz, Trier, Rheinhessen-Pfalz, Saarland, Dresden, Chemnitz, Leipzig, Sachsen-Anhalt, Schleswig-Holstein,Thüringen, Eesti, Attiki, Notio Aigaio, País Vasco, Comunidad Foral de Navarra
%, Aragón, Comunidad de Madrid, Cataluña, Illes Balears, Alsace,  Midi-Pyrénées, Rhône-Alpes, Provence-Alpes-Côte d'Azur, Guadeloupe, Martinique, Piemonte, Valle d'Aosta/Vallée d'Aoste, Liguria, Lombardia, Provincia Autonoma di Trento, Veneto, Friuli-Venezia Giulia, Emilia-Romagna, Toscana, Marche, Lazio, Vidurio ir vakaru Lietuvos regionas, Malta, Groningen, Overijssel, Gelderland, Flevoland, Zuid-Holland, Zeeland, Noord-Brabant, Limburg (NL), Burgenland (AT), Niederösterreich, Kärnten, Steiermark, Oberösterreich, Slaskie, Wielkopolskie, Dolnoslaskie, Pomorskie, Lódzkie, Área Metropolitana de Lisboa, Regiao Autónoma da Madeira (PT), Nord-Vest, Centru, Sud-Est, Vest, Zahodna Slovenija, Länsi-Suomi, Etelä-Suomi, 	
%Pohjois- ja Itä-Suomi, Åland, Östra Mellansverige, Småland med öarna, Sydsverige, Västsverige, Norra Mellansverige, Mellersta Norrland, Övre Norrland, Cheshire, Herefordshire, Worcestershire and Warwickshire, Bedfordshire and Hertfordshire, Outer London - West and North West, Berkshire, Buckinghamshire and Oxfordshire, Surrey, East and West Sussex, Hampshire and Isle of Wight, Gloucestershire, Wiltshire and Bath/Bristol area, North Eastern Scotland, Eastern Scotland


%\item Club 4:\\
P%rov. Liège, Prov. Namur, Strední Čechy, Jihozápad, Severovýchod, Strední Morava, Moravskoslezsko, Sjælland, Lüneburg, Galicia, La Rioja, Castilla y León, Canarias, Centre - Val de Loire, Bourgogne, Haute-Normandie, Nord-Pas-de-Calais, Champagne-Ardenne, Pays-de-la-Loire, Bretagne, Aquitaine, Poitou-Charentes, Auvergne, Corse, Umbria, Abruzzo, Basilicata, Latvija, Nyugat-Dunántúl, Friesland (NL), Drenthe, Malopolskie, Lubuskie, Opolskie, Mazowiecki regionalny, Algarve, Sud - Muntenia, Sud-Vest Oltenia, Západné Slovensko, Cumbria, Greater Manchester, North Yorkshire, West Yorkshire, Leicestershire, Rutland and Northamptonshire, West Midlands, East Anglia, Essex, Outer London - South, Kent, East Wales, Highlands and Islands, West Central Scotland

%\item Club 5:\\
	
%Prov. Hainaut, Prov. Luxembourg (BE), Severozapaden, Severen tsentralen, Severoiztochen, Yugoiztochen, Yuzhen tsentralen, Severozápad, Northern and Western, Voreio Aigaio, Kriti, Anatoliki Makedonia, Thraki, Kentriki Makedonia, Dytiki Makedonia, Ipeiros, Thessalia, Ionia Nisia, Dytiki Ellada, Sterea Ellada, Peloponnisos, Principado de Asturias, Cantabria, Castilla-la Mancha, Extremadura, Comunitat Valenciana, Andalucía, Región de Murcia, Ciudad de Ceuta, Ciudad de Melilla, Franche-Comté, Basse-Normandie, Picardie, Lorraine, Limousin, Languedoc-Roussillon, Jadranska Hrvatska, Kontinentalna Hrvatska (NUTS 2016), Molise, Campania, Puglia, Calabria, Sicilia, Sardegna, Pest, Közép-Dunántúl, Dél-Dunántúl, Észak-Magyarország, Dél-Alföld, Észak-Alföld, Zachodniopomorskie, Kujawsko-Pomorskie, Warminsko-Mazurskie, Swietokrzyskie, Lubelskie, Podkarpackie, Podlaskie, Norte, Centro (PT), Alentejo, Nord-Est, Vzhodna Slovenija, Stredné Slovensko, Východné Slovensko, Tees Valley and Durham, Northumberland and Tyne and Wear, Lancashire, Merseyside, East Yorkshire and Northern Lincolnshire, South Yorkshire, Derbyshire and Nottinghamshire, Lincolnshire, Shropshire and Staffordshire, Outer London - East and North East, Dorset and Somerset, Cornwall and Isles of Scilly, Devon, West Wales and The Valleys, Southern Scotland, Northern Ireland


%\end{itemize}


%\subsection{Club membership (2015)}
%\begin{itemize}
%\item Club 1:\\
%Wien, Salzburg, Région de Bruxelles-Capitale / Brussels Hoofdstedelijk Gewest, Praha, Stuttgart, Oberbayern, Hamburg, Darmstadt, Hovedstaden, Helsinki-Uusimaa, Île de France, Groningen, Noord-Holland, Bucuresti - Ilfov, Stockholm, Bratislavský kraj, Inner London - East 
%\item Club 2:\\
%Oberösterreich, Tirol, Vorarlberg, Prov. Antwerpen \& Vlaams-Brabant, Prov. Brabant Wallon, Kypros, Karlsruhe, Freiburg, Tübingen, Niederbayern, Oberpfalz, Oberfranken, Mittelfranken, Unterfranken, 
%Schwaben, Berlin, Bremen, Braunschweig, Düsseldorf, Köln, Rheinhessen-Pfalz, Southern and Eastern, Valle d'Aosta/Vallée d'Aoste, Lombardia, Provincia Autonoma di Bolzano/Bozen, Provincia Autonoma di Trento, Emilia-Romagna, Utrecht, Zuid-Holland, Noord-Brabant, Mazowieckie, Berkshire, Buckinghamshire and Oxfordshire, North Eastern Scotland 
%\item Club 3: \\
%Burgenland (AT), Niederösterreich, Kärnten, Steiermark,
%Prov. Limburg (BE), Prov. Oost-Vlaanderen, Prov. West-Vlaanderen, Yugozapaden, Jihovýchod, Brandenburg, Gießen, Kassel, Mecklenburg-Vorpommern, Hannover, Lüneburg, Weser-Ems, Münster, Detmold, Arnsberg, Koblenz, Trier, Saarland, Dresden, Chemnitz, Leipzig,        Sachsen-Anhalt, Schleswig-Holstein, Thüringen, Syddanmark, Midtjylland, Eesti, Attiki, Notio Aigaio, País Vasco, Comunidad Foral de Navarra, Aragón, Comunidad de Madrid, Cataluña, Canarias (ES), Länsi-Suomi, Etelä-Suomi, Alsace, Pays de la Loire, Midi-Pyrénées, Rhône-Alpes,      Provence-Alpes-Côte d'Azur, Corse, Közép-Magyarország, Piemonte, Liguria, Veneto, Friuli-Venezia Giulia, Toscana, Marche, Lazio, Lietuva, Friesland (NL), Overijssel, Gelderland, Flevoland, Zeeland, Limburg (NL), Lódzkie, Slaskie, Wielkopolskie, Dolnoslaskie, Pomorskie, Área Metropolitana de Lisboa, Centru, Sud-Est, Vest, Östra Mellansverige, Småland med öarna, Sydsverige, Västsverige, Norra Mellansverige, Mellersta Norrland, Övre Norrland, Zahodna Slovenija, Západné Slovensko, Cumbria, Cheshire (NUTS 2006), East Anglia, Bedfordshire and Hertfordshire, Outer London - West and North West, Surrey, East and West Sussex, Hampshire and Isle of Wight, Gloucestershire, Wiltshire and Bristol/Bath area 
%\item Club 4: \\
 %Prov. Liège, Střední Čechy, Jihozápad, Severozápad, Severovýchod, Střední Morava, Moravskoslezsko, Voreio Aigaio, Kriti, Galicia, Principado de Asturias, Cantabria, La Rioja, Castilla y León,                    Comunidad Valenciana, Illes Balears, Región de Murcia, Champagne-Ardenne, Picardie, Haute-Normandie, Centre (FR), Basse-Normandie,  Bourgogne, Lorraine, Franche-Comté, Bretagne, Poitou-Charentes, Aquitaine,          Limousin, Auvergne, Languedoc-Roussillon, Közép-Dunántúl, Nyugat-Dunántúl, Border, Midland and Western, Abruzzo,                Molise, Basilicata, Sardegna, Umbria, Latvija, Drenthe,                 Malopolskie, Lubelskie, Swietokrzyskie, Podlaskie,                  Zachodniopomorskie, Lubuskie, Opolskie, Kujawsko-Pomorskie, Centro (PT), Nord-Vest, Sud - Muntenia, Sud-Vest Oltenia, Vzhodna Slovenija, Stredné Slovensko, Východné Slovensko, Tees Valley and Durham, Northumberland and Tyne and Wear,  Greater Manchester, Lancashire, Merseyside, East Yorkshire and Northern Lincolnshire, North Yorkshire, South Yorkshire,  West Yorkshire, Derbyshire and Nottinghamshire, Leicestershire, Rutland and Northamptonshire, Lincolnshire, Herefordshire, Worcestershire and Warwickshire, Shropshire and Staffordshire, West Midlands, Essex, Outer London - East and North East, Outer London - South, Kent, Dorset and Somerset, Cornwall and Isles of Scilly, Devon, East Wales, Eastern Scotland, South Western Scotland, Highlands and Islands, Northern Ireland (UK)
%\item Club 5:\\
%Severozapaden, Severen tsentralen, Severoiztochen, Yugoiztochen, Yuzhen tsentralen, Kentriki Makedonia, Castilla-la Mancha, Extremadura, Andalucía, Dél-Dunántúl, Észak-Magyarország, Észak-Alföld, Dél-Alföld, Campania, Puglia, Calabria, Sicilia, Podkarpackie, Warminsko-Mazurskie, Norte, Nord-Est, West Wales and The Valleys

%\end{itemize}



\begin{table}[!htbp] \centering 
 \caption{\textbf{Convergence club classification before merging.}} 
  \label{Table_clubs2} 
 \scalebox{0.85}{
\begin{tabular}{@{\extracolsep{5pt}}lcll} 
\\[-1.8ex]\hline 
\hline \\[-1.8ex] 
Club & \multicolumn{1}{c}{N} & \multicolumn{1}{c}{log(t)} & \multicolumn{1}{c}{t value}  \\ 
\hline \\[-1.8ex] 
Club 1 & 10 &  0.0194  & 0.113 \\ 
Club 2 & 23 & 0.146 & 0.930  \\ 
Club 3 & 21 & .168   & 1.95   \\ 
Club 4 & 27 & 0.101  & 0.987  \\ 
Club 5 & 63 & 0.0919 & 0.746  \\
Club 6 & 52 & 0.163 & 1.53  \\ 
Club 7 & 79 & -0.0963 & -0.968  \\ 
\hline \\[-1.8ex]
%\textit{Note:}  & \multicolumn{1}{r}{$^{**}$p$<$0.05} \\

\end{tabular}
}
\end{table}


\begin{table}[!htbp] \centering 
    \caption{\textbf{Convergence club classification before merging (2015).}} \label{Table_clubs2_2015}
 \scalebox{0.85}{
\begin{tabular}{@{\extracolsep{5pt}}lcll} 
\\[-1.8ex]\hline 
\hline \\[-1.8ex] 
Club & \multicolumn{1}{c}{N} & \multicolumn{1}{c}{log(t)} & \multicolumn{1}{c}{t value}  \\ 
\hline \\[-1.8ex] 
Club 1 & 17 &  0.26612  & 2.693** \\ 
Club 2 & 36 & -0.0148 & -0.088 \\ 
Club 3 & 25 &  0.3725  & 2.334** \\ 
Club 4 & 24 & 0.1924 & 0.974 \\ 
Club 5 & 30 & 0.02397 & 0.223  \\ 
Club 6 & 14 & 0.04248 & 0.328 \\ 
Club 7 & 28 & 0.08075  & 0.720 \\ 
Club 8 & 30 & 0.2519  & 1.895** \\ 
Club 9 & 40 & 0.1067   & 0.692  \\ 
Club 10 & 29 &  -0.09851   & -0.638 \\ 
\hline \\[-1.8ex]
\textit{Note:}  & \multicolumn{1}{r}{ $^{**}$p$<$0.05} \\
\end{tabular} 
}
\end{table}



\begin{table}[!htbp] \centering 
  \caption{\textbf{Descriptive statistics of the main explanatory variables based on the 267 regions with neighbors (year 2008).}} 
  \label{descriptive_vars_filtered} 
\begin{tabular}{@{\extracolsep{5pt}}lcccccc} 
\\[-1.8ex]\hline 
\hline \\[-1.8ex] 
Statistic  & \multicolumn{1}{c}{Mean} & \multicolumn{1}{c}{St. Dev.} & \multicolumn{1}{c}{Min} & \multicolumn{1}{c}{Pctl(25)} & \multicolumn{1}{c}{Pctl(75)} & \multicolumn{1}{c}{Max} \\ 
\hline \\[-1.8ex] 
Spec. Agriculture & 0.055 & 0.065 & 0.001 & 0.017 & 0.063 & 0.428 \\ 
Spec. Industry  & 0.192 & 0.072 & 0.056 & 0.139 & 0.238 & 0.388 \\ 
Spec. Trade, Accommodation  & 0.238 & 0.037 & 0.147 & 0.214 & 0.255 & 0.453 \\ 
Spec. Finances  & 0.027 & 0.014 & 0.008 & 0.017 & 0.034 & 0.106 \\ 
Spec. Other Services & 0.048 & 0.017 & 0.010 & 0.037 & 0.058 & 0.113 \\ 
Tertiary Edu. Share & 0.146 & 0.094 & 0.030 & 0.063 & 0.210 & 0.446 \\ 
Share of Scientists & 0.006 & 0.004 & 0.001 & 0.003 & 0.007 & 0.026 \\ 
Governance Quality & 0.173 & 1.012 & $-$2.398 & $-$0.665 & 0.930 & 2.120 \\  
Business Services & 0.120 & 0.060 & 0.000 & 0.076 & 0.154 & 0.329 \\ 
Investment & 8.904 & 0.919 & 6.139 & 8.305 & 9.489 & 11.825 \\  
Initial GDP level & 10.048 & 0.388 & 8.868 & 9.870 & 10.284 & 11.115 \\ 
Patent activity  & 115.381 & 132.421 & 0.177 & 13.712 & 160.176 & 626.102 \\ 
Highest neighbor Income & 10.262 & 0.411 & 9.159 & 10.071 & 10.513 & 11.826 \\ 
\hline \\[-1.8ex] 
\end{tabular} 
\end{table} 


\begin{table}[!htbp] \centering 
\resizebox{0.75 \textwidth}{!}{\begin{minipage}{\textwidth}
  \captionsetup{justification=centering}

  \caption{\textbf{Regression results: Logistic regression with Sectoral and Human capital variables. Merged Convergence Clubs, FAMD imputation and robust standard errors.}} 
  \label{model_famd_filtered_spec1}
\centering
\begin{tabular}{@{\extracolsep{5pt}}lc} 
\\[-1.8ex]\hline 
\hline \\[-1.8ex] 
 & \multicolumn{1}{c}{\textit{Dependent variable:}} \\ 
\cline{2-2} 
\\[-1.8ex] & Club \\ 
\hline \\[-1.8ex] 
 Initial GDP level & $-$7.492$^{***}$ \\ 
  & (0.860) \\ 
  & \\ 
 Investment & 0.002 \\ 
  & (0.208) \\ 
  & \\ 
 Spec. Agriculture  & $-$14.049$^{***}$ \\ 
  & (3.769) \\ 
  & \\ 
 Spec. Finances & $-$36.010$^{***}$ \\ 
  & (13.605) \\ 
  & \\ 
 Spec. Industry & $-$15.963$^{***}$ \\ 
  & (2.927) \\ 
  & \\ 
 Spec. Trade, Accommodation & $-$3.659 \\ 
  & (4.369) \\ 
  & \\ 
 Spec. Other Services & 17.237 \\ 
  & (11.817) \\ 
  & \\ 
 Business Services & $-$6.603$^{*}$ \\ 
  & (3.529) \\ 
  & \\ 
 Share of Scientists  & $-$152.849$^{***}$ \\ 
  & (39.670) \\ 
  & \\ 
 Tertiary Edu. Share & 3.504$^{**}$ \\ 
  & (1.599) \\ 
  & \\ 
\hline \\[-1.8ex] 
Observations & 267 \\ 
\hline 
\hline \\[-1.8ex] 
\textit{Note:}  & \multicolumn{1}{r}{$^{*}$p$<$0.1; $^{**}$p$<$0.05; $^{***}$p$<$0.01} \\ 
\end{tabular}
\end{minipage}
}
\end{table} 


\begin{table}[!htbp] \centering 
\resizebox{0.75 \textwidth}{!}{\begin{minipage}{\textwidth}
  \captionsetup{justification=centering}
  \caption{\textbf{Regression results: Logistic regression with with Geographic determinants. Merged Convergence Clubs, FAMD imputation and robust standard errors.}} 
  \label{model_famd_filtered_spec2}
\centering
\begin{tabular}{@{\extracolsep{5pt}}lc} 
\\[-1.8ex]\hline 
\hline \\[-1.8ex] 
 & \multicolumn{1}{c}{\textit{Dependent variable:}} \\ 
\cline{2-2} 
\\[-1.8ex] & Club \\ 
\hline \\[-1.8ex] 
 Initial GDP level & $-$7.633$^{***}$ \\ 
  & (0.786) \\ 
  & \\ 
 Investment  & 0.180 \\ 
  & (0.212) \\ 
  & \\ 
 Capital & $-$0.977$^{***}$ \\ 
  & (0.362) \\ 
  & \\ 
 Metro area & $-$1.073$^{***}$ \\ 
  & (0.384) \\ 
  & \\ 
 Governance Quality & 0.248 \\ 
  & (0.187) \\ 
  & \\ 
 Highest neighbor Income & 1.236$^{***}$ \\ 
  & (0.388) \\ 
  & \\ 
 Border Region & $-$1.257$^{***}$ \\ 
  & (0.277) \\ 
  & \\ 
\hline \\[-1.8ex] 
Observations & 267 \\ 
\hline 
\hline \\[-1.8ex] 
\textit{Note:}  & \multicolumn{1}{r}{$^{*}$p$<$0.1; $^{**}$p$<$0.05; $^{***}$p$<$0.01} \\ 
\end{tabular}
\end{minipage}
}
\end{table} 

% Table created by stargazer v.5.2.2 by Marek Hlavac, Harvard University. E-mail: hlavac at fas.harvard.edu
% Date and time: st, čvc 13, 2022 - 23:17:43
\begin{table}[!htbp] \centering 
\resizebox{0.75 \textwidth}{!}{\begin{minipage}{\textwidth}
  \captionsetup{justification=centering}
  \caption{\textbf{Regression results: Logistic regression with merged Convergence Clubs, FAMD imputation and robust standard errors.}} 
  \label{model_famd_neigh_filtered} 
\centering
\begin{tabular}{@{\extracolsep{5pt}}lc} 
\\[-1.8ex]\hline 
\hline \\[-1.8ex] 
 & \multicolumn{1}{c}{\textit{Dependent variable:}} \\ 
\cline{2-2} 
\\[-1.8ex] &   \\ 
\hline \\[-1.8ex] 
 Initial GDP level & $-$9.555$^{***}$ \\ 
  & (1.081) \\ 
  & \\ 
 Investment & 0.095 \\ 
  & (0.245) \\ 
  & \\ 
 Spec. Agriculture  & $-$17.175$^{***}$ \\ 
  & (4.322) \\ 
  & \\ 
 Spec. Finances & $-$26.968$^{*}$ \\ 
  & (15.169) \\
  & \\ 
 Spec. Industry & $-$19.760$^{***}$ \\ 
  & (3.468) \\ 
  & \\ 
 Spec. Trade, Accommodation & $-$3.790 \\ 
  & (4.763) \\ 
  & \\ 
 Spec. Other Services  & 12.003 \\ 
  & (12.120) \\  
  & \\ 
 Business Services & $-$8.381$^{**}$ \\ 
  & (3.880) \\ 
  & \\ 
 Share of Scientists & $-$94.294$^{**}$ \\ 
  & (45.829) \\  
  & \\ 
 Tertiary Edu. Share  & 2.688 \\ 
  & (1.671) \\  
  & \\ 
 Capital & $-$1.713$^{***}$ \\ 
  & (0.455) \\ 
  & \\ 
 Metro area  & $-$0.631 \\ 
  & (0.446) \\  
  & \\ 
 Governance Quality & $-$0.203 \\ 
  & (0.252) \\
  & \\
 Highest neighbor Income & 1.258$^{***}$ \\ 
  & (0.475) \\ 
  & \\ 
 Border Region & $-$1.123$^{***}$ \\ 
  & (0.310) \\ 
  & \\ 
\hline \\[-1.8ex] 
Observations & 276 \\ 
\hline 
\hline \\[-1.8ex] 
\textit{Note:}  &\multicolumn{1}{r}{$^{*}$p$<$0.1; $^{**}$p$<$0.05; $^{***}$p$<$0.01} \\ 
\end{tabular}
\end{minipage}
}
\end{table} 





% Table created by stargazer v.5.2.2 by Marek Hlavac, Harvard University. E-mail: hlavac at fas.harvard.edu
% Date and time: st, čvc 06, 2022 - 17:57:02
\begin{table}[!htbp] \centering 
\resizebox{0.75\textwidth}{!}{\begin{minipage}{\textwidth}
    \caption{\textbf{Marginal effects: Logistic regression with merged Clubs, FAMD imputation and robust standard errors (variables in 2008 levels).}}
  \label{famd_selected_marginal} 
\begin{tabular}{@{\extracolsep{5pt}} lccccc} 
\\[-1.8ex]\hline 
\hline \\[-1.8ex] 
 & Club 1 & Club 2 & Club 3 & Club 4 & Club 5 \\ 
\hline \\[-1.8ex] 
% log\_init\_gdp & $0.010$ & $0.102$ & $1.776$ & $$-$1.052$ & $$-$0.835$ \\ 
Initial GDP level & $0.010$$^{*}$ & $0.102$$^{***}$ & $1.776$$^{***}$ & $ $-$1.052$$^{***}$ & $ $-$0.835$$^{***}$ \\ 
& (0.006) &  (0.033) & (0.245) & (0.197) & (0.134)\\
&\\
Investment & $0.000$ & $ $-$0.004$ & $ $-$0.063$ & $0.037$ & $0.030$ \\ 
& (0.000) &  (0.003) & (0.054) & (0.033) & (0.025)\\
&\\
\multicolumn{6}{c}{\textit{Sectoral Variables}}\\

Spec. Agriculture & $0.021$$^{*}$ & $0.213$$^{**}$ & $3.705$$^{***}$ & $ $-$2.196$$^{**}$ & $ $-$1.743$$^{***}$ \\ 
& (0.012) &  (0.089) & (1.114) & (0.735) & (0.520)\\
&\\
Spec. Finances  & $0.033$ & $0.342$ & $5.938$$^{**}$ & $ $-$3.519$$^{**}$ & $ $-$2.793$$^{*}$ \\ 
& (0.024) &  (0.210) & (2.840) & (1.673) & (1.443)\\
&\\
Spec. Industry & $0.022$$^{*}$ & $0.231$$^{***}$ & $4.012$$^{***}$ & $ $-$2.378$$^{***}$ & $ $-$1.887$$^{***}$ \\ 
& (0.013) &  (0.078) & (0.743) & (0.539) & (0.372)\\
&\\
Spec. Trade, Accommodation & $0.011$ & $0.114$$^{*}$ & $1.982$$^{*}$ & $ $-$1.174$$^{*}$ & $ $-$0.932$$^{*}$ \\
& (0.008) &  (0.068) & (1.010) & (0.625) & (0.474)\\
&\\
Spec. Other Services & $ 0.001$ & $ 0.007$ & $ 0.123$ & $ $-$0.073$ & $ $-$0.058$ \\
& (0.018) &  (0.184) & (3.198) & (1.895) & (1.505)\\
&\\
Business Services & $0.007$ & $0.072$ & $1.246$ & $ $-$0.738$ & $ $-$0.586$ \\
& (0.005) &  (0.050) & (0.858) & (0.519) & (0.399)\\
&\\
\multicolumn{6}{c}{\textit{Human Capital Variables}}\\
Share of Scientists & $0.160$$^{*}$ & $1.638$$^{**}$ & $28.472$$^{***}$ & $ $-$16.876$$^{***}$ & $ $-$13.394$$^{***}$ \\
& (0.093) &  (0.683) & (8.804) & (5.625) & (4.263)\\
&\\
Governance Quality & $0.000$ & $0.002$ & $0.040$ & $ $-$0.023$ & $ $-$0.019$ \\
& (0.000) &  (0.003) & (0.058) & (0.034) & (0.027)\\
&\\
Tertiary Edu. Share & $ $-$0.004$ & $ $-$0.046$$^{*}$ & $ $-$0.798$$^{**}$ & $0.473$$^{**}$ & $0.375$$^{**}$ \\
& (0.003) &  (0.024) & (0.372) & (0.227) & (0.178)\\
&\\
\multicolumn{6}{c}{\textit{Geographic Determinants}}\\

Capital & $0.003$ & $0.032$$^{*}$ & $0.253$$^{***}$ & $ $-$0.181$$^{***}$ & $ $-$0.107$$^{***}$ \\
& (0.002) &  (0.017) & (0.060) & (0.055) & (0.026)\\
&\\
Metro area & $0.001$ & $0.008$$^{*}$ & $0.173$$^{*}$ & $ $-$0.089$$^{**}$ & $ $-$0.092$ \\ 
& (0.001) &  (0.005) & (0.093) & (0.044) & (0.056)\\
&\\
Border Region & $0.001$ & $0.010$$^{**}$ & $0.173$$^{**}$ & $ $-$0.100$$^{**}$ & $ $-$0.084$$^{**}$ \\ 
& (0.001) &  (0.005) & (0.070) & (0.042) & (0.035)\\

\hline \hline \\[-1.8ex]
McFadden's $R^{2}$: 0.365 \\
Observations: 276\\
No. of Parameters 14\\
\hline
\end{tabular}
\begin{tablenotes}
\small 
\item Note: $^{*}$p$<$0.1; $^{**}$p$<$0.05; $^{***}$p$<$0.01
\end{tablenotes}
\end{minipage}}
\end{table} 


\begin{table}[!htbp] \centering 
\resizebox{0.8\textwidth}{!}{\begin{minipage}{\textwidth}
  \caption{\textbf{Marginal effects: Logistic regression with merged Clubs, initial values from 2003 are used. Sectoral variables use NACE Rev. 1.1 clasification  FAMD imputation and robust standard errors. 9 regions with no neighboring regions have been removed.}} 
  \label{famd_selected_marginal_ecosys_robust_filtered} 
\begin{tabular}{@{\extracolsep{5pt}} lccccc} 
\\[-1.8ex]\hline 
\hline \\[-1.8ex] 
 & Club 1 & Club 2 & Club 3 & Club 4 & Club 5 \\ 
\hline \\[-1.8ex] 
Initial GDP level & $0.013$$^{**}$ & $0.112$$^{***}$ & $1.043$$^{***}$ & $ $-$0.570$$^{***}$ & $  $-$0.599$$^{***}$ \\
& (0.007) &  (0.034) & (0.208) & (0.135) & (0.125)\\
&\\
Investment & $ $-$0.001$ & $ $-$0.008$$^{*}$  & $ $-$0.075$$^{*}$ & $0.041$$^{*}$ & $0.043$$^{*}$ \\
&\\
& (0.001) &  (0.005) & (0.039) & (0.022) & (0.022)\\
&\\
\multicolumn{6}{c}{\textit{Sectoral Variables}}\\
Spec. Agriculture, fishing & $0.019$ & $0.158$ & $1.472$$^{*}$ & $ $-$0.804$ & $ $-$0.845$$^{*}$ \\ 
& (0.013) &  (0.104) & (0.861) & (0.503) & (0.479)\\
&\\
Spec. Industry & $0.036$$^{*}$ & $0.302$$^{***}$ & $2.820$$^{***}$ & $ $-$1.540$$^{***}$ & $ $-$1.619$$^{***}$ \\
& (0.019) &  (0.105) & (0.615) & (0.414) & (0.353)\\
&\\
Spec. Trade, Accomodation & $ $-$0.004$ & $ $-$0.030$ & $ $-$0.280$ & $0.153$ & $0.161$ \\ 
& (0.014) &  (0.111) & (1.056) & (0.575) & (0.606)\\
&\\
Spec. Finance, Real estate & $0.077$$^{*}$ & $0.644$$^{***}$ & $6.007$$^{***}$ & $ $-$3.280$$^{***}$ & $ $-$3.448$$^{**}$ \\ 
& (0.042) &  (0.242) & (1.725) & (1.097) & (0.945)\\
&\\
\multicolumn{6}{c}{\textit{Human Capital Variables}}\\
Share of Scientists & $0.230$ & $1.921$$^{*}$ & $17.921$$^{**}$ & $ $-$9.787$$^{**}$ & $ $-$10.286$$^{*}$ \\
& (0.159) &  (1.108) & (8.677) & (4.788) & (5.228)\\
&\\
Tertiary Edu. Share & $ $-$0.010$ & $ $-$0.079$$^{*}$ & $ $-$0.740$$^{**}$ & $0.404$$^{**}$ & $0.424$$^{**}$ \\
& (0.006) &  (0.042) & (0.359) & (0.200) & (0.210)\\
&\\
\multicolumn{6}{c}{\textit{Geographic Determinants}}\\
Capital & $0.005$ & $0.037$ & $0.195$$^{***}$ & $ $-$0.132$$^{**}$ & $ $-$0.104$$^{***}$ \\
& (0.004) &  (0.023) & (0.066) & (0.056) & (0.036)\\
&\\
Metro area & $0.001$ & $0.009$ & $0.092$ & $ $-$0.046$ & $ $-$0.056$ \\
& (0.001) &  (0.007) & (0.080) & (0.037) & (0.051)\\
&\\
Highest neighbor Income & $ $-$0.004$ & $ $-$0.029$$^{**}$ & $ $-$0.273$$^{**}$ & $ 0.149$$^{**}$ & $ 0.157$$^{**}$ \\ 
& (0.002) &  (0.013) & (0.111) & (0.062) & (0.065)\\
&\\
Border region & $0.003$$^{*}$ & $0.029$$^{***}$ & $0.258$$^{***}$ & $ $-$0.132$$^{***}$ & $ $-$0.158$$^{***}$ \\ 
& (0.002) &  (0.010) & (0.067) & (0.038) & (0.043)\\
\hline \hline \\[-1.8ex]
McFadden's $R^{2}$: 0.315 \\
Observations: 267\\
No. of Parameters 12\\
\hline
\end{tabular}
\begin{tablenotes}
\small 
\item Note: $^{*}$p$<$0.1; $^{**}$p$<$0.05; $^{***}$p$<$0.01
\end{tablenotes}
\end{minipage}}
\end{table} 




\begin{figure}%
\centering 
  {\includegraphics[scale = 0.5]{RTC_mer_overall.png} }
  \caption{\textbf{Average transition paths across merged convergence clubs. Diverging regions are represented by the highest transition curve.}}
  \label{paths_overall}
  \end{figure}



\begin{figure}%
    \centering
    \includegraphics[scale = 0.5]{RTC_mer_club2.png}
    \caption{\textbf{Transition paths for members of the second Club.}}
    \label{paths2}
\end{figure}


\begin{figure}%
    \centering
    \includegraphics[scale = 0.5]{RTC_mer_club3.png}
    \caption{\textbf{Transition paths for members of the third Club.}}
    \label{paths3}
\end{figure}

\begin{figure}%
    \centering
    \includegraphics[scale = 0.5]{RTC_mer_club4.png}
    \caption{\textbf{Transition paths for members of the forth Club}}
    \label{paths4}
\end{figure}


%\section{Additional material}

%\subsection{Data availability}
%The replication package can be found using Mendeley Data under DOI: 10.17632/mcmkwc2394.1.
%"Pintera, Jan (2023), “Regional Convergence in the European Union - Factors of Growth Between the Great Recession and the COVID Crisis”, Mendeley Data, V1, doi: 10.17632/mcmkwc2394.1"

\end{document}

