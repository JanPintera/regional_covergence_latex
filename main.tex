\documentclass[11pt]{article}
\usepackage[numbib]{tocbibind}
\usepackage[utf8]{inputenc}
\usepackage{lmodern}
\usepackage[T1]{fontenc}
\usepackage{natbib}
\usepackage{amsmath}
\usepackage{caption}
\usepackage{graphicx,subcaption}
\usepackage[a4paper, total={6in, 8in}]{geometry}
\setcitestyle{authoryear,open={(},close={)}} %Citation-related commands
\usepackage[labelfont=bf]{caption}
\usepackage[flushleft]{threeparttable}


\makeatletter
\renewcommand{\maketitle}{\bgroup\setlength{\parindent}{0pt}
\begin{flushright}
  \textbf{\@title}\\
  \vspace{5mm}
  \@author\\
  \vspace{5mm}
  \@date
\end{flushright}\egroup
}
\makeatother

\renewenvironment{abstract}
 {\small
  \begin{flushleft}
  \bfseries \abstractname\vspace{-.5em}\vspace{0pt}
  \end{flushleft}
  \list{}{%
    \setlength{\leftmargin}{0mm}% <---------- CHANGE HERE
    \setlength{\rightmargin}{\leftmargin}%
  }%
  \item\relax}
 {\endlist}

\title{Regional Convergence in the European Union - Factors of Growth Between the Great Recession and the COVID Crisis}
\author{
        \begin{large}Jan Pintera\end{large} \\\vspace{5mm} \begin{small} Institute of Economic Studies, Faculty of Social Sciences, Charles University,\\ Prague, Czech Republic\\
        Email (corresponding author): jan.pintera@fsv.cuni.cz 
        \end{small}
}
\date{September 2023}
\def \Keywords {Club Convergence, European Regions, log $t$ test, Logistic regression}



\begin{document}

\maketitle


\thispagestyle{empty}
\begin{abstract}
In this paper, we provide a new look at convergence in the EU while focusing on development at the regional level between the Great Recession and the recent COVID crisis. We use the log $t$ convergence test by \citet{phillips2007transition} to analyze convergence in the level of income among the European regions. We identified five convergence clubs rather than supporting the overall convergence hypothesis. Furthermore, we investigated the determinants of convergence club membership using logistic regression. Our results confirmed high inequality within the member states and a shifting geographic pattern of the top performing regions, with the increasing prominence of the manufacturing core in southern Germany and the surrounding areas. We found a positive association between membership in higher clubs, research and patent activities, and specialization in manufacturing. We also confirmed the good economic performance of capital cities and the main metropolitan areas.

\bigskip


\textbf{JEL Classification:} C23, C40, R11 \\
\textbf{Keywords:}  \Keywords \\

\bigskip

\textbf{Acknowledgements:} This work was supported by the Grant Agency of the Czech Republic, Grant No. GACR no. 20-14990S. I would like to thank my supervisor, Jaromír Baxa, for his many insightful comments on this work.

\end{abstract}
\clearpage
\setcounter{page}{1}

\section{Introduction}
The gap between the prosperous and divergent regions has recently received attention from policymakers and economists. The topic now even has the potential to become an important part of a new economic paradigm that moves away from globalisation toward more equal sharing of resources and opportunities across different regions \citep{rodrik_2022}. This general shift can be seen as a recognition of the deep socioeconomic impact that regional differences have, including serious consequences for a political consensus. The convergence of EU regions is also one of the goals of EU cohesion policy. However, despite large transfers and a long history of such policies, large differences in income persist and the speed and nature of convergence are changing over time \citep{eckey2007convergence,zarotiadis2013european, iammarino2019regional}.
This article contributes by providing evidence of income convergence across the regions of the European Union between the Great Recession and the Covid-related crisis.

Previous empirical research has suggested several distinguishing patterns of convergence in the EU. It is often pointed out that economic development in large urban areas, and above all capital cities, is found to be significantly more dynamic than in the rest of the regions, leading to high economic inequality within the EU states \citep{geppert2008regional}. This seems to be especially valid for the new member states, with capital cities and main urban regions already among the EU's richest regions by GDP per capita, while many other regions lag behind, and the overall discrepancy between the old and the new member states diminishes rather slowly. The implication is that inequality within the CEE countries is considered especially high \citep{geppert2008regional,sme2012regional, smketkowski2013regional}.


We investigate the dynamics of regional income convergence with the main focus on finding out whether there is a convergence among all the EU countries or whether there is evidence of convergence clubs, in which case certain regions converge to each other but without a tendency to overall convergence. The work uses a test by \citet{phillips2007transition}, which has been used to analyze convergence at various levels and, in the case of the EU regions, and usually led to the identification of several convergence clubs \citep{bartkowska2012regional, borsi2015evolution, von2017regional}. Its design is especially suited for investigating the club convergence that can be expected among EU regions because it allows for an endogenous spectrum of transitional behaviour among the regions. Furthermore, we employ logistic regression on the resulting convergence clubs, examining the contribution of a wide range of variables suggested in the economic growth literature to the current state of convergence in the EU.

% Topic of this thesis
This work contributes to the existing literature in two ways. First, it examines the economic convergence in the EU between two major crises (2008 - 2019) and how the convergence dynamic has changed in the last years before the COVID-related crisis. This period was previously not discussed in its entirety. Our approach allows us to compare two important states of the European economy, 2015 as the year by which most European countries reached the final recovery after the Great Recession in terms of output per capita, and 2019 as representing a point after a consistent growth period and just before the COVID outbreak. As its second contribution, this work attempts to draw a connection between various socioeconomic characteristics of the EU regions and their relative economic performance.

Our results imply the club convergence hypothesis rather than general convergence among all regions. Next, we show that the convergence patterns evolved during the post-crisis recovery and the late 2010s expansion. Despite the continuing dominance of large agglomerations, there is a shift from a large continuous area of top-performing regions in the urbanized part of western Europe (the "Blue Banana") to a central European Manufacturing Core encompassing regions of both old and new EU member states and centered around southern Germany. We also found the worst-performing regions concentrated on both the eastern and southern peripheries.


The paper is structured as follows. Section 2 provides an overview of the methods used for convergence analysis. Section 3 briefly describes the \citeauthor{phillips2007transition}'s test. It also discusses the test results and the composition of the convergence clubs. Section 4 analyses the factors determining convergence club membership, and finally, Section 5 provides conclusions and policy implications.  


\section{Literature Review}\label{Methodology}
\subsection{The convergence hypothesis and its empirical measurement}

\citet{eckey2007convergence} bring a comprehensive survey of methods used for convergence analysis and their empirical results. Probably the most common form of the convergence test is the $\beta$-convergence which departs from the transitional income dynamics equation \eqref{eq3} and, in its basic version, has the following form:
\begin{equation} \label{eq1}\log y_{it} = a_{0} + (1 - a_{1})\log y_{it-1} + u_{it} \end{equation}
where we assume \(0 < a_{1} < 1\). In this setting, \(a_{1} > 0 \) implies convergence, as the growth rate \(\log y_{it} - \log y_{it-1}\) is inversely related to income in period $t$. The error term $u_{it}$ in the equation captures all sorts of temporary changes in the parameters of the production function.

The regression above is often augmented by control variables, creating the conditional convergence model \citep{sala1996regional}. It is worth noting that the $\beta$-convergence is designed to test the sign of the convergence parameter $\beta$ in Equation \ref{eq3} while we assume the parameter's homogeneity across time and cross-sections. This assumption is the main difference between the classical  $\beta$-convergence and the log $t$ test of \citet{phillips2007transition} used in this work, which relaxes this assumption.

The $\beta$-convergence is still widely used for estimating convergence. Recently, however, many other methods have been utilized to find convergence at both the state and regional level. The following part presents a brief description of some of these methods and their results.
  
%Despite growing consensus pointing at existence of several economic clubs in Europe as opposed to any form of overall convergence \citep{cutrini2019economic, iammarino2017regional}. The empirical research about the converge has a long history, with most of the papers concentrating on pre 2008-crisis era \citep{cutrini2019economic}.  %NOTE: this is a new paragraph

\citet{eckey2007convergence} conclude in their meta-analysis that most of the empirical convergence literature finds a significant, however rather small, convergence rate among European regions in the time span from 1980s to early 2000s. This finding is, however, not universal. Some papers find a high rate of convergence, whereas others fail to find convergence at all. This variety of conclusions is caused by differences in methods and in the number of regions included \citep{eckey2007convergence}. Other studies also suggest that the convergence pattern among the European regions was not stable in the post-war period. \citet{eckey2007convergence} note that several empirical works found that the convergence rate among the EU regions had diminished during the second half of the twentieth century. Others concluded that convergence in Europe appears to be U-shaped \citep{basile2001regional, geppert2008regional}. It seems to reach its lowest point around the beginning of the 1980s, accelerating again later on. \citet{geppert2008regional} also conclude that convergence is happening mainly at the national level, whereas regionally, we see a strengthening of metropolitan areas. From the newer contributions, \citet*{sme2012regional} find, using the $\beta$-convergence, only a weak tendency for regional convergence in the CEE countries, while in the case of some countries (for example, Poland), there was absolute divergence even when the capital region was excluded. %As far as the $\sigma$-convergence is concerned, the results seem as inconclusive as before, with conclusions differing based on the period and the sample of regions investigated \citep{eckey2007convergence}.
 
\subsection{Club convergence} 
 Another concept of convergence that appears to gain popularity in recent years is club convergence. \citet{iammarino2017regional}, notes a reversal in inequality trend after 1990s with medium-sized manufacturing cities stagnating while large metropolitan areas being the most dynamic, he also defines 4 convergence clubs in EU based on their level of GDP per capita.  % TODO finish!!!!
 A standard model for testing this type of convergence is the so-called LISA (Local Indicators of Spatial Association). It is based on finding clusters of regions with a value of a variable measuring spatial association between the regions constantly above average for a longer period of time. The statistic defined by \citet{getis1992analysis} is then used as this variable most often. One of the distinguishing features of the LISA methodology is that it is aimed at finding clusters of neighbouring regions. As a result, this method often leads to large clusters along the south-north  (as found, for example, in \citet{baumont2003spatial}) or between new and old members states \citep{eckey2007convergence}. The evidence for regional clustering, however, goes beyond this specified methods such as LISA, \citet{cartone2021does} utilize quantile regression within the traditional  beta-convergence framework, they confirm convergence among European but with speed of convergence being faster at the lower quantiles, this support the evidence clustering among European regions. \citeauthor{cartone2021does}'s results are supported by \citet{panzera2022impact} who, while investigating impact of inequality on economic growth, find that for less developed regions the converge process is taking place at a faster rate than for more developed ones.
 
Other approaches used for convergence clubs detection are kernel density estimation, Markov chains \citep{eckey2007convergence}, Bayesian \citep{fischer2015bayesian} and clustering methods \citep{maasoumi2008economic}. 
Among the methods used for convergence analysis is also the log $t$ test of \citet{phillips2007transition} used in this work. This framework can distinguish between absolute convergence and transitory divergence and also detect convergence in situations where traditional convergence tests fail due to its treatment of convergence as an asymptotic property \citep{bartkowska2012regional, borsi2015evolution}.
Moreover, as the results of the papers cited below suggest, the log $t$ test is able to produce convergence clubs, members of which are not geographic neighbours. This is in contrast with the LISA analysis described previously, which almost necessarily leads to large continuous blocks of regions.

The \cite{phillips2007transition}'s test
has been relatively widely used at the national level, as shown by \citet{borsi2015evolution, fritsche2011analysing, monfort2013real,apergis2010old} for Europe and by \citet{rodriguez2014there} for Latin America. The test is also used at the regional level where both \citet{bartkowska2012regional} and \citet{pinho2010regional} work with Western European regions. In the first case between years 1990 and 2005, in the second case between 1980 and 2007, both using NUTS 2 units. \citet{ghosh2013regional} employs the same method for states in India. \citet{bartkowska2012regional} found, using the log $t$ test, 5 separate clubs in western Europe. Analyzing the spatial distribution of these clubs, they found an agglomeration effect among the western European regions in the form of the tendency of the regions with capital cities to belong to higher convergence clubs than their neighbouring regions. Furthermore, there is also a tendency of regions within one country and regions belonging to the same club to cluster together \citep{bartkowska2012regional}. The newest contributions for western Europe are \citet{von2017regional}, finding 4 convergence Clubs in the 1980–2011 regional dataset and \citet{cutrini2019economic} finding 5 clubs among the EU regions, working with data ending in 2015. 

Other empirical works confirm an inclination of large urban areas and, above all capital cities, to grow faster than other regions. Especially often is then mentioned resulting large regional income inequality, particularly strong among the CEE countries (\citet{cuaresma2014determinants}; \citet{sme2012regional}; \citet{szendi2013convergence}; \citet{chapman2012income}; \citet{monastiriotis2011regional}. 
More specifically, \citet{sme2012regional} conclude, using the LISA analysis, that in the CEE countries regions with large metropolitan areas and some almost stagnating agricultural areas are forming different convergence clubs, nevertheless, there is also relatively high income mobility found in the case of the remaining regions. Interestingly, the authors find that  slow-growing regions tend to be often located at the eastern border of the EU or on some geographically disadvantaged locations. \citet{monastiriotis2011regional}, among others, mentions a strong tendency to growing income inequality in CEEC visible already shortly after the fall of the Iron Curtain. According to \citeauthor{monastiriotis2011regional} this increase is many times higher than in the case of the old EU members. Persistent regional disparities in CEE countries, even in composite measures of well-being (Human Development Index), are also confirmed by \citet{benedek2015paths}. They found no signs of $\sigma$-convergence between 1995 and 2000.  

From further research, \citet{ehrlich2020place} provide an extensive discussion of EU cohesion policy in light of metro areas performance, concluding that these place-based policies have a positive impact but especially for areas with higher education levels. \citet{rodriguez2020institutional} focus on investigating impact of government quality on the economic performance of European regions in dynamic panel regression framework between 2009 and 2013. They find that government quality is a key determinant of economic growth in the southern periphery and less important in the lagging regions of Eastern Europe. These findings help us to justify the choice of the explanatory variable in the regression below.

Our review suggests that most of the convergence literature works with samples ending in the 2000s and 2010s. As documented above, the convergence pattern changes over time. Therefore, there is room to reassess convergence and its development in light of  key economic events. This work will pay attention primarily to the aftermath of the Great Recession and the development preceding the COVID crisis.

\section{Analysis of Convergence}
\subsection{Methodology}

This work uses log $t$ test by \citet{phillips2007transition}  to identify the convergence pattern within the EU and to test the convergence in log GDP per capita across its regions. \citet{phillips2007transition} show that the assumption of a homogenous technological process  leads to inconsistency of classical $\beta$-convergence due to omitted variable bias and endogeneity. Consequently, they proposed an alternative testing procedure capable of capturing multiple types of convergence observed in reality, known as the log $t$ test. This convergence test is a variation on the neoclassical growth model that allows for transitional cross-sectional divergence  because the model parameters are allowed to vary across cross-sections.  Crucially, the model tests the overall relative convergence among the cross sections, with the alternative allowing both overall divergence and club convergence within regional subgroups. 

\citeauthor{phillips2009economic}'s framework also allows for tracing  trajectory of each cross-section $i$ relative to the club's average in time, called by \citet{phillips2009economic} the relative transition path (denoted as $h_{it}$ ). The transition path reflects any divergence of the individual unit $i$ from the common trend and allows to monitor the relative economic path of each unit including transitional or permanent divergence.

For identifying the convergence clubs, we used the clustering mechanism proposed by \citet{phillips2009economic} and \citet{bartkowska2012regional}. This procedure allows testing for convergence within subgroups that are formed through a multistep procedure. Importantly, it utilizes the log $t$ test's statistic that is being compared against a selected criterion level to determine if certain regions form a convergence club. Details of the test are and the clustering procedure are available in the Appendix.

As often noted (\citet{dall2008regional}, \citet{magrini2004regional}, \citet{anselin1991properties}, \citet{anselin2001spatial}), spatial autocorrelation is a frequent problem in the analysis of regional units, leading to biased $t$-tests and measures of fit. We indeed confirm the existence of autocorrelation among EU regions in log GPD per capita in our data. This could harm the results of \citeauthor{phillips2007transition}'s log $t$ test, as the key parameter $b$ is based on an OLS regression of this variable. Therefore, we will use a filtering approach by \cite{getis2002comparative} based on the number of geographic connections of each region within a certain distance, in order  to remove the spatial dependence from an autocorrelated variable and thus produce a new, spatially independent, variable. We apply filtering by \cite{getis2002comparative} to log GDP per capita and use the resulting variable in the log $t$ test below.



\subsection{Results: Convergence versus Convergence Clubs}
In this work, we analyzed 275 European NUTS 2 regions in the new and old member states of the EU. The log $t$ test and the associated clustering procedure were applied to the log GDP per capita (in PPS) between the years 2003 and 2019. To assess changes in the convergence patterns, we also estimate the convergence clubs in the sample until 2015. Data were obtained primarily from the "Regional statistics by NUTS classification" database on the Eurostat website.

%\begin{table}[!htbp] \centering 
% \caption{Log $t$ test result for all regions} 
%  \label{Table_overall} 
% \scalebox{0.85}{
%\begin{tabular}{@{\extracolsep{5pt}}cll} 
%\\[-1.8ex]\hline 
%\hline \\[-1.8ex] 
%\multicolumn{1}{c}{N} & \multicolumn{1}{c}{log(t)} & \multicolumn{1}{c}{t value}  \\ 
%\hline \\[-1.8ex] 
%277 &  -0.758  & -20.110 \\ 
%\hline \\[-1.8ex]
%\textit{Note:}  & \multicolumn{1}{r}{$^{**}$p$<$0.05} \\
%\end{tabular}
%}
%\end{table}

Applying the log $t$ test (Equation \ref{eq10} in the Appendix), the convergence among all regions of the EU is rejected with the log $t$ test statistic ($t_b$) equal to -20.110. Similar results are also obtained for the sample ending in 2015, with the test statistics equal to -20.576. Therefore, we can see that there is no overall convergence in the EU in the measured periods.

Then, we tested for the club convergence. Following the \citeauthor{phillips2007transition}'s clustering procedure, we initially obtained 7 convergence clubs and one diverging region, Inner London – West. As suggested by \citet{bartkowska2012regional}, we subsequently tried to merge the adjoining groups and tested whether they converge. This process resulted in five convergence clubs with the third and fourth clubs merged into a new club. The same procedure was applied for the 2015 sample, also leading to the same number of merged clubs and two diverging regions, Luxembourg and Inner London – West. Table \ref{Table_clubs1} shows the results of the procedure after merging for the year 2019 as well as the results of the \citeauthor{phillips2007transition} procedure until 2015.\footnote{Tables \ref{Table_clubs2} and \ref{Table_clubs2_2015} in the Appendix show the clubs before merging for 2019 and 2015 sample respectively.}  We can notice above all that the higher clubs have more members than in 2019 - top economic performance in Europe seems to be becoming more exclusive. This seems to be accompanied by a significant increase in the number of regions in the poorest Club 5. On the other hand, there is also an expansion of the intermediate Club 3.

\begin{table}[!htbp] \centering 
 \caption{\textbf{Convergence club classification after merging}} 
  \label{Table_clubs1}
\begin{center}
 \scalebox{0.85}{
\begin{tabular}{@{\extracolsep{5pt}}lclllll} 
\\[-1.8ex]\hline 
\hline \\[-1.8ex] 
Club & \multicolumn{1}{c}{N} & \multicolumn{1}{c}{log($t$)} & \multicolumn{1}{c}{$t$-stat.} & \multicolumn{1}{c}{Y}
& \multicolumn{1}{c}{$\Delta$Y} & \multicolumn{1}{c}{$\sigma$Y} \\ 
\hline \\[-1.8ex]
\multicolumn{7}{c}{2019 Convergence Clubs}\\
Club 1 & 10 &  0.019  & 0.113 &  11.000 & 0.584 & 0.156  \\ 
Club 2 & 24 & 0.108 & 0.710 & 10.700 & 0.461 & 0.133   \\ 
Club 3 & 111 & -0.123   & -1.400 & 10.400 &  0.418 & 0.181 \\ 
Club 4 & 52 & 0.163  & 1.530 & 10.100 & 0.407 & 0.148 \\ 
Club 5 & 79 & -0.096 & -0.968 & 9.880 & 0.340 & 0.229 \\
\hline
\multicolumn{7}{c}{2015 Convergence Clubs}\\
Club 1 & 17 &  0.266  & 2.693 & 10.900 & 0.444  & 0.190 \\ 
Club 2 & 36 & -0.015 & -0.088 & 10.600 & 0.350 & 0.169 \\ 
Club 3 & 93 &  -0.111  & -1.032 & 10.200 & 0.307 & 0.224\\ 
Club 4 & 98 & -0.063  & -0.593 & 10.000 & 0.289 & 0.197\\ 
Club 5 & 29 & -0.099 & -0.638 & 9.770 & 0.236 & 0.263\\ 
\hline
\end{tabular}
}
\caption*{\scriptsize Note: the table contains from left to right: the number of regions included in each Club, coefficient $b$ of the convergence test (eq. \ref{eq10}) and its $t$-statistics, and statistics relating to the log GDP per capital - its value in  2019 or 2015, change from 2008 and standard deviation in 2015/19.}
\end{center}
\end{table}


The value of coefficient $b$ from equation \ref{eq10} plays a crucial role in the log $t$ test. It is interpreted as the speed of convergence and also shows the sign and magnitude of the $t$-statistic. By analyzing the value of $b$ for both versions of the test for 2019 (5 and 7 clubs), we found that the convergence clubs are rather weak. We can see mostly negative, although insignificant, values of the $b$ coefficients suggesting relative convergence at a very slow rate, as the estimate of the parameter $\alpha$ is not significantly different from 0 \citep{phillips2007transition}. The clubs before merging show a higher value of $t$-statistics with all $b$ coefficients positive. However, the quantitative conclusions are the same for both versions.

We also plotted the convergence behaviour using the relative transition coefficient $h_{it}$ from Equation \ref{eq8}, which traces out individual transition path of each cross-sectional unit with respect to a group average, for graphical investigation of relative convergence over time. Figures \ref{paths1} and \ref{paths5} show the relative transition paths for the first and the last convergence clubs in 2019.\footnote{Analogical Figures for the rest of the Clubs can be found in the Figures \ref{paths2} - \ref{paths4} in the Appendix.} These pictures confirm that the convergence paths can be various, in spite of the ultimate convergence, with some regions rising from very low relative level, other descending from relatively higher level of GDP per capita to the group average. We can also note alternating states of convergence and divergence for some regions. Figure \ref{paths_overall} then documents the overall non-convergence in the EU as a whole using average transition paths for all five clubs.

\begin{figure}[!htbp]%
    \centering
\begin{subfigure}{0.45\textwidth}
    \centering
    \includegraphics[width=0.95\linewidth]{RTC_mer_club1.png}
    \caption{First Club}
    \label{paths1}   
\end{subfigure}\hfill
\begin{subfigure}{0.45\textwidth}
    \centering
    \includegraphics[width=0.95\linewidth]{RTC_mer_club5.png}
    \caption{Fifth Club}
    \label{paths5}
\end{subfigure}
\caption{\textbf{Transition paths for members of the First and Last Convergence Clubs (2019)}}
\label{convergence_paths}

\end{figure}


Figure \ref{clubs_graphic} presents the geographic composition of the convergence clubs in Europe. Focusing on results till 2019 first (Figure \ref{clubs_graphic}\subref{clubs_graphic_2019}), we can see that the capital regions tend to be part of the highest clubs. This also includes the eastern periphery, and the only exception seems to be the south with the major cities belonging to Club 3.\footnote{Complete list of the convergence clubs is available in the Appendix.} The capitals are also very often surrounded by regions belonging to, sometimes much, lower clubs. The most prominent example of this is the Paris region. Such a tendency appears in many old and new member states. There is also evident the previously suggested tendency of regions from the same club to cluster together, as demonstrated by the presence of continuous bright colored areas versus darker blue areas in the Figure \ref{clubs_graphic}.\footnote{Note, that there is also an outlying region (Inner London - West) included, forming the "Club 0"} The clusters are not nationally bounded but seem to have an interstate nature (see the cluster of Club 3 regions spanning from the Czech republic to southeastern France or the group of low-performing areas on the eastern border).


The dominant position of the capital cities confirms previous findings about convergence clubs made by \citet{sme2012regional} and \citet{bartkowska2012regional}. However, our results do not support the country effect in the form of regions belonging to the same country tending to cluster together, as \citet{bartkowska2012regional} found for the old EU members in 2005. Instead, we frequently see the entire range of clubs within a single country. We can see this phenomenon in eastern countries such as Poland or Romania. However,
even Western Europe is not spared from this phenomenon (France or the UK can serve as an example).


\begin{figure}[!htbp]%
\centering
\begin{subfigure}[c]{0.75\linewidth}
  {
  \includegraphics[width=\linewidth]{convergence_clubs2019.png}
    \caption{2019}
    \label{clubs_graphic_2019}
  }
\end{subfigure}
\qquad
\begin{subfigure}[c]{0.75\linewidth}
  \includegraphics[width=\linewidth]{map_4062.jpeg}
  \caption{2015}
  \label{clubs_graphic_2015}
\end{subfigure}
  \caption{\textbf{Comparison of convergence clubs and diverging regions in 2015 and 2019}}
\label{clubs_graphic}
\end{figure} 


Comparing the results for 2015 and 2019, we can notice a stronger concentration of high-convergence clubs around southern Germany combined with a loss of relative economic dynamic in Northern Italy and, to a certain extent, in the Benelux. This seems to be in line with the "Manufacturing core" hypothesis \citep{cutrini2019economic, stollinger2016structural}, pointing out the importance of manufacturing-oriented areas in Central Europe. On the other hand, the so-called "Blue banana," a large urbanization corridor in Western Europe \citep{hospers2002beyond}, visible until 2015, disappears when the time span is prolonged to 2019. The top club membership can thus be characterized by southern Germany and capitals/major urban areas.

It is worth noting that the difference between the old and the new members in their club membership does not seem very strict, with large parts of rural France and Poland belonging to the same club. Moreover, the top convergence club is becoming more exclusive, with the number of regions falling from 15 to 10 - this points to uneven economic development in Europe before the COVID pandemic.

At the same time, there are many CEE capitals in the two highest clubs, which leads us to conclusion that we do not see much evidence for the new-old division, as suggested in the previous research \citep{eckey2007convergence}.
On the other hand, there is a visible concentration of Club 4 and 5 regions along the eastern border of the EU. Moreover, compared to the 2015 results, where the concentration of the lowest clubs is clearly the highest in the southern part of the EU (Figure \ref{clubs_graphic}\subref{clubs_graphic_2015}), their relative performance appears to be deteriorating. We, therefore, cannot confirm any east-west (old-new) general convergence and rather side with a broadly defined centre-periphery division, with the "centre" being defined on the one hand by the financial and administrative urban centres and on the other by the EU's manufacturing core concentrated around Southern Germany.



\section{Determinants of Club Membership}
\label{Determinants of club membership}


 
After identifying the convergence clubs, we analyzed the factors that influence the economic performance of European regions using the ordered logistic regression framework.

\subsection{Explanatory variables}

% Table created by stargazer v.5.2 by Marek Hlavac, Harvard University. E-mail: hlavac at fas.harvard.edu
% Date and time: so, lis 04, 2017 - 22:15:15
\begin{table}[!htbp] \centering 
\resizebox{0.8\textwidth}{!}{\begin{minipage}{\textwidth}
  \caption{\textbf{Explanatory variables - description}} 
  \label{var_legend}
  \renewcommand{\arraystretch}{1.5}
\begin{tabular}{@{\extracolsep{5pt}} p{4cm}p{9.5cm}} 
\\[-1.8ex]\hline 
\hline \\[-1.8ex] 
  Regression variable  & Description \\ 
\hline \\[-1.8ex] 
Initial GDP level &  Level of GDP per capita (PPS) in 2008 (log) \\
Investment & Gross fixed capital formation in millions of Euros, across all NACE activities (log) \\
Tertiary edu. share & Average percentage of population with tertiary education. \\ 
Share of scientists  & Share of scientists in active population  \\ 
Spec. agriculture & Share of population from 15 to 64 years employed in agriculture (NACE Rev.2 category A) in all employed (in 2008).  \\ 
Spec. Industry & Share of population from 15 to 64 years employed in Industry (NACE Rev.2 categories B-E) in all employed (in 2008)  \\
Spec. Financial Services & Share of population from 15 to 64 years employed in Financial and insurance activities (NACE Rev.2 category K) in all employed (in 2008)\\
Spec. Trade, transport, accommodation \& food services & Share of population from 15 to 64 years employed in NACE Rev.2 categories G-I (2008). \\
Spec. Other Services & Share of population from 15 to 64 years employed in Arts, entertainment and recreation (NACE Rev.2 categories R-U, 2008). \\
Spec. Business Services & Specialization in Business Services (NACE Rev.2 subcategories J58, J62, J63, M69, M70, M71, M73, N78) in 2008.  \\ 
Metropolitan regions & Dummy indicating presence of  a large urban area  \\
Patent activity & Average number of patents per million inhabitants \\
Regional Innovation Index & European and Regional Innovation Scoreboards an extension of the European innovation scoreboard (EIS) published by the European Commission between years 2014 and 2019
\\
Governance Quality & Measured by European Quality of Government Index (EQI) in 2010  \citep{charron2014regional}.\\
Highest Neighbour Income & Log GDP per capita of the richest neighbouring region.  \\
Border Region &  Dummy indicating region lying on EU's internal or external border.\\
\hline 
\hline \\[-1.8ex]
\textit{Note:} & Unless stated otherwise, the variables are used as initial conditions in 2008.   \\
\end{tabular}
\end{minipage}}
\end{table} 

We used several explanatory variables for our analysis. We again worked with Eurostat data ("Regional statistics by NUTS classification"). Since we focused on post-crisis development and wanted to investigate the role of sectoral composition, we used the data from 2008 in the logistic regression instead of the 2003 - 2019 period used for the convergence analysis. This is due to the limited availability of the older data for the new member states and changes in the Eurostat classification of economic activity that affect the sectoral variables. But it also allows us to concentrate on the EU's development between two distinct crises (the Great Recession of 2008 and 2009 and the COVID crisis). The choice of these variables follows New Economic Geography (NEG) and New Growth Theory (NGT), as well as previous research papers such as \citet{mora2008factors} or \citet{bartkowska2012regional}. Table \ref{var_legend} presents an overview of the variables with their names as they appear in the regression results below.

In line with the literature, we included variables measuring the level of initial economic specialization in most of the NACE Rev.2 classification sectors.\footnote{Some of the sectors were omitted due to large number of missing observations, such as the Real Estate activities.} We defined the specialization as the share of the population employed in a given sector over the overall employment in the year 2008. We worked with six sectors spanning from Industry to Financial Services, as defined by NACE classification of economic activity, and the Business services. Business services are a dynamically developing subsector of the economy, shown by the previous research as potentially highly innovative and productivity promoting \citep{corrocher2014kibs}. Following \citet{guastella2015knowledge}, we defined the specialization in Business Services as the share of workers employed in Business services in total employed population. This work uses the definition of the business services given by Regulation of the European Parliament No. 298/2008. This regulation defines business services as consisting of five divisions and three groups of NACE Rev. 2 classification (divisions 62, 69, 71, 73 and 78 and groups 58.2, 63.1 and 70.2). The data are not available in such detail in the Eurostat database for the Business services, therefore whole divisions (58, 63 and 70) are used as reasonable approximations for the respective groups.

We also added the log of initial (as of 2008) GDP per capital and investment as measures of initial economic conditions. We find these variables in line with the neoclassical growth theory \cite{iammarino2017regional}.% maybe see Cutrini or Iammarino's Solow comment

Apart from the variable measuring sectoral specialization mentioned above, we also included variables measuring human capital developments in the regions. Furthermore, we experimented with several variables measuring the regional innovation level. We used the percentage of the population with tertiary education and the share of scientists in the active population as measures of regional population skill sophistication. We also included the average number of patent applications to the European Patent Organization per million inhabitants. 

Next, we include the Regional Innovation Index issued by the European Commission and the European Quality of Government Index (EQI) with the assumption that better-governed regions should be attractive for both business and workers and, therefore, be able to perform better in all aspects of economic life.

Inspired by New Economic Geography's emphasis on agglomeration economies as key drivers of growth but also drivers of territorial divergence. The two main equalizing 
counterfactors should then be knowledge spillovers and within-country migration  \citep{iammarino2017regional}. Therefore, we included dummies measuring the level of regional urbanization and indicating capital cities. %As an alternative, we utilize Eustostat's distinction between urban, rural and intermediate areas.
Last but not least, we use income level of the richest neighbour of a given region to check the impact of a region's surrounding.

The next section shows the impact of the explanatory variables discussed above on the probability of membership in a given convergence club, tested via logistic regression. More specifically, we were interested in the impact of the initial conditions on future economic performance. Therefore, the 2008 values of the listed variables were used.\footnote{Models presented below use Factorial Analysis for Mixed Data (FAMD) as an imputation method to deal with the missing observations.}

\subsection{Regression Results}

The results of the ordered logistic regression are presented in the tables below. We focus on marginal effects that are calculated at the means of the independent variables and show the impact of a unit change in an explanatory variable on the probability that a certain region will enter a particular club \citep{carrolloglmx}.\footnote{Regression coefficients are reported in the Appendix.}

% Table created by stargazer v.5.2.2 by Marek Hlavac, Harvard University. E-mail: hlavac at fas.harvard.edu
% Date and time: čt, srp 24, 2023 - 20:47:57
\begin{table}[!htbp] \centering 
\resizebox{0.8\textwidth}{!}{\begin{minipage}{\textwidth}
  \caption{\textbf{Marginal effects of Logistic regression with Sectoral and Human capital variables (2008 levels). Logistic regression with merged Clubs, FAMD imputation and robust standard errors.}}
  \label{famd_filtered_marginal_spec1}
\begin{tabular}{@{\extracolsep{5pt}} lccccc} 
\\[-1.8ex]\hline 
\hline \\[-1.8ex] 
 & Club 1 & Club 2 & Club 3 & Club 4 & Club 5 \\ 
\hline \\[-1.8ex] 
Initial GDP level & $0.011$$^{*}$ & $0.108$$^{***}$ & $1.712$$^{***}$ & $ $-$0.980$$^{***}$ & $ $-$0.851$$^{***}$ \\ 
& (0.006) &  (0.032) & (0.253) & (0.183) & (0.145)\\
&\\
Investment  & $0.000$ & $0.000$ & $ $-$0.001$ & $0.000$ & $0.000$ \\
& (0.000) &  (0.003) & (0.052) & (0.030) & (0.026)\\
&\\
Spec. Agriculture & $0.021$ & $0.202$$^{**}$ & $3.211$$^{***}$ & $ $-$1.838$$^{***}$ & $ $-$1.596$$^{***}$ \\ 
& (0.013) &  (0.090) & (1.061) & (0.681) & (0.508)\\
&\\
Spec. Finances & $0.053$ & $0.519$$^{**}$ & $8.229$$^{**}$ & $ $-$4.712$$^{***}$ & $ $-$4.090$$^{***}$ \\
& (0.035) &  (0.241) & (2.746) & (1.636) & (1.462)\\
&\\
Spec. Industry & $0.023$$^{*}$ & $0.230$$^{***}$ & $3.648$$^{***}$ & $ $-$2.089$$^{***}$ & $ $-$1.813$$^{***}$ \\
& (0.014) &  (0.078) & (0.637) & (0.454) & (0.341)\\
&\\
Spec. Trade, Accommodation & $0.005$ & $0.053$ & $0.836$ & $ $-$0.479$ & $ $-$0.416$ \\ 
& (0.008) &  (0.080) & (1.193) & (0.692) & (0.590)\\
&\\
Spec. Other Services & $ $-$0.025$ & $ $-$0.248$ & $ $-$3.939$ & $2.255$ & $1.958$ \\ 
& (0.027) &  (0.216) & (3.119) & (1.804) & (1.572)\\
&\\
Business Services & $0.010$ & $0.095$$^{*}$  & $1.509$$^{*}$  & $ $-$0.864$$^{*}$  & $ $-$0.750$$^{*}$  \\
& (0.006) &  (0.057) & (0.810) & (0.479) & (0.401)\\
&\\
Share of Scientists & $0.225$$^{*}$  & $2.203$$^{***}$  & $34.931$$^{***}$  & $ $-$19.999$$^{***}$  & $ $-$17.359$$^{***}$  \\ 
& (0.132) &  (0.841) & (8.495) & (5.594) & (4.306)\\
&\\
Tertiary Edu. Share & $ $-$0.005$ & $ $-$0.051$$^{*}$  & $ $-$0.801$$^{**}$  & $0.458$$^{**}$  & $0.398$$^{**}$  \\
& (0.004) &  (0.026) & (0.390) & (0.228) & (0.196)\\
\hline \hline \\[-1.8ex]
McFadden's $R^{2}$: 0.366 \\
Observations: 267\\
No. of Parameters: 10\\
\hline
\end{tabular}
\begin{tablenotes}
\small 
\item Note: $^{*}$p$<$0.1; $^{**}$p$<$0.05; $^{***}$p$<$0.01
\item Club 1 represents the regions with the highest income, Club 5 with the lowest.
\end{tablenotes}
\end{minipage}}
\end{table} 



We use three alternative specifications. First, we tested only sectoral variables together with variables linked to human capital in Table \ref{famd_filtered_marginal_spec1}. In the second specification, in Table \ref{famd_filtered_marginal_spec2},  we investigate the geographic factors linked to economic performance, such as dummies for capital cities and the impact of neighbouring regions. The results in Table \ref{famd_selected_marginal_neigh_robust_filtered} then bring all the variables together. 

Across the specifications, our results suggest that the magnitude of the marginal effects peaks around Club 3 and weakens for the very top Club. There is a minimal impact on the top Club membership of any of our variables, this holds both in terms of significance and magnitude. 
Analysing the impact of individual variables, we can first notice that the initial income per capita does have a significant effect on club membership, with a higher initial GDP per capita in 2008 increasing the probability of being in the highest convergence clubs in 2019. This suggests a certain level of inertia in economic development.
 
Table \ref{famd_filtered_marginal_spec1} documents the effect of the structural specialization variables as suggested by \citet{cutrini2019economic}. Especially strong seems to be the impact of industrial sector\footnote{Composed primarily of Manufacturing but also encompassing NACE Sections B, E, F (Mining, Electricity, gas, steam and air conditioning supply and Water supply) besides Manufacturing (NACE Section C) itself, which however plays a dominant role in the category.} together with Finance and Insurance Services. Concentration of Financial Services. The two variables turn out to be the most important predictors of Club 1 membership in Table \ref{famd_filtered_marginal_spec1}. This finding seems to be in line with two specific trends visible in present-day Europe - the first is the importance and emergence of the "Manufacturing Core" in its central part and the second is the key role of the urban financial centres. Concentration in financial services appears, however, to overlap with the incidence of capital cities, putting their importance into question and speaking in favor of agglomeration effects as suggested by New Economic Geography.
We can also note the negative impact of the "Other Services" category, which includes entertainment and recreation, on the probability of a higher club membership. This could be a result of the strong tourism dependence of the southern EU periphery. However, the marginal effects are never significant for this variable. On the contrary, specialization in business services has a positive impact on the probability of entering the highest clubs, yet it is again only marginally significant.


% Table created by stargazer v.5.2.2 by Marek Hlavac, Harvard University. E-mail: hlavac at fas.harvard.edu
% Date and time: pá, srp 25, 2023 - 08:30:02
\begin{table}[!htbp] \centering 
\resizebox{0.8\textwidth}{!}{\begin{minipage}{\textwidth}
  \caption{\textbf{Marginal effects of Logistic regression with Geographic determinants (2008 levels). Logistic regression with merged Clubs, FAMD imputation and robust standard errors.}} 
  \label{famd_filtered_marginal_spec2} 
\begin{tabular}{@{\extracolsep{5pt}} lccccc} 
\\[-1.8ex]\hline 
\hline \\[-1.8ex] 
 & Club 1 & Club 2 & Club 3 & Club 4 & Club 5 \\ 
\hline \\[-1.8ex] 
Initial GDP level & $0.023$$^{**}$ & $0.173$$^{***}$ & $1.684$$^{***}$ & $ $-$0.870$$^{***}$ & $ $-$1.010$$^{***}$ \\ 
& (0.010) &  (0.045) & (0.260) & (0.173) & (0.166)\\
&\\
Investment & $ $-$0.001$ & $ $-$0.004$ & $ $-$0.040$ & $0.020$ & $0.024$ \\
& (0.001) &  (0.005) & (0.046) & (0.024) & (0.027)\\
&\\
Capital & $0.004$$^{*}$ & $0.031$$^{**}$ & $0.186$$^{***}$ & $ $-$0.119$$^{***}$ & $ $-$0.103$$^{***}$ \\
& (0.002) &  (0.013) & (0.054) & (0.040) & (0.029)\\
&\\
Metro area & $0.003$$^{*}$ & $0.019$$^{**}$ & $0.240$$^{***}$ & $ $-$0.089$$^{***}$ & $ $-$0.173$$^{***}$ \\
& (0.001) &  (0.008) & (0.076) & (0.027) & (0.064)\\
&\\
Governance Quality & $ $-$0.001$ & $ $-$0.006$ & $ $-$0.055$ & $0.028$ & $0.033$ \\ 
& (0.001) &  (0.005) & (0.039) & (0.021) & (0.023)\\
&\\
Highest Neighbour Income & $ $-$0.004$$^{*}$ & $ $-$0.028$$^{**}$ & $ $-$0.273$$^{***}$ & $0.141$$^{***}$ & $0.164$$^{***}$ \\ 
& (0.002) &  (0.011) & (0.081) & (0.047) & (0.050)\\
&\\
Border Region & $0.004$$^{*}$ & $0.029$$^{***}$ & $0.269$$^{***}$ & $ $-$0.129$$^{***}$ & $ $-$0.173$$^{***}$ \\ 
& (0.002) &  (0.009) & (0.062) & (0.034) & (0.041)\\
\hline \hline \\[-1.8ex]
McFadden's $R^{2}$: 0.313 \\
Observations:  267\\
No. of Parameters:  7\\
\hline 
\end{tabular}
\begin{tablenotes}
\small 
\item Note: $^{*}$p$<$0.1; $^{**}$p$<$0.05; $^{***}$p$<$0.01
\end{tablenotes}
\end{minipage}}
\end{table} 

We can also see a strong effect of the variables connected with research activity in Table \ref{famd_filtered_marginal_spec1}, such as the share of scientists and patent activity.\footnote{As the research activity variables, including the Regional Innovation Index, are highly correlated, Table \ref{famd_filtered_marginal_spec1} shows only the share of the scientists, which turned out to be the most impactful variable, other results are available upon request.} The results suggest that the initial share of scientists has a large impact on the probability of entering the First Club. In the case of the share of  the tertiary educated in the population, we see that the initial levels of this variable turn out to be also significant; the marginal effects, however, give a somewhat counterintuitive result with a positive association with the lower clubs. One of the possible interpretations of these results could be that increasing the formal level of education in a given region is on its own insufficient for economic prosperity and needs to be accompanied by functioning innovation ecosystems capable of employing the increasing number of tertiary education graduates in adequate jobs.

Turning to our second specification in Table \ref{famd_filtered_marginal_spec2}, we see a positive impact of metropolitan areas and especially capital cities that seem to be a predictor of high club membership. In fact, the highest club is primarily composed of capital cities (6 of the 10 regions in Club 1 contain a capital city).  We can see a very similar impact of the metropolitan areas dummy on the club membership as we saw for the capital cities.

Table \ref{famd_filtered_marginal_spec2} further shows that the presence of a high-income neighbour decreases the economic prospects of its neighbours. These results suggest that high-income localities can bring a negative drag on their surroundings, most likely in the form of a drain on the human capital of poor neighbours. We can also see that the dummy denoting the border regions is significant, with a positive effect, which is not surprising, as many of the capital cities and prosperous regions lie near the internal border of the EU. We also tested the impact of the Quality of Governance index, yet it turned out to be insignificant.

% Table created by stargazer v.5.2.2 by Marek Hlavac, Harvard University. E-mail: hlavac at fas.harvard.edu
% Date and time: so, srp 05, 2023 - 22:21:23
\begin{table}[!htbp] \centering
\resizebox{0.75\textwidth}{!}{\begin{minipage}{\textwidth}
  \caption{\textbf{Marginal effects: Logistic regression with merged Clubs, FAMD imputation and robust standard errors (variables in 2008 levels). 9 regions with no neighbouring regions have been removed.}} 
  \label{famd_selected_marginal_neigh_robust_filtered} 
\begin{tabular}{@{\extracolsep{5pt}} lccccc} 
\\[-1.8ex]\hline 
\hline \\[-1.8ex] 
 & Club 1 & Club 2 & Club 3 & Club 4 & Club 5 \\ 
\hline \\[-1.8ex]
%log\_init\_gdp & $0.005$ & $0.076$ & $2.182$ & $$-$1.453$ & $$-$0.810$ \\
Initial GDP level & $0.005$ & $0.076$$^{***}$ & $2.182$$^{***}$ & $ $-$1.453$$^{***}$ & $ $-$0.810$$^{***}$ \\ 
%init\_gfcf & $0$ & $$-$0.001$ & $$-$0.022$ & $0.014$ & $0.008$ \\ 
& (0.004) &  (0.027) & (0.292) & (0.263) & (0.141)\\
&\\
Investment & $0.000$ & $ $-$0.001$ & $ $-$0.022$  & $0.014$ &  $0.008$ \\
& (0.000) &  (0.002) & (0.057) & (0.038) & (0.021)\\
&\\
\multicolumn{6}{c}{\textit{Sectoral Variables}}\\

%spec\_A & $0.010$ & $0.136$ & $3.921$ & $$-$2.611$ & $$-$1.456$ \\ 
Spec. Agriculture & $0.010$ & $0.136$$^{**}$ & $3.921$$^{***}$ & $ $-$2.611$$^{***}$ & $ $-$1.456$$^{***}$ \\ 
& (0.006) &  (0.058) & (1.110) & (0.806) & (0.439)\\
&\\
% spec\_K & $0.015$ & $0.213$ & $6.157$ & $$-$4.100$ & $$-$2.286$ \\  
Spec. Finances & $0.015$ & $0.213$ & $6.157$$^{**}$ & $ $-$4.100$$^{*}$ & $ $-$2.286$$^{*}$ \\ 
& (0.012) &  (0.132) & (3.105) & (2.079) & (1.230)\\
&\\
%spec\_BE & $0.011$ & $0.156$ & $4.512$ & $$-$3.004$ & $$-$1.675$ \\ 
Spec. Industry & $0.011$ & $0.156$$^{***}$ & $4.512$$^{***}$ & $ $-$3.004$$^{***}$ & $ $-$1.675$$^{***}$ \\ 
& (0.007) &  (0.059) & (0.799) & (0.642) & (0.355)\\
&\\
% spec\_GI & $0.002$ & $0.030$ & $0.865$ & $$-$0.576$ & $$-$0.321$ \\ 
Spec. Trade, Accommodation & $0.002$ & $0.005$ & $0.143$ & $ $-$0.099$ & $ $-$0.049$ \\
& (0.003) &  (0.041) & (1.127) & (0.753) & (0.421)\\
&\\
%spec\_RU & $$-$0.007$ & $$-$0.095$ & $$-$2.740$ & $1.825$ & $1.017$ \\ 
Spec. Other Services & $ $-$0.007$ & $ $-$0.095$ & $ $-$2.740$ & $1.825$ & $1.017$ \\ 
& (0.010) &  (0.117) & (3.083) & (2.066) & (1.156)\\
&\\
%BS & $0.005$ & $0.066$ & $1.913$ & $$-$1.274$ & $$-$0.710$ \\ 
Business Services & $0.005$ & $0.066$$^{*}$ & $1.913$$^{**}$ & $ $-$1.274$$^{**}$ & $ $-$0.710$$^{**}$ \\ 
& (0.003) &  (0.038) & (0.924) & (0.639) & (0.345)\\


\\
\multicolumn{6}{c}{\textit{Human Capital Variables}}\\
%init\_scientists\_share & $0.054$ & $0.745$ & $21.529$ & $$-$14.336$ & $$-$7.992$ \\ 
Share of Scientists & $0.054$ & $0.745$$^{*}$ & $21.529$$^{**}$ & $ $-$14.336$$^{**}$ & $ $-$7.992$$^{**}$ \\
& (0.039) &  (0.404) & (9.691) & (6.622) & (3.751)\\
&\\
%init\_ter\_edu & $$-$0.002$ & $$-$0.021$ & $$-$0.614$ & $0.409$ & $0.228$ \\ 
% init\_eqi & $0$ & $0.002$ & $0.046$ & $$-$0.031$ & $$-$0.017$ \\  
Governance Quality & $0.000$ & $0.002$ & $0.046$ & $ $-$0.031$ & $ $-$0.017$ \\
& (0.000) &  (0.002) & (0.063) & (0.042) & (0.024)\\
&\\
Tertiary Edu. Share  & $ $-$0.002$ & $ $-$0.021$ & $ $-$0.614$ & $0.409$ & $0.228$ \\
& (0.001) &  (0.015) & (0.397) & (0.267) & (0.151)\\
\\
\multicolumn{6}{c}{\textit{Geographic Determinants}}\\
%capital & $0.002$ & $0.027$ & $0.294$ & $$-$0.227$ & $$-$0.096$ \\ 
Capital & $0.002$ & $0.027$$^{**}$ & $0.294$$^{***}$ & $ $-$0.227$$^{***}$ & $ $-$0.096$$^{***}$ \\
& (0.001) &  (0.013) & (0.054) & (0.051) & (0.021)\\
&\\
% metro & $0$ & $0.004$ & $0.149$ & $$-$0.091$ & $$-$0.062$ \\ 
Metro area & $0.000$ & $0.004$ & $0.149$ & $ $-$0.091$ & $ $-$0.062$ \\ 
& (0.000) &  (0.003) & (0.106) & (0.061) & (0.050)\\
&\\
% neigh\_highest\_income & $$-$0.001$ & $$-$0.010$ & $$-$0.287$ & $0.191$ & $0.107$ \\ 
Highest Neighbour Income & $ $-$0.001$ & $ $-$0.010$$^{**}$ & $ $-$0.287$$^{**}$ & $0.191$$^{**}$ & $0.107$$^{**}$ \\
& (0.001) &  (0.005) & (0.116) & (0.081) & (0.044)\\
&\\
% border\_region & $0.001$ & $0.009$ & $0.252$ & $$-$0.162$ & $$-$0.100$ \\
Border Region & $0.001$ & $0.009$$^{**}$ & $0.252$$^{***}$ & $ $-$0.162$$^{***}$ & $ $-$0.100$$^{***}$ \\ 
& (0.000) &  (0.004) & (0.074) & (0.051) & (0.031)\\

\hline \hline \\[-1.8ex]
McFadden's $R^{2}$: 0.42  \\
Observations: 267\\
No. of Parameters 17\\
\hline
\end{tabular}
\begin{tablenotes}
\small 
\item Note: $^{*}$p$<$0.1; $^{**}$p$<$0.05; $^{***}$p$<$0.01
\end{tablenotes}
\end{minipage}}
\end{table} 


Using all the explanatory variables together in a single regression, as shown in Table \ref{famd_selected_marginal_neigh_robust_filtered}, we can see that any significant effect on the very first club is lost. However, the key conclusion from Tables \ref{famd_filtered_marginal_spec1}  and \ref{famd_filtered_marginal_spec2} seems to remain - we see a strong positive impact of industry and human capital, represented by the share of scientist variable. Compared to the partial results represented in the Tables above,  financial services lose their impact on the higher club membership when we add control for capital cities. As visible in Table \ref{famd_selected_marginal_neigh_robust_filtered}, the variable loses significance beyond Club 3. Also, the marginal effect of the dummy variable representing metropolitan cities becomes insignificant despite the continuing impact of the capitals - this could suggest an increasing level of centralization of economic activity within the national borders. The key factor for membership in the first two clubs thus becomes the specialization in Manufacturing, which is positive and significant across the specifications.

Our work can be compared with previous ones such as \citet{cutrini2019economic}, \citet{von2017regional}, and \citet{bartkowska2012regional}, and contributes to a discussion between initial GDP levels and sectoral specialization as primary determinants of economic development. We confirm the importance of geographic factors, noted in \cite{von2017regional}, and the initial capita GDP level as stressed by \cite{bartkowska2012regional}. We also share the emphasis on manufacturing as an important determinant of economic growth with \citet{cutrini2019economic}, who concentrates on the initial specialization levels, but we also find the variable linked to the New Economic Geography, such as agglomeration effects or innovation activity to play a role.

In summary, the results suggest that regional economic well-being is determined by a combination of initial sectoral specialization and its ability to create functional local innovation environments. A specific category then seems to be the large metropolitan areas, and the capitals in particular, with their strong position in finance. Our results show the importance of industrial concentration; we can also note that this variable seems to be more important than the initial level of economic development. Similarly, variables related to the development of the knowledge-based economy lead to a higher club membership.

\subsection{Robustness checks}
Apart from the variables included above, we also discuss two alternative specifications as a robustness check. The tables above incorporate a variable measuring the income of their richest neighbour; this results in a small number of regions (9) being filtered out as they have no neighbours.\footnote{These regions are Cyprus, Madeira, the Balearic Islands, Ceuta and Melilla, Canary Islands, Guadeloupe, Martinique and Malta} Table \ref{famd_selected_marginal} in the Appendix shows the alternative version of the regression exercise with all the regions but leaving out the richest neighbour variable. We can see that our results regarding sectoral specialization are kept. We can notice that specialization in manufacturing and the human capital variables are closer to significantly impacting the top club.

As a second robustness check, we include regression using the 2003 data in Table \ref{famd_selected_marginal_ecosys_robust_filtered}.  2008 was chosen as the starting year of our analysis in the previous section, as it is the year when the change in Eurostat's classification of economic activity from NACE Rev. 1.1 to NACE Rev. 2 was applied. Despite the imperfect correspondence in sectoral categories (e.g., the business services had to be dropped), the results in Table \ref{famd_selected_marginal_ecosys_robust_filtered} confirm the main conclusions reached so far. The only significant difference seems to be the less significant impact of capital and metro dummies on the top clubs.

\section{Conclusion}

This work investigated the regional convergence in the European Union between the Great Recession and the Covid crisis of 2019. We applied a convergence test developed by \citet{phillips2009economic}, allowing us to test for convergence among EU regions and distinguish between absolute and club convergence.

The \citeauthor{phillips2009economic}'s test rejected the hypothesis of an overall convergence among the EU regions. Therefore, we continued our analysis, searching for potential convergence clubs. Our final results gave us five distinct clubs 
that converge at a slow rate. Examining the composition of these clubs, we found significant disparities in club membership in many countries, which implies a high within-country inequality. The highest club consists predominantly of the capital cities, both from the old and the new countries of the EU, reflecting the capital vs. periphery contrast. Furthermore, we found a high concentration of regions belonging to the two highest clubs in southern Germany. This implies a shrinkage of the area of top economic performance compared to 2015 when the top clubs were concentrated in a continuous area from northern Italy to the Benelux. The finding thus seems to be more in line with the "Manufacturing core" hypothesis, combined with the continued dominance of major urban areas. We can also see a significant concentration of the regions from the lowest clubs in the southern periphery of the EU and also at its eastern border. Therefore, we consider the Centre-Periphery division the most prominent phenomenon in the EU.

The results of this analysis were then used as a dependent variable in logistic regression, employed to investigate key determinants of regional economic performance. Our choice of variables was inspired by the New Economic Geography and New Growth Theory.


Interestingly, we did not find a substantial difference between the new and old members in terms of their convergence pattern.


The regression results show that a combination of financial services, manufacturing, research, and innovation drives upward economic mobility among the EU regions. However, none of these variables is really significant for the highest club that is dominated by a group of capital cities and major agglomerations.
 
 
Bringing our results together, we see further confirmation of the center-periphery division in the EU, mainly in its "capitals versus the rest" form. We also see the top convergence clubs becoming more exclusive between 2015 and 2019 - shrinking to an area around Southern Germany, suggesting the validity of the Manufacturing Core hypothesis instead of the "Blue Banana" hypothesis. These changes also give more weight to the impact of sectoral specialization. Despite changes in club composition, we consistently confirm the importance of regional research as a success factor. We also confirm the importance of sectoral specialization in economic success. 


We see the key policy implication of this work in comparing the club convergence pattern between 2015 and 2019. This comparison points to the growing exclusivity of the top two convergence clubs and also documents their significant geographic transformation from the "Blue Banana" to the "Manufacturing Core," which is sectorally much narrower, within a very short period of only four years. Moreover, this change happened during a time of relative economic expansion in Europe, showing that the benefits of economic growth are not shared equally between the EU regions. Connected to this topic is the importance of manufacturing for economic growth in pre-COVID Europe, as suggested by the results. However, this manufacturing-based growth that was in place in the critical part of the EU between 2015 and 2019 seems to be in question in a time of energy crisis that undermines German competitiveness and threatens widespread deindustrialization of the "Core." This highlights the need to find a new growth model for Europe.



\newpage
\bibliographystyle{apalike}
\bibliography{References}
\appendix
\section{Appendix}

\subsection{Log $t$ test}

Log $t$ test departs from a variation of the neoclassical growth model where the transitional cross-sectional divergence is possible because the parameters $\beta_{it}$ and $x_{it}$ are allowed to vary across cross-sections:

The model departs from a variation of the neoclassical growth model where the transitional cross-sectional divergence is possible because the parameters $\beta_{it}$ and $x_{it}$ are allowed to vary across cross-sections:
\begin{equation} \label{eq3}
 y_{it} = y_i^* + a_{i0} + (y_{i0} - y_i^*)e^{-\beta_{it}t} + x_{it}t 
\end{equation}

 Variables $y_{i0}$ and $y_{i}^{*}$ in \eqref{eq3} are initial and steady state levels of log per capita income, $x_{it}$ expresses technology accumulation over time and $a_{i0}$ captures initial technology accumulation. $\beta_{it}$ is a transition parameter and, together with the technology accumulation parameter $x_{it}$, it is assumed to be homogeneous between countries in the neoclassical theory.

\citeauthor{phillips2007transition} assume that $\beta_{it}$ is an increasing function of the technological progress parameter $x_{it}$ and therefore can explain divergence and income traps among countries (or regions).

If we assume $x_{it}t$ to contain both idiosyncratic and shared elements across economies and we express equation \eqref{eq3} as
\begin{equation} \label{eq4} y_{it} = \left(\frac{ y_i^* + a_{i0} + (y_{i0} - y_i^*)e^{-\beta_{it}t} + x_{it}t}{\mu_t}\right)\mu_t = \delta_{it}\mu_t  \end{equation}


With $\mu_t$ representing a common growth component.
$\mu_t$ is a broadly defined trend that can have both deterministic and stochastic component and can arise, for example, from knowledge and technology sharing among countries. $\mu_t$ also determines the common growth in the steady state. $\delta_{it}$ then captures how much is a particular economy close to steady state growth represented by $\mu_t$.

Note that by defining \(a_{it} = y_i^* + a_{i0} + (y_{i0} - y_i^*)e^{-\beta_{it}t} \), we can rewrite the loadings as \(\delta_{it} = \frac{a_{it} + x_{it}t}{\mu_t}\). Furthermore, if we represent the common steady state growth element $\mu_t$ by a simple linear deterministic trend \(\mu_t = t\), we can see that
\begin{equation} \label{eq5}\delta_{it} = x_{it} + \frac{a_{it}}{t} \end{equation}  and thus \(\delta_{it} \rightarrow x_i\) as $t$ comes to infinity, assuming that $x_{it}$ converges to $x_{i}$. $\delta_{it}$ therefore plays a key role of a transition parameter and is supposed to have the following structure: \(\delta_{it} = \delta_{i} + \sigma_{it}\xi_{it}\) where \( \sigma_{it} = \frac{\sigma_{i}}{log(t)t^\alpha}\). The parameter $\alpha$ then sets the rate at which \(\delta_{it} \rightarrow \delta_{i}\) with \(t \rightarrow \infty\) and can be interpreted as the speed of convergence. In particular, the convergence of $\delta_{it}$ to $\delta_{i}$ is guaranteed for all \(\alpha \geq 0\). This inequality, therefore, becomes subject of the null hypothesis of the test below together with the condition of shared value of $\delta_{i}$ across cross-sections: 

\begin{equation} \label{eq6} H_0: \delta_{i} = \delta \quad \& \quad \alpha \geq 0 \end{equation} Thus, we test the overall relative convergence among the cross sections, with the alternative allowing both overall divergence and club convergence.
For the testing procedure as well as modelling of the transition parameter $\delta_{it}$ is then used following formula:
\begin{equation}\label{eq8}h_{it} = \frac{y_{it}}{N^{-1}\sum\limits_{i=1}^Ny_{it}} = \frac{\delta_{it}}{N^{-1}\sum\limits_{i=1}^N\delta_{it}}\end{equation}
This formula traces the trajectory of each cross-section $i$ relative to the club's average and is thus called by \citet{phillips2009economic} the relative transition path. It also reflects any divergence of the individual unit $i$ from the common trend $\mu_t$. While individual transition paths $h_{it}$ may be various, including transitional or permanent divergence, the ultimate growth convergence implies \(h_{it} \rightarrow 1\).

In the convergence test itself, the authors focus on cross-sectional convergence of individual $\delta_{it}$. Mainly due to the ease of calculation from the data, the authors concentrate on $h_{t}$, rather than $\delta_{it}$ itself, in the test derivation and they compute mean square cross-sectional "transition differential" of $h_{it}$:
\begin{equation}\label{eq9}H_t = N^{-1}\sum\limits_{i=1}^N(h_{it} - 1)^2 \end{equation}
As the ultimate growth convergence implies \(h_{it} \rightarrow 1\), the value $H_{t}$, which can also be interpreted as a quadratic distance of the club from the common limit, has to converge to zero with time going to infinity. If it remains positive, we conclude that convergence did not happen \citep{phillips2009economic}.

The test itself is then motivated by the problem that it is relatively hard to distinguish whether $H_t$ converges to zero or to a constant. \citet{phillips2007transition} therefore developed a model based on following OLS regression, together with a testing procedure introduced later. They first show that under the model specification shown above, the term $H_{t}$ has the following limiting form: \(H_{t} = \frac{A}{log (t)^2t^{2\alpha}}\) as \(t \rightarrow \infty\), which then leads to the final formulation of log $t$ regression:
\begin{equation}\label{eq10}\log\left(\frac{H_1}{H_t}\right)-2\log(\log(t)) = a + b\log(t) + u_t\end{equation}

The test (called the log $t$ convergence test) is then a one-sided $t$-test test of convergence against no or partial convergence. The coefficient $b$ converges in probability to the speed of the convergence parameter $2\alpha$ and the convergence hypothesis is tested using a one-sided $t$-test of the inequality \(\alpha \geq 0 \), using the estimated parameter $b$ with the HAC standard errors.% In more detail, the $t$-statistic of the test parameter converges to either positive infinity in case of \(\alpha > 0 \) or weakly to a standard normal distribution in case of \(\alpha = 0\). Under the alternative, the estimate of $b$ converges to zero, but the $t$-statistics diverges to $-\infty$.

\subsection{The log $t$ test - formation of convergence clubs }
If the hypothesis of overall convergence is rejected, we check for convergence in subgroups of the investigated sample. The clustering procedure follows these steps:
\begin{itemize}
    \item Order the individuals by the amount of last period income (or other variable).
    \item  A core group of $k^{*}$ highest individuals is chosen by maximizing the log $t$ test's statistic $t_{k}$ over the various sizes of $k^{*}$:
    \(k^{*} = arg max_{k}\{t_{k}\}\) subject to min \( \{t_{k}\}> -1.65\). If \(t_{k} \leq -1.65 \) for \(k=2\), the highest individual is dropped and this step is repeated starting from the second highest observation. 
    \item One region at a time is added to the core group formed in the previous step, and the log $t$ test is run again. The respective  $t$-statistics is than compared to the criterion level $c^{*}$. In our case, we choose $c^{*}=0$. If the associated $t$-statistic is greater than  $c^{*}$, we add the individual to the club.
    \item We run the log $t$ test for all the remaining observations; if they fulfill \(t_b > -1.65\) we conclude that they form a second convergence club. If not, we repeat all the previous steps with the remaining observations to see whether we can find convergence clubs among these remaining individuals.
\end{itemize}    

 \(c^{*} = 0\), in the second step of the procedure, plays an important role in the final composition of the convergence clubs, with higher values of $c^*$ meaning a lower probability of including a wrong region in the club. \citet{phillips2009economic} note that $c^{*}$ can vary between 0 and -1.65 and $c^{*}=0$ is considered a very conservative choice, which tends to detect a larger number of clubs than it should. On the other hand, $c^{*}=-1.65$ was recommended when we have a relatively large dataset. As our data starts from various reasons only after 2003, we adopt $c^*= 0$ combined with a club merging procedure suggested by \citet{phillips2009economic} and further elaborated by \citet{bartkowska2012regional}. 

This procedure contains a step-by-step merging of several groups together and testing whether the log $t$ test statistics of this merged group is larger than -1.65. If it is larger, we conclude that these two clubs form a convergence club together. Concretely, we start by merging the first and second clubs together and proceed by adding the following clubs until the null hypothesis of the log $t$ test is rejected. We conclude that all clubs that passed the test form a single convergence club. Subsequently, we continue, starting again from the first club for which this merging hypothesis was rejected. We try to merge it with all remaining clubs, using the same procedure. If the null hypothesis is rejected for the first and second clubs, we leave the first club untouched and start by merging the second with the third club and continue in the same fashion as just described.

%\begin{itemize}
%\item Club 1:\\
%"LU00" "IE05" "CZ01" "BE10" "DE60" "IE06" "RO32" "DE21" "FR10" "DK01"
%\item Club 2:\\
% "PL91" "SE11" "SK01" "NL32" "DE11" "ITH1" "HU11" "FI1B" "DE71" "AT32" "NL31" "AT13" "CY00"
%"DE91" "UKI4" "BE31" "DE50" "AT33" "AT34" "DE25" "DE14" "LT01" "DE23" "BE21"
%\item Club 3: \\
%"DE12" "AT31" "ITC4" "UKJ1" "ITH2" "NL41" "ITC2" "DEA2" "DE30" "BE24" "DE26" "NL33" "DEA1"
% "DE27" "DE22" "AT22" "DE13" "DE24" "FI20" "UKM5" "BG41" "ITH5" "UKI7" "DK04" "DE92" "ITI4"
% "ES30" "SE23" "DK03" "DEA4" "UKD6" "AT21" "NL42" "BE25" "DEB3" "DE73" "EL30" "DEC0" "AT12"
% "MT00" "BE23" "DE94" "DEA5" "DED5" "PL51" "PL41" "RO42" "RO12" "SE33" "NL11" "ITH3" "FI1D"
% "FI1C" "FRK2" "SI04" "ES21" "ITI1" "ITC3" "ITH4" "NL21" "NL22" "DK05" "ITC1" "FI19" "DE72"
% "UKH2" "ES22" "SE22" "ES51" "SE21" "UKJ2" "PT17" "NL34" "FRL0" "FRF1" "DEB1" "DEA3" "UKJ3"
% "SE12" "UKK1" "SE32" "DEF0" "FRY2" "DED2" "FRJ2" "EL42" %"NL23" "BE22" "ES53" "ITI3" "UKG1"
% "ES24" "AT11" "UKM7" "EE00" "FRY1" "SE31" "DEB2" "DEG0" "DE40" "PT30" "DED4" "DE80" "DEE0"
%"CZ06" "PL22" "PL63" "LT02" "PL71" "RO11" "RO22"
%\item Club 4: \\
%"FRM0" "ES23" "FRG0" "DK02" "FRI1" "ITI2" "UKI6" "UKH1" "ES70" "PT15" "FRF2" "UKM8" "CZ02"
%"NL12" "UKD3" "UKE2" "NL13" "FRC1" "FRK1" "UKF2" "ITF1" "DE93" "UKJ4" "FRD2" "UKL2" "FRE1"
%"BE33" "FRB0" "UKM6" "FRH0" "UKH3" "FRI3" "UKE4" "CZ03" "UKD1" "ITF5" "ES41" "CZ05" "BE35"
%"UKG3" "ES11" "CZ07" "CZ08" "LV00" "HU22" "PL21" "SK02" "PL92" "PL43" "RO31" "PL52" "RO41"
%\item Club 5:\\
%"FRJ1" "FRC2" "ES13" "UKE1" "FRD1" "UKK2" "UKF1" "SI03" "UKD4" %"FRF3" "ES52" "FRE2" "UKN0"
%"UKC2" "ES12" "UKG2" "ITG2" "UKK4" "BE32" "BE34" "IE04" "UKD7" %"UKI5" "FRI2" "ITF2" "EL43"
%"UKF3" "PT18" "EL62" "ES62" "HU21" "ES63" "EL64" "UKM9" "HR04" %"UKK3" "ES42" "UKE3" "CZ04"
%"PT16" "HR03" "ITF4" "EL65" "UKC1" "PT11" "ES61" "UKL1" "ITF3" %"ES43" "ES64" "PL42" "ITG1"
%"HU12" "EL53" "SK03" "PL61" "ITF6" "EL52" "EL61" "HU33" "PL84" %"EL63" "PL72" "EL41" "PL82"
%"EL51" "PL81" "HU23" "PL62" "SK04" "EL54" "HU31" "HU32" "RO21" %"BG33" "BG34" "BG42" "BG32"
%"BG31"

%\end{itemize}

\subsection{Club membership (2019)}
\begin{itemize}
\item Club 1:\\
Région de Bruxelles-Capitale / Brussels Hoofdstedelijk Gewest, Praha, Hovedstaden, Oberbayern, Hamburg, Southern, Eastern and Midland, Île de France, Luxembourg, Bucuresti - Ilfov
\item Club 2:\\
Prov. Antwerpen, Prov. Brabant wallon, Stuttgart, Tübingen   , Oberpfalz, Mittelfranken, Bremen, Darmstadt, Braunschweig, Provincia Autonoma di Bolzano/Bozen, Kypros, Sostines regionas, Budapest, Utrecht, Noord-Holland, Wien, Salzburg, Tirol, Vorarlberg, Warszawski stoleczny, Bratislavský kraj, Helsinki-Uusimaa, Stockholm, Inner London - East
\item Club 3:\\
Prov. Limburg (BE), Prov. Oost-Vlaanderen, Prov. Vlaams-Brabant, Prov. West-Vlaanderen, Yugozapaden, Jihovýchod, Syddanmark, Midtjylland, Nordjylland, Karlsruhe, Freiburg, Niederbayern, Oberfranken, Unterfranken, Schwaben, Berlin, Brandenburg, Gießen, Kassel, Mecklenburg-Vorpommern, Hannover, Weser-Ems, Düsseldorf, Köln, Münster, Detmold, Arnsberg, Koblenz, Trier, Rheinhessen-Pfalz, Saarland, Dresden, Chemnitz, Leipzig, Sachsen-Anhalt, Schleswig-Holstein,Thüringen, Eesti, Attiki, Notio Aigaio, País Vasco, Comunidad Foral de Navarra
, Aragón, Comunidad de Madrid, Cataluña, Illes Balears, Alsace,  Midi-Pyrénées, Rhône-Alpes, Provence-Alpes-Côte d'Azur, Guadeloupe, Martinique, Piemonte, Valle d'Aosta/Vallée d'Aoste, Liguria, Lombardia, Provincia Autonoma di Trento, Veneto, Friuli-Venezia Giulia, Emilia-Romagna, Toscana, Marche, Lazio, Vidurio ir vakaru Lietuvos regionas, Malta, Groningen, Overijssel, Gelderland, Flevoland, Zuid-Holland, Zeeland, Noord-Brabant, Limburg (NL), Burgenland (AT), Niederösterreich, Kärnten, Steiermark, Oberösterreich, Slaskie, Wielkopolskie, Dolnoslaskie, Pomorskie, Lódzkie, Área Metropolitana de Lisboa, Regiao Autónoma da Madeira (PT), Nord-Vest, Centru, Sud-Est, Vest, Zahodna Slovenija, Länsi-Suomi, Etelä-Suomi, 	
Pohjois- ja Itä-Suomi, Åland, Östra Mellansverige, Småland med öarna, Sydsverige, Västsverige, Norra Mellansverige, Mellersta Norrland, Övre Norrland, Cheshire, Herefordshire, Worcestershire and Warwickshire, Bedfordshire and Hertfordshire, Outer London - West and North West, Berkshire, Buckinghamshire and Oxfordshire, Surrey, East and West Sussex, Hampshire and Isle of Wight, Gloucestershire, Wiltshire and Bath/Bristol area, North Eastern Scotland, Eastern Scotland


\item Club 4:\\
Prov. Liège, Prov. Namur, Strední Čechy, Jihozápad, Severovýchod, Strední Morava, Moravskoslezsko, Sjælland, Lüneburg, Galicia, La Rioja, Castilla y León, Canarias, Centre - Val de Loire, Bourgogne, Haute-Normandie, Nord-Pas-de-Calais, Champagne-Ardenne, Pays-de-la-Loire, Bretagne, Aquitaine, Poitou-Charentes, Auvergne, Corse, Umbria, Abruzzo, Basilicata, Latvija, Nyugat-Dunántúl, Friesland (NL), Drenthe, Malopolskie, Lubuskie, Opolskie, Mazowiecki regionalny, Algarve, Sud - Muntenia, Sud-Vest Oltenia, Západné Slovensko, Cumbria, Greater Manchester, North Yorkshire, West Yorkshire, Leicestershire, Rutland and Northamptonshire, West Midlands, East Anglia, Essex, Outer London - South, Kent, East Wales, Highlands and Islands, West Central Scotland

\item Club 5:\\
	
Prov. Hainaut, Prov. Luxembourg (BE), Severozapaden, Severen tsentralen, Severoiztochen, Yugoiztochen, Yuzhen tsentralen, Severozápad, Northern and Western, Voreio Aigaio, Kriti, Anatoliki Makedonia, Thraki, Kentriki Makedonia, Dytiki Makedonia, Ipeiros, Thessalia, Ionia Nisia, Dytiki Ellada, Sterea Ellada, Peloponnisos, Principado de Asturias, Cantabria, Castilla-la Mancha, Extremadura, Comunitat Valenciana, Andalucía, Región de Murcia, Ciudad de Ceuta, Ciudad de Melilla, Franche-Comté, Basse-Normandie, Picardie, Lorraine, Limousin, Languedoc-Roussillon, Jadranska Hrvatska, Kontinentalna Hrvatska (NUTS 2016), Molise, Campania, Puglia, Calabria, Sicilia, Sardegna, Pest, Közép-Dunántúl, Dél-Dunántúl, Észak-Magyarország, Dél-Alföld, Észak-Alföld, Zachodniopomorskie, Kujawsko-Pomorskie, Warminsko-Mazurskie, Swietokrzyskie, Lubelskie, Podkarpackie, Podlaskie, Norte, Centro (PT), Alentejo, Nord-Est, Vzhodna Slovenija, Stredné Slovensko, Východné Slovensko, Tees Valley and Durham, Northumberland and Tyne and Wear, Lancashire, Merseyside, East Yorkshire and Northern Lincolnshire, South Yorkshire, Derbyshire and Nottinghamshire, Lincolnshire, Shropshire and Staffordshire, Outer London - East and North East, Dorset and Somerset, Cornwall and Isles of Scilly, Devon, West Wales and The Valleys, Southern Scotland, Northern Ireland


\end{itemize}


\subsection{Club membership (2015)}
\begin{itemize}
\item Club 1:\\
Wien, Salzburg, Région de Bruxelles-Capitale / Brussels Hoofdstedelijk Gewest, Praha, Stuttgart, Oberbayern, Hamburg, Darmstadt, Hovedstaden, Helsinki-Uusimaa, Île de France, Groningen, Noord-Holland, Bucuresti - Ilfov, Stockholm, Bratislavský kraj, Inner London - East 
\item Club 2:\\
Oberösterreich, Tirol, Vorarlberg, Prov. Antwerpen \& Vlaams-Brabant, Prov. Brabant Wallon, Kypros, Karlsruhe, Freiburg, Tübingen, Niederbayern, Oberpfalz, Oberfranken, Mittelfranken, Unterfranken, 
Schwaben, Berlin, Bremen, Braunschweig, Düsseldorf, Köln, Rheinhessen-Pfalz, Southern and Eastern, Valle d'Aosta/Vallée d'Aoste, Lombardia, Provincia Autonoma di Bolzano/Bozen, Provincia Autonoma di Trento, Emilia-Romagna, Utrecht, Zuid-Holland, Noord-Brabant, Mazowieckie, Berkshire, Buckinghamshire and Oxfordshire, North Eastern Scotland 
\item Club 3: \\
Burgenland (AT), Niederösterreich, Kärnten, Steiermark,
Prov. Limburg (BE), Prov. Oost-Vlaanderen, Prov. West-Vlaanderen, Yugozapaden, Jihovýchod, Brandenburg, Gießen, Kassel, Mecklenburg-Vorpommern, Hannover, Lüneburg, Weser-Ems, Münster, Detmold, Arnsberg, Koblenz, Trier, Saarland, Dresden, Chemnitz, Leipzig,        Sachsen-Anhalt, Schleswig-Holstein, Thüringen, Syddanmark, Midtjylland, Eesti, Attiki, Notio Aigaio, País Vasco, Comunidad Foral de Navarra, Aragón, Comunidad de Madrid, Cataluña, Canarias (ES), Länsi-Suomi, Etelä-Suomi, Alsace, Pays de la Loire, Midi-Pyrénées, Rhône-Alpes,      Provence-Alpes-Côte d'Azur, Corse, Közép-Magyarország, Piemonte, Liguria, Veneto, Friuli-Venezia Giulia, Toscana, Marche, Lazio, Lietuva, Friesland (NL), Overijssel, Gelderland, Flevoland, Zeeland, Limburg (NL), Lódzkie, Slaskie, Wielkopolskie, Dolnoslaskie, Pomorskie, Área Metropolitana de Lisboa, Centru, Sud-Est, Vest, Östra Mellansverige, Småland med öarna, Sydsverige, Västsverige, Norra Mellansverige, Mellersta Norrland, Övre Norrland, Zahodna Slovenija, Západné Slovensko, Cumbria, Cheshire (NUTS 2006), East Anglia, Bedfordshire and Hertfordshire, Outer London - West and North West, Surrey, East and West Sussex, Hampshire and Isle of Wight, Gloucestershire, Wiltshire and Bristol/Bath area 
\item Club 4: \\
 Prov. Liège, Střední Čechy, Jihozápad, Severozápad, Severovýchod, Střední Morava, Moravskoslezsko, Voreio Aigaio, Kriti, Galicia, Principado de Asturias, Cantabria, La Rioja, Castilla y León,                    Comunidad Valenciana, Illes Balears, Región de Murcia, Champagne-Ardenne, Picardie, Haute-Normandie, Centre (FR), Basse-Normandie,  Bourgogne, Lorraine, Franche-Comté, Bretagne, Poitou-Charentes, Aquitaine,          Limousin, Auvergne, Languedoc-Roussillon, Közép-Dunántúl, Nyugat-Dunántúl, Border, Midland and Western, Abruzzo,                Molise, Basilicata, Sardegna, Umbria, Latvija, Drenthe,                 Malopolskie, Lubelskie, Swietokrzyskie, Podlaskie,                  Zachodniopomorskie, Lubuskie, Opolskie, Kujawsko-Pomorskie, Centro (PT), Nord-Vest, Sud - Muntenia, Sud-Vest Oltenia, Vzhodna Slovenija, Stredné Slovensko, Východné Slovensko, Tees Valley and Durham, Northumberland and Tyne and Wear,  Greater Manchester, Lancashire, Merseyside, East Yorkshire and Northern Lincolnshire, North Yorkshire, South Yorkshire,  West Yorkshire, Derbyshire and Nottinghamshire, Leicestershire, Rutland and Northamptonshire, Lincolnshire, Herefordshire, Worcestershire and Warwickshire, Shropshire and Staffordshire, West Midlands, Essex, Outer London - East and North East, Outer London - South, Kent, Dorset and Somerset, Cornwall and Isles of Scilly, Devon, East Wales, Eastern Scotland, South Western Scotland, Highlands and Islands, Northern Ireland (UK)
\item Club 5:\\
Severozapaden, Severen tsentralen, Severoiztochen, Yugoiztochen, Yuzhen tsentralen, Kentriki Makedonia, Castilla-la Mancha, Extremadura, Andalucía, Dél-Dunántúl, Észak-Magyarország, Észak-Alföld, Dél-Alföld, Campania, Puglia, Calabria, Sicilia, Podkarpackie, Warminsko-Mazurskie, Norte, Nord-Est, West Wales and The Valleys

\end{itemize}



%\begin{figure}%
%\centering 
%  {\includegraphics[scale = 0.65]{map_4062.jpeg} }
%  \caption{Convergence clubs and diverging regions (2015)}
%  \label{clubs_graphic_2015}
%  \end{figure}  
  
  



\begin{table}[!htbp] \centering 
 \caption{Convergence club classification before merging} 
  \label{Table_clubs2} 
 \scalebox{0.85}{
\begin{tabular}{@{\extracolsep{5pt}}lcll} 
\\[-1.8ex]\hline 
\hline \\[-1.8ex] 
Club & \multicolumn{1}{c}{N} & \multicolumn{1}{c}{log(t)} & \multicolumn{1}{c}{t value}  \\ 
\hline \\[-1.8ex] 
Club 1 & 10 &  0.0194  & 0.113 \\ 
Club 2 & 23 & 0.146 & 0.930  \\ 
Club 3 & 21 & .168   & 1.95   \\ 
Club 4 & 27 & 0.101  & 0.987  \\ 
Club 5 & 63 & 0.0919 & 0.746  \\
Club 6 & 52 & 0.163 & 1.53  \\ 
Club 7 & 79 & -0.0963 & -0.968  \\ 
\hline \\[-1.8ex]
%\textit{Note:}  & \multicolumn{1}{r}{$^{**}$p$<$0.05} \\

\end{tabular}
}
\end{table}


\begin{table}[!htbp] \centering 
    \caption{Convergence club classification before merging (2015) } \label{Table_clubs2_2015}
 \scalebox{0.85}{
\begin{tabular}{@{\extracolsep{5pt}}lcll} 
\\[-1.8ex]\hline 
\hline \\[-1.8ex] 
Club & \multicolumn{1}{c}{N} & \multicolumn{1}{c}{log(t)} & \multicolumn{1}{c}{t value}  \\ 
\hline \\[-1.8ex] 
Club 1 & 17 &  0.26612  & 2.693** \\ 
Club 2 & 36 & -0.0148 & -0.088 \\ 
Club 3 & 25 &  0.3725  & 2.334** \\ 
Club 4 & 24 & 0.1924 & 0.974 \\ 
Club 5 & 30 & 0.02397 & 0.223  \\ 
Club 6 & 14 & 0.04248 & 0.328 \\ 
Club 7 & 28 & 0.08075  & 0.720 \\ 
Club 8 & 30 & 0.2519  & 1.895** \\ 
Club 9 & 40 & 0.1067   & 0.692  \\ 
Club 10 & 29 &  -0.09851   & -0.638 \\ 
\hline \\[-1.8ex]
\textit{Note:}  & \multicolumn{1}{r}{ $^{**}$p$<$0.05} \\
\end{tabular} 
}
\end{table}



% Table created by stargazer v.5.2.2 by Marek Hlavac, Harvard University. E-mail: hlavac at fas.harvard.edu
% Date and time: čt, srp 10, 2023 - 22:04:17
\begin{table}[!htbp] \centering 
  \caption{Descriptive statistics of the main explanatory variables based on the 267 regions with neighbours (year 2008).} 
  \label{descriptive_vars_filtered} 
\begin{tabular}{@{\extracolsep{5pt}}lcccccc} 
\\[-1.8ex]\hline 
\hline \\[-1.8ex] 
Statistic  & \multicolumn{1}{c}{Mean} & \multicolumn{1}{c}{St. Dev.} & \multicolumn{1}{c}{Min} & \multicolumn{1}{c}{Pctl(25)} & \multicolumn{1}{c}{Pctl(75)} & \multicolumn{1}{c}{Max} \\ 
\hline \\[-1.8ex] 
Spec. Agriculture & 0.055 & 0.065 & 0.001 & 0.017 & 0.063 & 0.428 \\ 
Spec. Industry  & 0.192 & 0.072 & 0.056 & 0.139 & 0.238 & 0.388 \\ 
Spec. Trade, Accommodation  & 0.238 & 0.037 & 0.147 & 0.214 & 0.255 & 0.453 \\ 
Spec. Finances  & 0.027 & 0.014 & 0.008 & 0.017 & 0.034 & 0.106 \\ 
Spec. Other Services & 0.048 & 0.017 & 0.010 & 0.037 & 0.058 & 0.113 \\ 
%spec\_A\_delta & 260 & $-$0.009 & 0.021 & $-$0.107 & $-$0.011 & $-$0.0001 & 0.060 \\ 
%spec\_BE\_delta & 274 & $-$0.015 & 0.021 & $-$0.072 & $-$0.030 & $-$0.004 & 0.071 \\ 
%spec\_GI\_delta & 276 & 0.002 & 0.026 & $-$0.061 & $-$0.015 & 0.015 & 0.139 \\ 
%spec\_K\_delta & 270 & $-$0.002 & 0.006 & $-$0.024 & $-$0.006 & 0.001 & 0.017 \\ 
%spec\_L\_delta & 211 & 0.0004 & 0.004 & $-$0.028 & $-$0.001 & 0.002 & 0.014 \\ 
%spec\_RU\_delta & 275 & 0.002 & 0.011 & $-$0.060 & $-$0.004 & 0.008 & 0.051 \\ 
%pop\_growth & 276 & 3.386 & 3.103 & 0.631 & 1.111 & 4.943 & 20.960 \\ 
%Population Growth & 3.971 & 6.998 & $-$18.900 & $-$0.300 & 7.900 & 29.300 \\ 
%delta\_pop\_growth & 221 & $-$1.641 & 6.658 & $-$31.900 & $-$5.000 & 2.400 & 34.100 \\ 
%ter\_edu & 276 & 0.294 & 0.088 & 0.118 & 0.229 & 0.349 & 0.596 \\ 
Tertiary Edu. Share & 0.146 & 0.094 & 0.030 & 0.063 & 0.210 & 0.446 \\ 
%delta\_ter\_edu & 204 & 0.262 & 0.097 & 0.064 & 0.196 & 0.321 & 0.560 \\ 
%scientists\_share & 250 & 0.006 & 0.005 & 0.0004 & 0.003 & 0.008 & 0.026 \\ 
Share of Scientists & 0.006 & 0.004 & 0.001 & 0.003 & 0.007 & 0.026 \\ 
%delta\_scientist\_share & 250 & 0.002 & 0.002 & $-$0.005 & 0.0004 & 0.003 & 0.011 \\ 
%eqi & 230 & 0.169 & 1.006 & $-$2.398 & $-$0.654 & 0.930 & 2.120 \\ 
Governance Quality & 0.173 & 1.012 & $-$2.398 & $-$0.665 & 0.930 & 2.120 \\  
%delta\_eqi & 230 & 0.066 & 0.409 & $-$1.127 & $-$0.173 & 0.378 & 1.262 \\ 
%BS\_avg & 276 & 0.125 & 0.057 & 0.022 & 0.086 & 0.155 & 0.355 \\ 
Business Services & 0.120 & 0.060 & 0.000 & 0.076 & 0.154 & 0.329 \\ 
%delta\_BS & 276 & 0.014 & 0.029 & $-$0.083 & 0.001 & 0.027 & 0.165 \\ 
%y\_o & 260 & 0.869 & 0.264 & 0.412 & 0.703 & 0.981 & 2.355 \\ 
%init\_y\_o & 0.954 & 0.291 & 0.415 & 0.763 & 1.105 & 2.434 \\ 
%delta\_y\_o & 260 & $-$0.168 & 0.139 & $-$1.061 & $-$0.227 & $-$0.077 & 0.227 \\ 
%gfcf & 235 & 8.752 & 1.049 & 5.558 & 8.142 & 9.456 & 11.895 \\ 
Investment & 8.904 & 0.919 & 6.139 & 8.305 & 9.489 & 11.825 \\  
%delta\_gfcf & 235 & 0.072 & 0.402 & $-$1.197 & $-$0.091 & 0.305 & 1.562 \\ 
%wreg & 276 & 114.373 & 107.551 & 13.209 & 46.769 & 136.351 & 660.924 \\ 
Initial GDP level & 10.048 & 0.388 & 8.868 & 9.870 & 10.284 & 11.115 \\ 
%patents & 217 & 108.209 & 126.982 & 0.342 & 11.079 & 150.735 & 612.325 \\ 
Patent activity  & 115.381 & 132.421 & 0.177 & 13.712 & 160.176 & 626.102 \\ 
%delta\_patents & 213 & $-$20.462 & 34.642 & $-$166.860 & $-$34.003 & 0.615 & 102.204 \\ 
%init\_inno & 0.949 & 0.460 & 0.000 & 0.530 & 1.314 & 1.916 \\ 
%init\_migration & 3.177 & 5.132 & $-$10.300 & $-$0.175 & 5.875 & 21.300 \\ 
Highest Neighbour Income & 10.262 & 0.411 & 9.159 & 10.071 & 10.513 & 11.826 \\ 
\hline \\[-1.8ex] 
\end{tabular} 
\end{table} 


\begin{table}[!htbp] \centering 
\resizebox{0.75 \textwidth}{!}{\begin{minipage}{\textwidth}
  \captionsetup{justification=centering}

  \caption{Regression results: Logistic regression with Sectoral and Human capital variables. Merged Convergence Clubs, FAMD imputation and robust standard errors.} 
  \label{model_famd_filtered_spec1}
\centering
\begin{tabular}{@{\extracolsep{5pt}}lc} 
\\[-1.8ex]\hline 
\hline \\[-1.8ex] 
 & \multicolumn{1}{c}{\textit{Dependent variable:}} \\ 
\cline{2-2} 
\\[-1.8ex] & Club \\ 
\hline \\[-1.8ex] 
 Initial GDP level & $-$7.492$^{***}$ \\ 
  & (0.860) \\ 
  & \\ 
 Investment & 0.002 \\ 
  & (0.208) \\ 
  & \\ 
 Spec. Agriculture  & $-$14.049$^{***}$ \\ 
  & (3.769) \\ 
  & \\ 
 Spec. Finances & $-$36.010$^{***}$ \\ 
  & (13.605) \\ 
  & \\ 
 Spec. Industry & $-$15.963$^{***}$ \\ 
  & (2.927) \\ 
  & \\ 
 Spec. Trade, Accommodation & $-$3.659 \\ 
  & (4.369) \\ 
  & \\ 
 Spec. Other Services & 17.237 \\ 
  & (11.817) \\ 
  & \\ 
 Business Services & $-$6.603$^{*}$ \\ 
  & (3.529) \\ 
  & \\ 
 Share of Scientists  & $-$152.849$^{***}$ \\ 
  & (39.670) \\ 
  & \\ 
 Tertiary Edu. Share & 3.504$^{**}$ \\ 
  & (1.599) \\ 
  & \\ 
\hline \\[-1.8ex] 
Observations & 267 \\ 
\hline 
\hline \\[-1.8ex] 
\textit{Note:}  & \multicolumn{1}{r}{$^{*}$p$<$0.1; $^{**}$p$<$0.05; $^{***}$p$<$0.01} \\ 
\end{tabular}
\end{minipage}
}
\end{table} 


\begin{table}[!htbp] \centering 
\resizebox{0.75 \textwidth}{!}{\begin{minipage}{\textwidth}
  \captionsetup{justification=centering}
  \caption{Regression results: Logistic regression with with Geographic determinants. Merged Convergence Clubs, FAMD imputation and robust standard errors.} 
  \label{model_famd_filtered_spec2}
\centering
\begin{tabular}{@{\extracolsep{5pt}}lc} 
\\[-1.8ex]\hline 
\hline \\[-1.8ex] 
 & \multicolumn{1}{c}{\textit{Dependent variable:}} \\ 
\cline{2-2} 
\\[-1.8ex] & Club \\ 
\hline \\[-1.8ex] 
 Initial GDP level & $-$7.633$^{***}$ \\ 
  & (0.786) \\ 
  & \\ 
 Investment  & 0.180 \\ 
  & (0.212) \\ 
  & \\ 
 Capital & $-$0.977$^{***}$ \\ 
  & (0.362) \\ 
  & \\ 
 Metro area & $-$1.073$^{***}$ \\ 
  & (0.384) \\ 
  & \\ 
 Governance Quality & 0.248 \\ 
  & (0.187) \\ 
  & \\ 
 Highest Neighbour Income & 1.236$^{***}$ \\ 
  & (0.388) \\ 
  & \\ 
 Border Region & $-$1.257$^{***}$ \\ 
  & (0.277) \\ 
  & \\ 
\hline \\[-1.8ex] 
Observations & 267 \\ 
\hline 
\hline \\[-1.8ex] 
\textit{Note:}  & \multicolumn{1}{r}{$^{*}$p$<$0.1; $^{**}$p$<$0.05; $^{***}$p$<$0.01} \\ 
\end{tabular}
\end{minipage}
}
\end{table} 

% Table created by stargazer v.5.2.2 by Marek Hlavac, Harvard University. E-mail: hlavac at fas.harvard.edu
% Date and time: st, čvc 13, 2022 - 23:17:43
\begin{table}[!htbp] \centering 
\resizebox{0.75 \textwidth}{!}{\begin{minipage}{\textwidth}
  \captionsetup{justification=centering}
  \caption{\textbf{Regression results: Logistic regression with merged Convergence Clubs, FAMD imputation and robust standard errors.}} 
  \label{model_famd_neigh_filtered} 
\centering
\begin{tabular}{@{\extracolsep{5pt}}lc} 
\\[-1.8ex]\hline 
\hline \\[-1.8ex] 
 & \multicolumn{1}{c}{\textit{Dependent variable:}} \\ 
\cline{2-2} 
\\[-1.8ex] &   \\ 
\hline \\[-1.8ex] 
 Initial GDP level & $-$9.555$^{***}$ \\ 
  & (1.081) \\ 
  & \\ 
 Investment & 0.095 \\ 
  & (0.245) \\ 
  & \\ 
 Spec. Agriculture  & $-$17.175$^{***}$ \\ 
  & (4.322) \\ 
  & \\ 
 Spec. Finances & $-$26.968$^{*}$ \\ 
  & (15.169) \\
  & \\ 
 Spec. Industry & $-$19.760$^{***}$ \\ 
  & (3.468) \\ 
  & \\ 
 Spec. Trade, Accommodation & $-$3.790 \\ 
  & (4.763) \\ 
  & \\ 
 Spec. Other Services  & 12.003 \\ 
  & (12.120) \\  
  & \\ 
 Business Services & $-$8.381$^{**}$ \\ 
  & (3.880) \\ 
  & \\ 
 Share of Scientists & $-$94.294$^{**}$ \\ 
  & (45.829) \\  
  & \\ 
 Tertiary Edu. Share  & 2.688 \\ 
  & (1.671) \\  
  & \\ 
 Capital & $-$1.713$^{***}$ \\ 
  & (0.455) \\ 
  & \\ 
 Metro area  & $-$0.631 \\ 
  & (0.446) \\  
  & \\ 
 Governance Quality & $-$0.203 \\ 
  & (0.252) \\
  & \\
 Highest Neighbour Income & 1.258$^{***}$ \\ 
  & (0.475) \\ 
  & \\ 
 Border Region & $-$1.123$^{***}$ \\ 
  & (0.310) \\ 
  & \\ 
\hline \\[-1.8ex] 
Observations & 276 \\ 
\hline 
\hline \\[-1.8ex] 
\textit{Note:}  &\multicolumn{1}{r}{$^{*}$p$<$0.1; $^{**}$p$<$0.05; $^{***}$p$<$0.01} \\ 
\end{tabular}
\end{minipage}
}
\end{table} 





% Table created by stargazer v.5.2.2 by Marek Hlavac, Harvard University. E-mail: hlavac at fas.harvard.edu
% Date and time: st, čvc 06, 2022 - 17:57:02
\begin{table}[!htbp] \centering 
\resizebox{0.75\textwidth}{!}{\begin{minipage}{\textwidth}
    \caption{\textbf{Marginal effects: Logistic regression with merged Clubs, FAMD imputation and robust standard errors (variables in 2008 levels).}}
  \label{famd_selected_marginal} 
\begin{tabular}{@{\extracolsep{5pt}} lccccc} 
\\[-1.8ex]\hline 
\hline \\[-1.8ex] 
 & Club 1 & Club 2 & Club 3 & Club 4 & Club 5 \\ 
\hline \\[-1.8ex] 
% log\_init\_gdp & $0.010$ & $0.102$ & $1.776$ & $$-$1.052$ & $$-$0.835$ \\ 
Initial GDP level & $0.010$$^{*}$ & $0.102$$^{***}$ & $1.776$$^{***}$ & $ $-$1.052$$^{***}$ & $ $-$0.835$$^{***}$ \\ 
& (0.006) &  (0.033) & (0.245) & (0.197) & (0.134)\\
&\\
% init\_gfcf & $0$ & $$-$0.004$ & $$-$0.063$ & $0.037$ & $0.030$ \\  
Investment & $0.000$ & $ $-$0.004$ & $ $-$0.063$ & $0.037$ & $0.030$ \\ 
% spec\_A & $0.021$ & $0.213$ & $3.705$ & $$-$2.196$ & $$-$1.743$ \\
& (0.000) &  (0.003) & (0.054) & (0.033) & (0.025)\\
&\\
\multicolumn{6}{c}{\textit{Sectoral Variables}}\\

Spec. Agriculture & $0.021$$^{*}$ & $0.213$$^{**}$ & $3.705$$^{***}$ & $ $-$2.196$$^{**}$ & $ $-$1.743$$^{***}$ \\ 
& (0.012) &  (0.089) & (1.114) & (0.735) & (0.520)\\
&\\
% spec\_K & $0.033$ & $0.342$ & $5.938$ & $$-$3.519$ & $$-$2.793$ \\ 
Spec. Finances  & $0.033$ & $0.342$ & $5.938$$^{**}$ & $ $-$3.519$$^{**}$ & $ $-$2.793$$^{*}$ \\ 
& (0.024) &  (0.210) & (2.840) & (1.673) & (1.443)\\
&\\
%spec\_BE & $0.022$ & $0.231$ & $4.012$ & $$-$2.378$ & $$-$1.887$ \\ 
Spec. Industry & $0.022$$^{*}$ & $0.231$$^{***}$ & $4.012$$^{***}$ & $ $-$2.378$$^{***}$ & $ $-$1.887$$^{***}$ \\ 
& (0.013) &  (0.078) & (0.743) & (0.539) & (0.372)\\
&\\
% spec\_GI & $0.011$ & $0.114$ & $1.982$ & $$-$1.174$ & $$-$0.932$ \\ 
Spec. Trade, Accommodation & $0.011$ & $0.114$$^{*}$ & $1.982$$^{*}$ & $ $-$1.174$$^{*}$ & $ $-$0.932$$^{*}$ \\
& (0.008) &  (0.068) & (1.010) & (0.625) & (0.474)\\
&\\
% spec\_RU & $0.001$ & $0.007$ & $0.123$ & $$-$0.073$ & $$-$0.058$ \\ 
Spec. Other Services & $ 0.001$ & $ 0.007$ & $ 0.123$ & $ $-$0.073$ & $ $-$0.058$ \\
& (0.018) &  (0.184) & (3.198) & (1.895) & (1.505)\\
%BS & $0.007$ & $0.072$ & $1.246$ & $$-$0.738$ & $$-$0.586$ \\ 
&\\
Business Services & $0.007$ & $0.072$ & $1.246$ & $ $-$0.738$ & $ $-$0.586$ \\
& (0.005) &  (0.050) & (0.858) & (0.519) & (0.399)\\
&\\
\multicolumn{6}{c}{\textit{Human Capital Variables}}\\

% init\_scientists\_share & $0.160$ & $1.638$ & $28.472$ & $$-$16.876$ & $$-$13.394$ \\
Share of Scientists & $0.160$$^{*}$ & $1.638$$^{**}$ & $28.472$$^{***}$ & $ $-$16.876$$^{***}$ & $ $-$13.394$$^{***}$ \\
& (0.093) &  (0.683) & (8.804) & (5.625) & (4.263)\\
&\\
%init\_eqi & $0$ & $0.002$ & $0.040$ & $$-$0.023$ & $$-$0.019$ \\ 
Governance Quality & $0.000$ & $0.002$ & $0.040$ & $ $-$0.023$ & $ $-$0.019$ \\
& (0.000) &  (0.003) & (0.058) & (0.034) & (0.027)\\
&\\
% init\_ter\_edu & $$-$0.004$ & $$-$0.046$ & $$-$0.798$ & $0.473$ & $0.375$ \\ 
Tertiary Edu. Share & $ $-$0.004$ & $ $-$0.046$$^{*}$ & $ $-$0.798$$^{**}$ & $0.473$$^{**}$ & $0.375$$^{**}$ \\
& (0.003) &  (0.024) & (0.372) & (0.227) & (0.178)\\
&\\
\multicolumn{6}{c}{\textit{Geographic Determinants}}\\

% capital & $0.003$ & $0.032$ & $0.253$ & $$-$0.181$ & $$-$0.107$ \\  
Capital & $0.003$ & $0.032$$^{*}$ & $0.253$$^{***}$ & $ $-$0.181$$^{***}$ & $ $-$0.107$$^{***}$ \\
& (0.002) &  (0.017) & (0.060) & (0.055) & (0.026)\\
% metro & $0.001$ & $0.008$ & $0.173$ & $$-$0.089$ & $$-$0.092$ \\ 
&\\
Metro area & $0.001$ & $0.008$$^{*}$ & $0.173$$^{*}$ & $ $-$0.089$$^{**}$ & $ $-$0.092$ \\ 
& (0.001) &  (0.005) & (0.093) & (0.044) & (0.056)\\
% border\_region & $0.001$ & $0.010$ & $0.173$ & $$-$0.100$ & $$-$0.084$ \\ 
&\\
Border Region & $0.001$ & $0.010$$^{**}$ & $0.173$$^{**}$ & $ $-$0.100$$^{**}$ & $ $-$0.084$$^{**}$ \\ 
& (0.001) &  (0.005) & (0.070) & (0.042) & (0.035)\\

\hline \hline \\[-1.8ex]
McFadden's $R^{2}$: 0.365 \\
Observations: 276\\
No. of Parameters 14\\
\hline
\end{tabular}
\begin{tablenotes}
\small 
\item Note: $^{*}$p$<$0.1; $^{**}$p$<$0.05; $^{***}$p$<$0.01
\end{tablenotes}
\end{minipage}}
\end{table} 


\begin{table}[!htbp] \centering 
\resizebox{0.8\textwidth}{!}{\begin{minipage}{\textwidth}
  \caption{\textbf{Marginal effects: Logistic regression with merged Clubs, initial values from 2003 are used. Sectoral variables use NACE Rev. 1.1 clasification  FAMD imputation and robust standard errors. 9 regions with no neighbouring regions have been removed.}} 
  \label{famd_selected_marginal_ecosys_robust_filtered} 
\begin{tabular}{@{\extracolsep{5pt}} lccccc} 
\\[-1.8ex]\hline 
\hline \\[-1.8ex] 
 & Club 1 & Club 2 & Club 3 & Club 4 & Club 5 \\ 
\hline \\[-1.8ex] 
%log\_init\_gdp & $0.013$ & $0.112$ & $1.043$ & $$-$0.570$ & $$-$0.599$ \\ 
Initial GDP level & $0.013$$^{**}$ & $0.112$$^{***}$ & $1.043$$^{***}$ & $ $-$0.570$$^{***}$ & $  $-$0.599$$^{***}$ \\
%init\_gfcf & $0.001$ & $$-$0.008$ & $$-$0.075$ & $0.041$ & $0.043$ \\ 
& (0.007) &  (0.034) & (0.208) & (0.135) & (0.125)\\
&\\
Investment & $ $-$0.001$ & $ $-$0.008$$^{*}$  & $ $-$0.075$$^{*}$ & $0.041$$^{*}$ & $0.043$$^{*}$ \\
&\\
& (0.001) &  (0.005) & (0.039) & (0.022) & (0.022)\\
&\\
\multicolumn{6}{c}{\textit{Sectoral Variables}}\\
%spec\_AB & $0.019$ & $0.158$ & $1.472$ & $$-$0.804$ & $$-$0.845$ \\ 
Spec. Agriculture, fishing & $0.019$ & $0.158$ & $1.472$$^{*}$ & $ $-$0.804$ & $ $-$0.845$$^{*}$ \\ 
& (0.013) &  (0.104) & (0.861) & (0.503) & (0.479)\\
&\\
% spec\_CE & $0.036$ & $0.302$ & $2.820$ & $$-$1.540$ & $$-$1.619$ \
Spec. Industry & $0.036$$^{*}$ & $0.302$$^{***}$ & $2.820$$^{***}$ & $ $-$1.540$$^{***}$ & $ $-$1.619$$^{***}$ \\
& (0.019) &  (0.105) & (0.615) & (0.414) & (0.353)\\
&\\
%spec\_GI & $$-$0.004$ & $$-$0.030$ & $$-$0.280$ & $0.153$ & $0.161$ \\ 
Spec. Trade, Accomodation & $ $-$0.004$ & $ $-$0.030$ & $ $-$0.280$ & $0.153$ & $0.161$ \\ 
& (0.014) &  (0.111) & (1.056) & (0.575) & (0.606)\\
&\\
% spec\_JK & $0.077$ & $0.644 & $6.007$ & $$-$3.280$ & $$-$3.448$ \\ 
Spec. Finance, Real estate & $0.077$$^{*}$ & $0.644$$^{***}$ & $6.007$$^{***}$ & $ $-$3.280$$^{***}$ & $ $-$3.448$$^{**}$ \\ 
& (0.042) &  (0.242) & (1.725) & (1.097) & (0.945)\\
&\\
%init\_scientists\_share & $0.230$ & $1.921$ & $17.921$ & $$-$9.787$ & $$-$10.286$ \\ 
\multicolumn{6}{c}{\textit{Human Capital Variables}}\\
Share of Scientists & $0.230$ & $1.921$$^{*}$ & $17.921$$^{**}$ & $ $-$9.787$$^{**}$ & $ $-$10.286$$^{*}$ \\
& (0.159) &  (1.108) & (8.677) & (4.788) & (5.228)\\
&\\
% init\_ter\_edu & $$-$0.010$ & $$-$0.079$ & $$-$0.740$ & $0.404$ & $0.424$ \\ 
Tertiary Edu. Share & $ $-$0.010$ & $ $-$0.079$$^{*}$ & $ $-$0.740$$^{**}$ & $0.404$$^{**}$ & $0.424$$^{**}$ \\
& (0.006) &  (0.042) & (0.359) & (0.200) & (0.210)\\
&\\
\multicolumn{6}{c}{\textit{Geographic Determinants}}\\
% capital & $0.005$ & $0.037$ & $0.195$ & $$-$0.132$ & $$-$0.104$ \\  
Capital & $0.005$ & $0.037$ & $0.195$$^{***}$ & $ $-$0.132$$^{**}$ & $ $-$0.104$$^{***}$ \\
& (0.004) &  (0.023) & (0.066) & (0.056) & (0.036)\\
&\\
% metro & $0.001$ & $0.009$ & $0.092$ & $$-$0.046$ & $$-$0.056$ \\ 
Metro area & $0.001$ & $0.009$ & $0.092$ & $ $-$0.046$ & $ $-$0.056$ \\
& (0.001) &  (0.007) & (0.080) & (0.037) & (0.051)\\
&\\
% neigh\_highest\_income & $$-$0.004$ & $$-$0.029$ & $$-$0.273$ & $0.149$ & $0.157$ \\ 
Highest Neighbour Income & $ $-$0.004$ & $ $-$0.029$$^{**}$ & $ $-$0.273$$^{**}$ & $ 0.149$$^{**}$ & $ 0.157$$^{**}$ \\ 
& (0.002) &  (0.013) & (0.111) & (0.062) & (0.065)\\
&\\
Border region & $0.003$$^{*}$ & $0.029$$^{***}$ & $0.258$$^{***}$ & $ $-$0.132$$^{***}$ & $ $-$0.158$$^{***}$ \\ 
& (0.002) &  (0.010) & (0.067) & (0.038) & (0.043)\\
\hline \hline \\[-1.8ex]
McFadden's $R^{2}$: 0.315 \\
Observations: 267\\
No. of Parameters 12\\
\hline
\end{tabular}
\begin{tablenotes}
\small 
\item Note: $^{*}$p$<$0.1; $^{**}$p$<$0.05; $^{***}$p$<$0.01
\end{tablenotes}
\end{minipage}}
\end{table} 




\begin{figure}%
\centering 
  {\includegraphics[scale = 0.5]{RTC_mer_overall.png} }
  \caption{Average transition paths across merged convergence clubs. Diverging regions are represented by the highest transition curve.}
  \label{paths_overall}
  \end{figure}



\begin{figure}%
    \centering
    \includegraphics[scale = 0.5]{RTC_mer_club2.png}
    \caption{Transition paths for members of the second Club}
    \label{paths2}
\end{figure}


\begin{figure}%
    \centering
    \includegraphics[scale = 0.5]{RTC_mer_club3.png}
    \caption{Transition paths for members of the third Club}
    \label{paths3}
\end{figure}

\begin{figure}%
    \centering
    \includegraphics[scale = 0.5]{RTC_mer_club4.png}
    \caption{Transition paths for members of the forth Club}
    \label{paths4}
\end{figure}


\section{Additional material}

\subsection{Data availability}
Replication package can be found using following to Mendeley data reference:
"Pintera, Jan (2023), “Regional Convergence in the European Union - Factors of Growth Between the Great Recession and the COVID Crisis”, Mendeley Data, V1, doi: 10.17632/mcmkwc2394.1"

\end{document}

